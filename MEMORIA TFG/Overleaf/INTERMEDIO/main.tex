\documentclass[12pt,a4paper]{book}
\usepackage[utf8]{inputenc}
\usepackage[spanish]{babel}
\usepackage{amsmath}
\usepackage{stackrel}
\usepackage{hyperref}
\usepackage[left=2.5cm,right=2.5cm,top=2cm,bottom=2.8cm]{geometry}
\setlength{\parskip}{4mm}

\usepackage[linesnumbered,ruled,vlined]{algorithm2e}

\usepackage{wrapfig}
\usepackage{bm}
\usepackage{url}

\PassOptionsToPackage{hyphens}{url} %Permite que los enlaces URL puedan dividirse en múltiples líneas 

\usepackage{mathtools}
\usepackage{amsfonts}
\usepackage{amssymb}
\usepackage{esvect}

%\usepackage{subfig}

\usepackage{float}
\usepackage{enumitem}

\usepackage{natbib}

\usepackage{amsthm}
\usepackage{graphicx}
\usepackage{xcolor}
\usepackage[nottoc]{tocbibind}

\usepackage{synttree} 

\graphicspath{ {images/} }
\let\olditemize\itemize
\def\itemize{\olditemize\itemsep=0pt }
\setlength{\parindent}{0cm}
\setlist[itemize]{topsep=0pt}
\setlist[enumerate]{topsep=0pt}

\newtheorem{teo}{Teorema}[section]
\newtheorem{cor}[teo]{Corolario}
\newtheorem{defi}[teo]{Definición}
\newtheorem{prop}[teo]{Proposición}
\newtheorem{lema}[teo]{Lema}
\newtheorem{conj}[teo]{Conjetura}
\newtheorem{obs}[teo]{Observación}
\newtheorem{ejem}[teo]{Ejemplo}
\newtheorem{axioma}[teo]{Axioma}

\newcommand\Item[1][]{%
  \ifx\mathbb{R}lax#1\mathbb{R}lax  \item \else \item[#1] \fi
  \abovedisplayskip=-4pt\abovedisplayshortskip=0pt~\vspace*{-\baselineskip}}

\DeclarePairedDelimiter{\abs}{\lvert}{\rvert} %Define un nuevo comando \abs para facilitar la escritura de valores absolutos. Cuando uses \abs{x}, esto se representará como |x| con el tamaño de los delimitadores automáticamente ajustado al tamaño de x.

\newcommand{\mbb}{\mathbb}
\newcommand{\lp}{\ensuremath{\left(}}
\newcommand{\rp}{\ensuremath{\right)}}

\usepackage[acronym]{glossaries}

\usepackage{multicol}
\usepackage{longtable}

\usepackage{listings}
\definecolor{codegreen}{rgb}{0,0.6,0}
\definecolor{codegray}{rgb}{0.5,0.5,0.5}
\definecolor{codepurple}{rgb}{0.58,0,0.82}
\definecolor{backcolour}{rgb}{0.95,0.95,0.92}
\lstdefinestyle{mystyle}{
    backgroundcolor=\color{backcolour},   
    commentstyle=\color{codegreen},
    keywordstyle=\color{magenta},
    numberstyle=\tiny\color{codegray},
    stringstyle=\color{codepurple},
    basicstyle=\ttfamily\footnotesize,
    breakatwhitespace=false,         
    breaklines=true,                 
    captionpos=b,                    
    keepspaces=true,                 
    numbers=left,                    
    numbersep=5pt,                  
    showspaces=false,                
    showstringspaces=false,
    showtabs=false,                  
    tabsize=2
}
\lstset{style=mystyle}


\usepackage{caption}
\usepackage{subcaption}
%\usepackage{glossaries}

%\usepackage{fancyhdr}
%\pagestyle{fancy}
%\fancyhf{} % Limpia la cabecera y pie de página actuales
%\fancyhead[LO,RE]{\nouppercase{\leftmark}} % Cabecera de la página par (izquierda) e impar (derecha)
%\fancyhead[RO,LE]{\thepage} % Número de página en la cabecera derecha (páginas impares) y izquierda (páginas pares)
%\renewcommand{\headrulewidth}{0pt}
%\renewcommand{\chaptermark}[1]{%
%  \markboth{\chaptername\ \thechapter.\ #1}{}%
%}








\makeatletter
\newcommand{\dotminus}{\mathbin{\text{\@dotminus}}}

\newcommand{\@dotminus}{%
  \ooalign{\hidewidth\raise1ex\hbox{.}\hidewidth\cr$\m@th-$\cr}%
}
\makeatother


\usepackage{mdframed}
\newmdenv[leftline=false,topline=false]{topbot}
            
\author{Pablo Jiménez Poyatos}
\title{TFG}
\date{Julio 2024}


\newacronym{uam}{UAM}{Urban Air Mobility}









\begin{document}

\pagenumbering{gobble}


\begin{titlepage}
		\centering
		
		{ \Large UNIVERSIDAD COMPLUTENSE DE MADRID}
		
		{ \Large \textbf{FACULTAD DE CIENCIAS MATEMÁTICAS}}
		\vspace{0.8cm}
		
		{ \large DEPARTAMENTO DE CIENCIAS DE LA  COMPUTACIÓN}
		\vspace{1cm}
		
		\vspace{0.6cm}
		
		\graphicspath{ {images/} }
		%%%%Logo Complutense%%%%%
		\includegraphics[width=0.35\textwidth]{img/ucm.png} 
		\vspace{0.4cm}
		
        {\Large \textbf{TRABAJO DE FIN DE GRADO}}
		
		\vspace{8mm}
        {\huge \bfseries Algoritmos de Aprendizaje Automático aplicados a problemas de Ciberseguridad\par}
		\vspace{1cm}

		{\large Presentado por: Pablo Jiménez Poyatos}
		
		{\large Dirigido por: Luis Fernando Llana Diaz}
		
		\vspace{1.5cm}
		{\large Grado en Matemáticas}
		
		{\large Curso académico 2023-24}
\end{titlepage}

\thispagestyle{empty}
\clearpage
\setcounter{page}{1}


\newpage
\begin{center}
   {\bf Agradecimientos} 
\end{center}




 

\thispagestyle{empty}
\clearpage
\setcounter{page}{1}


\newpage
\begin{center}
   {\bf Resumen} 
\end{center}
   

\vspace{0.6 cm}
\textsl{\textbf{Palabras clave:} } 



\begin{center}
   {\bf Abstract} 
\end{center}



\vspace{0.6 cm}
\textsl{\textbf{Keywords:} } 


\newpage
\tableofcontents

\newpage
\clearpage
\pagenumbering{arabic}


\chapter{Introducción} \label{Capitulo 1}
      



\section{Motivación y objetivos del trabajo} \label{Sec:1_1}

   
    
\section{Contexto y antecedentes del trabajo} \label{Sec:1_2}

\subsection{Movilidad Aérea Urbana} \label{Subsubsec: 1_2_1}
  
\subsection{Vertipuertos} \label{Subsec: 1_2_2}

\subsection{Optimización del diseño de un vertipuerto} \label{Subsec: 1_2_3}

\subsection{Optimización heurística} \label{Subsubsec: 1_2_4}



\section{Estructura de la memoria} \label{Subsubsec: 1_3}
  


\section{Contribuciones} \label{Subsec: 1_4}




\chapter{Optimización Heurística} \label{Capitulo_2}


 

\section{De la realidad a la solución} \label{Subsec: 3_1}



\section{Algoritmos exactos de optimización matemática} \label{Subsec: 3_2}



\section{Optimización heurística} \label{Subsec: 3_3}

\subsection{Heurística y Metaheurística}

\subsection{Heurísticas evolutivas} \label{Subsubsec: 3_3_2}



\section{Algoritmos genéticos} \label{Subsec: 3_4}




\chapter{Optimización del Diseño de un Vertipuerto} \label{Capitulo_3}


    

\section{Problema de Diseño de un Vertipuerto}

\subsection{Caracterización de la solución}

\subsection{Consideraciones del diseño de optimización}

\subsection{Definición del problema de optimización}


 
\section{Modelo de optimización}\label{Subsec: 2_2}




\chapter{Implementación en Python de un GA para el VDP} \label{Capítulo 4}




\section{Algoritmos genéticos para el VDP} \label{Subsubsec: 4_2_1}
  
\subsection{Caracterización de la solución}

\subsection{Función de generación}

\subsection{Función de aptitud} \label{Subsec: 4_1_3}

\subsection{Operadores genéticos: Selección}\label{subsec:4_1_4}

\subsection{Operadores genéticos: Cruce}\label{subsec:4_1_5}

\subsection{Operadores genéticos: Mutación}\label{subsec:4_1_6}

\subsection{Función de recombinación}\label{subsec:4_1_7}

\subsection{Función de parada}\label{subsec:4_1_8}




\chapter{Resultados y Análisis} \label{Capítulo 5}



   
\section{Datos del caso práctico} \label{Subsec: 5_1}



\section{Resultados} \label{Subsec: 5_2}



\section{Comparación entre algoritmos de selección} \label{Subsec: 5_3}



\section{Análisis de Sensibilidad} \label{Subsec: 5_4}




\chapter{Conclusiones y Trabajo Futuro} \label{Capítulo 6}



\section{Conclusiones} \label{Subsec: 6_1}



\section{Trabajo futuro} \label{Subsec: 6_2}



  
\newpage

\medskip
\nocite{*}
\bibliographystyle{plain}
\bibliography{biblio}
\clearpage

\newpage

\appendix


\end{document}


