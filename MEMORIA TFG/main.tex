\documentclass[11pt,a4paper]{book}
\usepackage[utf8]{inputenc}
\usepackage[spanish]{babel}
\usepackage{amsmath}
\usepackage{stackrel}
\usepackage{hyperref}
\usepackage{comment} 
\usepackage{xcolor}
\usepackage[left=2.5cm,right=2.5cm,top=2cm,bottom=2.8cm]{geometry}
\setlength{\parskip}{4mm}

\usepackage[linesnumbered,ruled,vlined]{algorithm2e}

\usepackage{wrapfig}
\usepackage{bm}
\usepackage{url}

\PassOptionsToPackage{hyphens}{url} %Permite que los enlaces URL puedan dividirse en múltiples líneas 

\usepackage{mathtools}
\usepackage{amsfonts}
\usepackage{amssymb}
\usepackage{esvect}
\usepackage{booktabs} % Para líneas horizontales de aspecto profesional
\usepackage{array} % Para personalizar el ancho de las columnas


%\usepackage{subfig}

\usepackage{float}
\usepackage{enumitem}

\usepackage{natbib}

\usepackage{amsthm}
\usepackage{graphicx}
\usepackage{xcolor}
\usepackage[nottoc]{tocbibind}

\usepackage{synttree} 

\graphicspath{ {images/} }
\let\olditemize\itemize
\def\itemize{\olditemize\itemsep=0pt }
\setlength{\parindent}{0cm}
\setlist[itemize]{topsep=0pt}
\setlist[enumerate]{topsep=0pt}

\newtheorem{teo}{Teorema}[section]
\newtheorem{cor}[teo]{Corolario}
\newtheorem{defi}[teo]{Definición}
\newtheorem{prop}[teo]{Proposición}
\newtheorem{lema}[teo]{Lema}
\newtheorem{conj}[teo]{Conjetura}
\newtheorem{obs}[teo]{Observación}
\newtheorem{ejem}[teo]{Ejemplo}
\newtheorem{axioma}[teo]{Axioma}



\newcommand\Item[1][]{%
  \ifx\mathbb{R}lax#1\mathbb{R}lax  \item \else \item[#1] \fi
  \abovedisplayskip=-4pt\abovedisplayshortskip=0pt~\vspace*{-\baselineskip}}

\DeclarePairedDelimiter{\abs}{\lvert}{\rvert} %Define un nuevo comando \abs para facilitar la escritura de valores absolutos. Cuando uses \abs{x}, esto se representará como |x| con el tamaño de los delimitadores automáticamente ajustado al tamaño de x.

\newcommand{\mbb}{\mathbb}
\newcommand{\lp}{\ensuremath{\left(}}
\newcommand{\rp}{\ensuremath{\right)}}

\usepackage{footmisc}  
\newcounter{samefootnote}
\newcommand{\samesfootnote}{%
    \stepcounter{footnote}%
    \footnotemark[\value{footnote}]%
}

%\usepackage{acronym}

\usepackage{multicol}
\usepackage{longtable}

\usepackage{listings}
\definecolor{deepblue}{rgb}{0,0,0.5}
\definecolor{deepred}{rgb}{0.6,0,0}
\definecolor{deepgreen}{rgb}{0,0.5,0}
\definecolor{lightgrey}{rgb}{0.95,0.95,0.95}

\renewcommand*{\lstlistingname}{Programa}

\lstset{
   language=Python,
   basicstyle=\ttfamily\small,
   keywordstyle=\color{deepblue}\bfseries\itshape,
   commentstyle=\color{deepgreen}\itshape,
   stringstyle=\color{deepred},
   backgroundcolor=\color{lightgrey},
   morekeywords={as,assert,nonlocal,with,yield},
   showstringspaces=false,
   numbers=left,
   rulecolor=\color{black},
   captionpos=b,
   frame=leftline,
   literate=
   {á}{{\'a}}1 {é}{{\'e}}1 {í}{{\'i}}1 {ó}{{\'o}}1 {ú}{{\'u}}1
   {Á}{{\'A}}1 {É}{{\'E}}1 {Í}{{\'I}}1 {Ó}{{\'O}}1 {Ú}{{\'U}}1
   {à}{{\`a}}1 {è}{{\`e}}1 {ì}{{\`i}}1 {ò}{{\`o}}1 {ù}{{\`u}}1
   {À}{{\`A}}1 {È}{{\`E}}1 {Ì}{{\`I}}1 {Ò}{{\`O}}1 {Ù}{{\`U}}1
   {ä}{{\"a}}1 {ë}{{\"e}}1 {ï}{{\"i}}1 {ö}{{\"o}}1 {ü}{{\"u}}1
   {Ä}{{\"A}}1 {Ë}{{\"E}}1 {Ï}{{\"I}}1 {Ö}{{\"O}}1 {Ü}{{\"U}}1
   {â}{{\^a}}1 {ê}{{\^e}}1 {î}{{\^i}}1 {ô}{{\^o}}1 {û}{{\^u}}1
   {Â}{{\^A}}1 {Ê}{{\^E}}1 {Î}{{\^I}}1 {Ô}{{\^O}}1 {Û}{{\^U}}1
   {ã}{{\~a}}1 {ẽ}{{\~e}}1 {ĩ}{{\~i}}1 {õ}{{\~o}}1 {ũ}{{\~u}}1
   {Ã}{{\~A}}1 {Ẽ}{{\~E}}1 {Ĩ}{{\~I}}1 {Õ}{{\~O}}1 {Ũ}{{\~U}}1
   {œ}{{\oe}}1 {Œ}{{\OE}}1 {æ}{{\ae}}1 {Æ}{{\AE}}1 {ß}{{\ss}}1
   {ű}{{\H{u}}}1 {Ű}{{\H{U}}}1 {ő}{{\H{o}}}1 {Ő}{{\H{O}}}1
   {ç}{{\c c}}1 {Ç}{{\c C}}1 {ø}{{\o}}1 {Ø}{{\O}}1 {å}{{\r a}}1 {Å}{{\r A}}1
   {€}{{\euro}}1 {£}{{\pounds}}1 {«}{{\guillemotleft}}1
   {»}{{\guillemotright}}1 {ñ}{{\~n}}1 {Ñ}{{\~N}}1 {¿}{{?`}}1 {¡}{{!`}}1 
}


\usepackage{caption}
\usepackage{subcaption}
\usepackage[acronym]{glossaries}

%\usepackage{fancyhdr}
%\pagestyle{fancy}
%\fancyhf{} % Limpia la cabecera y pie de página actuales
%\fancyhead[LO,RE]{\nouppercase{\leftmark}} % Cabecera de la página par (izquierda) e impar (derecha)
%\fancyhead[RO,LE]{\thepage} % Número de página en la cabecera derecha (páginas impares) y izquierda (páginas pares)
%\renewcommand{\headrulewidth}{0pt}
%\renewcommand{\chaptermark}[1]{%
%  \markboth{\chaptername\ \thechapter.\ #1}{}%
%}








\makeatletter
\newcommand{\dotminus}{\mathbin{\text{\@dotminus}}}

\newcommand{\@dotminus}{%
  \ooalign{\hidewidth\raise1ex\hbox{.}\hidewidth\cr$\m@th-$\cr}%
}
\makeatother


\usepackage{mdframed}
\newmdenv[leftline=false,topline=false]{topbot}
            
\author{Pablo Jiménez Poyatos}
\title{TFG}
\date{Julio 2024}


\newacronym{cnn}{CNN}{Convolutional Neural Network}
\newacronym{dnn}{DNN}{Deep Neural Network}
\newacronym{dbns}{DBNs}{Deep Belief Networks}
\newacronym{rnn}{RNN}{Recurrent Neural Network}
\newacronym{rbm}{RBM}{Restricted Boltzmann Machine}



\makeglossaries




\begin{document}
\raggedbottom 
\pagenumbering{gobble}


\begin{titlepage}
		\centering
		
		{ \Large UNIVERSIDAD COMPLUTENSE DE MADRID}
		
		{ \Large \textbf{FACULTAD DE CIENCIAS MATEMÁTICAS}}
		\vspace{0.8cm}
		
		{ \large DEPARTAMENTO DE CIENCIAS DE LA  COMPUTACIÓN}
		\vspace{1cm}
		
		\vspace{0.6cm}
		
		\graphicspath{ {images/} }
		%%%%Logo Complutense%%%%%
		\includegraphics[width=0.35\textwidth]{img/ucm.png} 
		\vspace{0.4cm}
		
        {\Large \textbf{TRABAJO DE FIN DE GRADO}}
		
		\vspace{8mm}
        {\huge \bfseries Algoritmos de Aprendizaje Automático aplicados a problemas de Ciberseguridad\par}
		\vspace{1cm}

		{\large Presentado por: Pablo Jiménez Poyatos}
		
		{\large Dirigido por: Luis Fernando Llana Diaz}
		
		\vspace{1.5cm}
		{\large Grado en Matemáticas}
		
		{\large Curso académico 2023-24}
\end{titlepage}

\thispagestyle{empty}
\clearpage
\setcounter{page}{1}


\newpage
\begin{center}
   {\bf Agradecimientos} 
\end{center}




 

\thispagestyle{empty}
\clearpage
\setcounter{page}{1}


\newpage
\begin{center}
   {\bf Resumen} 
\end{center}
   

\vspace{0.6 cm}
\textsl{\textbf{Palabras clave:} } 



\begin{center}
   {\bf Abstract} 
\end{center}



\vspace{0.6 cm}
\textsl{\textbf{Keywords:} } 


\newpage
\tableofcontents

\newpage
\clearpage
\pagenumbering{arabic}

\chapter{Introducción} \label{Capitulo 1}
      



\section{Motivación y objetivos del trabajo} \label{Sec:1_1}

    
\section{Contexto y antecedentes del trabajo} \label{Sec:1_2}

\subsection{Redes neuronales} \label{Subsubsec: 1_2_1}
  
\subsection{Importancia de la detección y prevención de ataques} \label{Subsec: 1_2_2}

Destaca la importancia crítica de la detección y prevención de ataques cibernéticos en entornos empresariales y gubernamentales, así como en la protección de datos sensibles y la infraestructura crítica.

\subsection{Evolución de las amenazas cibernéticas} \label{Subsec: 1_2_3}

Describe brevemente cómo han evolucionado las amenazas en el ámbito de la ciberseguridad a lo largo del tiempo, desde virus simples hasta ataques sofisticados como el ransomware y el phishing.

\subsection{Avances en el aprendizaje automático para ciberseguridad} \label{Subsubsec: 1_2_4}

Proporciona una visión general de cómo los algoritmos de aprendizaje automático han revolucionado el campo de la ciberseguridad, permitiendo la detección temprana de amenazas, el análisis de comportamiento anómalo y la automatización de respuestas.

\section{Metodología} \label{Subsubsec: 1_3}
\section{Estructura de la memoria} \label{Subsubsec: 1_4}
  
El entorno de hardware en el que he realizado todos los experimentos es un servidor proporcionado por la facultad de informática de la Universidad Complutense de Madrid llamado Simba. Tiene un sistema operativo Debian 12.2 con Linux version 6.1.0-17-amd64 con memoria RAM disponible de 128 GB. La CPU utilizada es un Intel(R) Xeon(R) W-2235 CPU con 3.8 GHz con 6 núcleos. 




\chapter{Fundamentos de las redes neuronales} \label{Capitulo_2}
\begin{itemize}
	\item Supervisado y no supervisado
	\item one-hot encoder
	\item validacion cruzada
	\item dropout, l2
	\item optiizadores
	\item arquitectuas
	\item metricas
\end{itemize}
en la pagina 458 de hands aparece la arquitectura mía


\section{Revisión teórica} \label{Subsec: 3_1}
Puedo introducir los tipos de funciones de activavion. Esta bien explicado en el TFG wuolah o en el articulo de KDD cup 199 de DNN network intrusion.
Puedo añadir overfitting y underfitting.
lo que es aprendizaje supervisado y no supervisao
Partes de una neurona y como trabaja(bias, pesos...)



\section{Introducción}

En la última década, la inteligencia artificial (IA) se ha convertido en un tema popular tanto dentro como fuera de la comunidad científica. Una abundancia de artículos en revistas tecnológicas y no tecnológicas han cubierto los temas de aprendizaje automático (ML, por sus siglas en inglés), aprendizaje profundo (DL, por sus siglas en inglés) e IA. Sin embargo, todavía persiste confusión en torno a IA, ML y DL. Los términos están estrechamente relacionados, pero no son intercambiables. 

En 1956, un grupo de científicos informáticos propuso que las computadoras podrían ser programadas para pensar y razonar, ``que cada aspecto del aprendizaje o cualquier otra característica de la inteligencia podría, en principio, ser descrito tan precisamente que una máquina podría simularlo'' \citep{moor2006dartmouth}. Describieron este principio como ``inteligencia artificial''. En pocas palabras, la IA es un campo enfocado en automatizar tareas intelectuales que normalmente realizan los humanos, y el Machine Learning es un método específico para lograr este objetivo. Es decir, está dentro del ámbito de la IA (Figura \ref{fig: ia}) \citep{choi2020introduction}. 

\begin{figure}[h!]
    \centering
    \includegraphics[width=0.4\textwidth]{img/ia.jpg}
    \caption{Relación entre Ciencia de Datos, Inteligencia Artificial, Machine Learning y Deep Learning.}
    \label{fig: ia}
\end{figure}


\section{Aprendizaje Automático}

Por otro lado, el Aprendizaje Automático (ML, por sus siglas en inglés) es la ciencia o el arte de programar ordenadores para que puedan aprender a partir de datos. Arthur Samuel lo definió en 1959 como ``el campo de estudio que otorga a las computadoras la capacidad de aprender sin ser explícitamente programadas''. Más formalmente, según Tom Mitchell (1997), ``se dice que un programa de computadora aprende de la experiencia \(E\) con respecto a alguna tarea \(T\) y alguna medida de rendimiento \(P\), si su rendimiento en \(T\), medido por \(P\), mejora con la experiencia \(E\)'' \citep{geron2022hands}. El aprendizaje automático ha revolucionado numerosos campos, permitiendo a las máquinas realizar tareas que antes requerían intervención humana directa. Desde la conducción autónoma hasta el diagnóstico médico, las aplicaciones del aprendizaje automático son diversas. A diferencia de los métodos tradicionales de programación, donde se codifican reglas explícitas, el aprendizaje automático permite que los sistemas descubran patrones y relaciones directamente a partir de los datos, adaptándose y mejorando con el tiempo.



Un ejemplo de aprendizaje automático es un filtro de spam que, dado ejemplos de correos electrónicos de spam y ejemplos de correos electrónicos normales (no spam, también llamados ``ham''), puede aprender a marcar el spam \citep{geron2022hands}. Los ejemplos que el sistema utiliza para aprender se llaman el conjunto de entrenamiento. Cada ejemplo de entrenamiento se llama una instancia de entrenamiento (o muestra). En este caso, la tarea \(T\) es marcar el spam en los nuevos correos electrónicos, la experiencia \(E\) son los datos de entrenamiento, y la medida de rendimiento \(P\) podría ser la precisión del filtro.

Un filtro de spam utilizando técnicas tradicionales de programación, en primer lugar consideraría cómo se ve normalmente el spam, detectando palabras comunes u otros patrones como el nombre del remitente y escribiendo reglas para cada una de estas. Pero si los encargados de mandar el spam detectan que todos los correos que incluyen la palabra ``Para usted'' o  ``cuenta bancaria'' son rechazados, pueden modificar estas palabras por otras y así ser aceptados por el filtro. Luego un filtro de spam que utiliza técnicas tradicionales de programación necesitaría ser actualizado continuamente para detectar correos electrónicos spam. 

Por otro lado, un filtro de spam basado en técnicas de aprendizaje automático nota automáticamente que "Para ti" se ha vuelto inusualmente frecuente en el spam marcado por los usuarios, y comienza a marcarlos sin intervención humana \citep{geron2022hands}.


\bigskip

El \textbf{esquema global de aprendizaje} consta de tres módulos principales: el generador, el entrenamiento y la decisión. El generador proporciona entradas estructuradas, principalmente vectores con atributos de los datos, para su procesamiento. El entrenamiento ajusta los parámetros del modelo basándose en las salidas deseadas, y por último, la decisión asigna categorías a nuevas muestras de entrada utilizando los parámetros aprendidos durante el entrenamiento \citep{pajares2021aprendizaje}.

\bigskip

En cuanto a la \textbf{clasificación de los sistemas de aprendizaje} automático, se distinguen cuatro tipos principales: el aprendizaje supervisado, no supervisado, semisupervisado y por refuerzo.

Un aprendizaje se dice que es \textbf{supervisado} si todos los datos tienen asociados las salidas deseadas (también llamadas etiquetas). Es decir, para todo dato de entrada x se conoce su salida y, luego tenemos información necesaria para aprender de la función que los relaciona y = f(x). El aprendizaje supervisado busca estimar la función $\hat{f}$ tal que $\hat{f}$ sea una buena aproximación de la función $f$ que relaciona las entradas x con las salidas Y.

Un aprendizaje se dice que es \textbf{supervisado} si todos los datos tienen asociadas las salidas deseadas (también llamadas etiquetas). Es decir, para todo dato de entrada \( x \) se conoce su salida \( y \), de tal forma que para una función desconocida $f$ que relaciona ambas, $y=f(x)$. El objetivo del aprendizaje supervisado es estimar la función \( f \) con una función \( \hat{f} \) de tal forma que \( \hat{f} \) sea una buena aproximación de la función \( f \) que relaciona las entradas \( x \) con las salidas \( y \). Un ejemplo de aprendizaje supervisado puede ser el del filtro de spam ya que se entrena el modelo con los correos y con la etiqueta de si son spam o no. Otros ejemplos incluyen regresión lineal, regresión logística, árboles de decisión y redes neuronales \citep{geron2022hands}.


Un aprendizaje se dice que es \textbf{no supervisado} si, al contrario que el aprendizaje supervisado, los datos de entrenamiento no están etiquetados. De esta forma, el modelo tiene que aprender sin supervisión (tambíen llamado sin ``profesor''). El objetivo principal de este tipo de algoritmos se basa en el agrupamiento, detección de anomalías o reducción de dimensionalidad \citep{geron2022hands}. 


En el caso de encontrarnos ante un problema en el que tengamos datos tanto etiquetados como sin etiquetar, nos encontramos ante un tipo de \textbf{aprendizaje semisupervisado}, que se encuentra entre el supervisado y el no supervisado. Este tipo de aprendizaje suele darse en situaciones en las que obtener etiquetas de los datos puede ser muy costoso pero sin embargo obtener datos sin etiquetar no tanto. Un ejemplo podría ser la agrupación de fotos donde sale la misma persona en Google Photos. La parte no supervisada sería la de agrupación y la supervisada la de dar una etiqueta a cada grupo \citep{geron2022hands}.


Por último, está el \textbf{aprendizaje por refuerzo}, un tipo de aprendizaje un poco diferente a los otros tres.  El sistema de aprendizaje, llamado agente, observa el entorno, selecciona y realiza acciones para obtener recompensas a cambio (o penalizaciones en forma de recompensas negativas). Luego debe aprender por sí mismo cuál es la mejor estrategia, llamada política, para obtener la mayor recompensa con el tiempo. Una política define qué acción debe elegir el agente cuando se encuentra en una situación determinada. Por ejemplo, muchos robots implementan algoritmos de aprendizaje por refuerzo para aprender a andar \citep{geron2022hands}.






¿PONER O NO?
El aprendizaje automático es excelente para:

\begin{itemize}
\item Problemas para los cuales las soluciones existentes requieren muchos ajustes o largas listas de reglas: un algoritmo de aprendizaje automático a menudo puede simplificar el código y funcionar mejor que el enfoque tradicional.

\item Problemas complejos para los que el enfoque tradicional no ofrece una buena solución: las mejores técnicas de Machine Learning quizás puedan encontrar una solución.

\item Entornos fluctuantes: un sistema de Machine Learning puede adaptarse a nuevos datos.

\item Obtener información sobre problemas complejos y grandes cantidades de datos.
\end{itemize}


\section{Aprendizaje profundo}

Dentro del Machine Learning, se encuentra el Deep Learning, cuya base son las redes neuronales artificiales (ANN). Una red neuronal artificial es un modelo matemático inspirado en la estructura de una red neuronal biológica. Consiste en una red de neuronas interconectadas organizadas por capas, con una capa de entrada, una o más capas ocultas y una capa de salida \citep{dolling2002artificial}. En cada neurona se aplica una suma ponderada de las señales recibidas a las que se le aplica una función de activación o conexión no lineal. La capa de entrada recibe la información del exterior y la agrupa en la capa de entrada, mandando una salida a la siguiente capa a través de sus neuronas con pesos asociados en cada conexión. Las capas ocultas reciben información de otras neuronas artificiales y cuyas señales de entrada y salida permanecen dentro de la red. Por último, la capa de salida recibe la información procesada y la devuelve al exterior con la salida predicha por nuestro modelo. Además de los pesos que se van ajustando durante el entrenamiento, también está el sesgo o bias, que es un valor que se asigna a cada neurona de cada capa para añadir características adicionales a la red neuronal que antes no tenía. 


\subsection{Perceptrón}

Antes de profundizar en los modelos de redes neuronales más complejos y profundos, veamos el funcionamiento del modelo más simple, el \textit{Perceptrón}. Este modelo es la base del resto de modelos de aprendizaje automático. Consiste en una capa de entrada y en una de salida en la que hay que aplicar dos etapas. En la figura \ref{img: perceptron} podemos ver el esquema para dos clases.

\begin{figure}[h!]
    \centering
    \includegraphics[width=0.4\textwidth]{img/perceptron.png}
    \caption{Modelo del perceptrón simple}
    \label{img: perceptron}
\end{figure}

La primera etapa del proceso consiste en calcular la suma promediada de sus entrada mediante una función lineal
\begin{equation}
f(x) = \sum_{i=1}^{n} w_i \cdot x_i + b
\end{equation}
Los coeficientes $w_i$, $i=1,2, \ldots, n$ llamados pesos, dan un valor determinado a cada una de las entradas en función de la importancia para obtener la salida. Además, el coeficiente $b$ es el sesgo o bias que se añade a la función. Otra forma práctica de escribir esta ecuación sería: $f(x) = \sum_{i=1}^{n+1} w_i \cdot z_i = w^t\cdot z  donde w = (w_1, \ldots, w_n, b)^t y z^t = (x_1, \ldots, x_n, 1)^t$

La segunda capa consiste en transformar la salida de la primera etapa mediante una función de activación. En el caso de un problema de clasificación binaria (0 o 1), si esta salida sobrepasa un cierto umbral predefinido al principio, su salida será 1 y en caso contrario, será 0. Es decir, siendo $c$ una constante real:

\begin{equation}
\text{y} = 
\begin{cases} 
1  \text{  si  }  \sum_{i=1}^{n} w_i \cdot x_i + b > c\\
0  \text{  si  }  \sum_{i=1}^{n} w_i \cdot x_i + b <= c.
\end{cases}
\end{equation}












\section{GENERALIZACIÓN}
\section{COMO SE APRENDE}
\section{OPTIMIZACIÓN}
\begin{comment}
\section{Métodos de Optimización}

La optimización se refiere a la tarea de minimizar o maximizar una función $f(x)$ modificando $x$. En general, la optimización alude a una minimización, ya que la maximización es equivalente a la minimización de $-f(x)$. La función a minimizar (o maximizar) se conoce como función objetivo o criterio, que se utiliza para evaluar una solución candidata, por ejemplo, los pesos en una red neuronal. Cuando se está minimizando, según Goodfellow y col. (2016), se le denomina también función de coste (\textit{cost function}), función de pérdida (\textit{loss function}), o función de error (\textit{error function}). La función de pérdida es necesaria para poder disponer de un error que retropropagar durante la etapa de aprendizaje. Así pues, la función o grupo de funciones que se están minimizando se denominan \textit{loss functions}, y el valor que obtiene la minimización o maximización se identifica con el símbolo $\hat{x} = \arg \min f(x)$.

Conviene destacar que la función de pérdida debe ser diferenciable, y en este sentido, durante el entrenamiento, mediante dicha función se compara la salida de la red neuronal (predicción) con la etiqueta del objetivo verdadera (\textit{ground-truth}). Según los distintos modelos de red, se definen diferentes funciones de pérdida. En la red neuronal se obtienen todos los cálculos realizados por cada una de las capas. Por tanto, si cada uno de estos cálculos es diferenciable, entonces la función de pérdida también será diferenciable. No obstante, a veces se utilizan funciones de activación, tales como ReLU, que no son diferenciables; en estos casos es necesario aplicar alguna aproximación derivativa para poder utilizar la función en los procesos de retropropagación, mediante gradiente descendente, como se verá posteriormente.

Una función de pérdida es una medida de cuán bueno es un modelo de predicción en términos de poder aproximarse lo más posible (predecir) al resultado esperado. Uno de los métodos más utilizados para encontrar el punto mínimo de una función es el denominado \textit{descenso de gradiente}. Se puede pensar en la función de pérdida como una montaña ondulada y el descenso en pendiente es como deslizarse por la montaña para llegar al punto más bajo; de aquí surge el concepto de optimización mediante el gradiente (\textit{gradient-based optimization}). Cuando se calcula la pérdida, se debe mejorar el modelo. Esto se hace propagando el error hacia atrás a través de la estructura del modelo, de forma que en las redes neuronales son los pesos del modelo los que terminan ajustándose. Esto cierra el ciclo de aprendizaje entre el envío de los datos hacia adelante, la generación de predicciones y la mejora durante la propagación hacia atrás. Al adaptar los pesos, el modelo probablemente mejore (a veces mucho, a veces ligeramente) y, por lo tanto, se dice que se ha realizado el aprendizaje.

Resulta bien conocido que la derivada de una función en un punto $x$ proporciona la pendiente de la función en dicho punto. La derivada permite escalar un pequeño cambio en la entrada para obtener el correspondiente cambio en la salida: $f(x+\delta) = f(x) + \delta f'(x)$. La derivada es, por tanto, útil para minimizar una función porque indica cómo cambiar $x$ para conseguir una pequeña mejora en la salida $y$. Por ejemplo, se sabe que $f(x - \delta \cdot \text{sign}(f'(x)))$ es menor que $f(x)$ para un $\delta$ suficientemente pequeño. Se puede entonces disminuir $f(x)$ moviéndose en pequeños pasos con el signo opuesto de la derivada. Esto se conoce como \textit{gradiente descendente} (\textit{gradient descent}).

\begin{figure}[h!]
    \centering
    \includegraphics[width=0.5\textwidth]{gradiente_descendente.png}
    \caption{Gradiente descendente}
    \label{fig:gradiente}
\end{figure}

Para $x < 0$, $f'(x) < 0$ y $f$ disminuye hacia la derecha. Para $x > 0$, $f'(x) > 0$ y $f$ disminuye hacia la izquierda. Para $x = 0$, el gradiente se detiene puesto que $f'(x) = 0$.

Cuando $f'(x) = 0$, la derivada no proporciona información sobre la dirección en la que hay que moverse, de esta forma, los puntos con $f'(x) = 0$ se conocen como puntos críticos o puntos estacionarios. Un mínimo local es un punto donde $f(x)$ es menor que todos los puntos vecinos, de modo que no es posible disminuir $f(x)$ mediante pasos infinitesimales. Un máximo local es un punto donde $f(x)$ es mayor que todos los puntos vecinos, de modo que no es posible incrementar $f(x)$ mediante pasos infinitesimales. Otros puntos críticos denominados puntos de inflexión son los que tienen vecinos que son mayores y menores que el punto mismo, pero no son ni máximos ni mínimos.

\begin{figure}[h!]
    \centering
    \includegraphics[width=0.5\textwidth]{puntos_criticos.png}
    \caption{Puntos críticos}
    \label{fig:puntos_criticos}
\end{figure}

Un punto que obtiene el valor más bajo de $f(x)$ es un mínimo global. Puede haber solo un mínimo global o múltiples mínimos globales de la función. También es posible que haya mínimos locales que no sean globalmente óptimos. En el contexto del aprendizaje profundo, se optimizan las funciones que pueden tener muchas limitaciones locales que no son óptimas y muchos puntos de inflexión rodeados de regiones muy planas. Todo esto hace que la optimización sea difícil, especialmente cuando la entrada a la función es multidimensional.

A menudo es necesario minimizar funciones que tienen múltiples ($n$) entradas: $f: \mathbb{R}^n \to \mathbb{R}$, con una única salida (escalar). Con múltiples entradas, se debe utilizar el concepto de derivadas parciales $\frac{\partial f(x)}{\partial x_i}$, que mide cómo cambia $f$ según aumenta la variable $x_i$ en el punto $x$. El gradiente generaliza la noción de derivada al caso en que la derivada es con respecto a un vector: el gradiente de $f$ es el vector que contiene todas las derivadas parciales, denotado por $\nabla f(x)$. El elemento $i$-ésimo del gradiente es la derivada parcial de $f$ con respecto a $x_i$. En múltiples dimensiones, los puntos críticos son puntos en los que cada elemento del gradiente es igual a cero. La derivada direccional en la dirección $u$ (vector unitario) es la pendiente de la función $f$ en esa dirección $u$. En otras palabras, la derivada direccional es la derivada de la función $f(x + \alpha u)$ con respecto a $\alpha$, evaluada en $\alpha = 0$. Usando la regla de la cadena, se determina que $f(x + \alpha u)$ evalúa $u \cdot \nabla f(x)$ cuando $\alpha = 0$.

Para minimizar $f$, es deseable encontrar la dirección en la que $f$ disminuye de forma más rápida, lo que se puede hacer usando la derivada direccional:
\[
\min_u \nabla f(x) \cdot u = \min_{\theta} \|\nabla f(x)\| \cos(\theta)
\]
donde $\theta$ es el ángulo entre $u$ y el gradiente. Sustituyendo $\|u\| = 1$ e ignorando factores que no dependen de $\theta$, se concreta en minimizar $\min_{u} \cos \theta$. Esto se consigue cuando $u$ apunta en la dirección opuesta del gradiente. En otras palabras, el gradiente apunta directamente hacia arriba. Se puede hacer decrecer $f$ moviéndose en la dirección opuesta del gradiente, lo que se conoce como método del gradiente descendente (\textit{steepest descent} o \textit{gradient descent}), que propone un nuevo punto,
\[
x' = x - \eta \nabla f(x)
\]
donde $\eta$ es la razón de aprendizaje, que es un valor escalar que determina la dimensión del paso. Se puede elegir de varias formas, una de ellas es fijarlo a un valor pequeño constante. Algunas veces se puede determinar la dimensión del paso que hace desaparecer la derivada direccional. Otra aproximación consiste en evaluar $f(x - \eta \nabla f(x))$ para varios valores de $\eta$ y elegir el que proporciona el valor de la función objetivo más pequeño. Esta estrategia se denomina \textit{búsqueda en línea} (\textit{line search}).

El gradiente descendente converge cuando cada elemento del gradiente es cero (o, en la práctica, muy próximo a cero). En algunos casos, se puede evitar ejecutar este algoritmo iterativo y simplemente saltar al punto en el que el gradiente es igual a cero. En muchas situaciones es posible resolver analíticamente la ecuación $\nabla f(x) = 0$. En la optimización convexa, la optimización de funciones cóncavas y diferenciables que tienen una única solución óptima global, se puede obtener una solución cerrada sin necesidad de recurrir a métodos iterativos, como el gradiente descendente. Sin embargo, las redes neuronales, así como otros modelos que pueden tener múltiples óptimos locales y limitaciones no convexas, se enfrentan a mayores desafíos en este aspecto.



\subsection{Gradiente Descendente Estocástico}

En el ámbito del aprendizaje automático, el problema del Gradiente Descendente Estocástico (SGD, Stochastic Gradient Descent) consiste en minimizar una función objetivo que tiene la forma (Bishop, 2006; Ruder, 2017),

\begin{equation}
J(w) = \sum_{i=1}^{n} L(y_i, f(x_i; w)),
\end{equation}

donde el parámetro \( w \), que minimiza \( J(w) \), debe estimarse. \( J_i \) se asocia con la i-ésima observación en el conjunto de datos utilizados para el entrenamiento (ajuste). El término estocástico proviene del hecho relativo a la selección de las muestras (observaciones) para el ajuste de forma aleatoria, que incluso se pueden seleccionar por lotes en cada iteración. \( J(w) \) es el valor de la función de pérdida (loss function) en el i-ésimo ejemplo y \( J(w) \) es el riesgo empírico.

El gradiente descendente se usa para minimizar la siguiente función de forma iterativa, con iteraciones \( t \). En el Anexo se proporciona un mayor nivel de detalle.

\begin{equation}
w(t+1) = w(t) - \eta \nabla J(w(t)),
\end{equation}

donde \( \eta \) es la tasa de aprendizaje. De esta forma iterativa, el método recorre el conjunto de entrenamiento y realiza la actualización indicada anteriormente para cada muestra de entrenamiento. Según la ecuación (2.5), gradientes negativos incrementan el peso y viceversa, siempre con el fin de minimizar la función objetivo. Se pueden realizar varios pasos sobre el conjunto de entrenamiento hasta que el algoritmo converja. Si se hace esto, los datos se pueden seleccionar aleatoriamente en cada paso para evitar ciclos; es lo que se conoce como criterio estocástico. Estas muestras así seleccionadas constituyen lo que se denomina un lote (batch). Las implementaciones típicas pueden usar una razón de aprendizaje adaptativa para que el algoritmo converja.

En pseudocódigo, el método del gradiente descendente estocástico es como sigue:

\begin{enumerate}
    \item Elegir un vector inicial de parámetros \( w \) (puede ser aleatoriamente) y una tasa de aprendizaje \( \eta \).
    \item Repetir hasta que se consiga un mínimo aproximado:
    \begin{enumerate}
        \item Seleccionar aleatoriamente ejemplos en el conjunto de entrenamiento.
        \item Para \( i = 1, 2, \ldots, n \), hacer:
        \begin{equation}
        w(t+1) = w(t) - \eta \nabla J_i(w).
        \end{equation}
    \end{enumerate}
\end{enumerate}

\subsection{Ejemplo}

Supóngase que se quiere ajustar una línea recta \( y = w_1 + w_2 x \) a partir de un conjunto de observaciones \( (x_1, x_2, \ldots, x_n) \) con sus correspondientes respuestas estimadas \( (y_1, y_2, \ldots, y_n) \) utilizando mínimos cuadrados. La función objetivo a minimizar es:

\begin{equation}
J(w) = \sum_{i=1}^{n} (w_1 + w_2 x_i - y_i)^2.
\end{equation}

El punto 2.2 del pseudocódigo resulta en:

\begin{equation}
w_1(t+1) = w_1(t) - \eta \sum_{i=1}^{n} 2(w_1 + w_2 x_i - y_i),
\end{equation}

\begin{equation}
w_2(t+1) = w_2(t) - \eta \sum_{i=1}^{n} 2(w_1 + w_2 x_i - y_i)x_i.
\end{equation}

En cada iteración (también llamada actualización), solo se evalúa el gradiente en un único punto \( x \) en lugar de evaluar en el conjunto de todas las muestras. La diferencia clave en comparación con el gradiente descendente estándar es que solo se utiliza un dato del total del conjunto de datos disponible para obtener la actualización, y el dato se selecciona de forma aleatoria en cada paso.

Existen diversas variantes, extensiones y mejoras del algoritmo. Resulta de particular interés el hecho de fijar la razón de aprendizaje, ya que valores altos tienden hacia la divergencia de los datos, mientras que valores demasiado pequeños hacen que la convergencia sea lenta. Una forma de abordar este problema es que dicha razón sea variable disminuyendo a medida que se incorporan nuevas muestras (mayor aprendizaje). Esto se consigue haciéndola dependiente del número de datos o iteraciones. Una de tales extensiones es la del método del momentum (Rumelhart y col., 1986; Murphy 2012). Este método recuerda la actualización \( \Delta w \) en cada iteración, determinando la siguiente actualización como una combinación lineal del gradiente y la actualización previa (Sutskever y col., 2013):

\begin{equation}
\Delta w(t) = \alpha \Delta w(t-1) - \eta \nabla J_i(w(t-1)),
\end{equation}

\begin{equation}
w(t) = w(t-1) + \Delta w(t).
\end{equation}

En resumen, la dimensión del lote expresa el número de muestras procesadas antes de actualizar los pesos de la red. El número de épocas (epochs) es el número de pases completos de todo el conjunto de muestras de entrenamiento \( N \). Se puede establecer que la razón de aprendizaje varíe cada cierto número de épocas, por ejemplo, cada 5 épocas disminuye un 10\% con respecto al valor inicial. Se puede ejecutar el algoritmo todo el tiempo que se desee e incluso detenerlo utilizando otros criterios además de un número fijo de épocas, tal como que el error del modelo o que la función objetivo no cambia con el número de iteraciones. En estos casos se dice que se ha alcanzado la convergencia y los criterios así establecidos se denominan de convergencia.

A modo de ejemplo, supóngase un conjunto de datos con 400 muestras, de forma que se define la dimensión del lote en 10 con 100 épocas. Esto significa que el conjunto de datos se divide en 40 lotes, cada lote con 10 muestras. La actualización de los pesos del modelo se realizará tras el paso de cada lote con 10 muestras. Así, una época involucra 40 lotes, es decir, 40 actualizaciones del modelo.




Esto se consigue haciéndola dependiente del número de datos o iteraciones. Una de tales extensiones es la del método del \textit{momentum} (Rumelhart y col., 1986; Murphy, 2012). Este método recuerda la actualización $\Delta w$ en cada iteración, determinando la siguiente actualización como una combinación lineal del gradiente y la actualización previa (Sutskever y col., 2013):

\[
\Delta w(t) = \alpha \Delta w(t-1) - \eta \nabla J_i(w(t-1))
\]

La anterior expresión conduce a:

\[
w(t) = w(t-1) + \alpha \Delta w(t-1) - \eta \nabla J_i(w(t-1))
\]

donde $\eta > 0$ es la razón de aprendizaje y $\alpha \in [0,1]$ es la constante de momento que controla la velocidad de actualización de $\Delta w(t-1)$. El vector de parámetros, $w$, que minimiza $J(w)$ es realmente el que se estima en el proceso de optimización. El nombre de \textit{momentum} proviene del concepto de momento en física, de forma que el vector de pesos $w$, visto como una partícula viajando a través del espacio de parámetros, adquiere una aceleración a partir de la fuerza, tendiendo a mantenerse en la misma dirección, evitando así oscilaciones.

En los procesos de optimización de redes neuronales, si los gradientes incrementan su magnitud exponencialmente, por ejemplo, porque los pesos toman valores altos afectando a los resultados de las multiplicaciones, el entrenamiento se hace inestable y puede divergir en unas pocas iteraciones. Esto es lo que se conoce como \textit{explosión del gradiente}. Pascanu y col. (2013) proponen el \textit{recorte de gradiente} (\textit{gradient clipping}), que ayuda a prevenir dicha explosión estabilizando el entrenamiento ante la presencia de \textit{outliers}. Para ello se propone considerar un determinado umbral $T$ de forma que si:

\[
\|\nabla J(w)\| > T \text{ entonces } \nabla J(w) = T \frac{\nabla J(w)}{\|\nabla J(w)\|}
\]

De forma similar, si los valores del gradiente son muy pequeños también por influencia de valores de pesos pequeños, el problema se conoce como \textit{desaparición del gradiente}. En la literatura se han investigado diversas aproximaciones en aras de la mejora de la convergencia, por ejemplo, mediante el promediado de trayectorias (Polyak y Juditsky, 1992).

Durante el proceso de actualización de los pesos de la red, mediante el cálculo del error, existen básicamente tres tipos de estrategias. En todas ellas, se dispone de $N$ muestras de entrenamiento, la diferencia estriba en cómo se realiza la actualización (Kim, 2017; Brownlee, 2018). El proceso de optimización es iterativo, lo que quiere decir que cuando se busca el mínimo, se realizan una serie de pasos, siendo el objetivo de cada paso ajustar lo mejor posible los pesos. Cada paso consiste en utilizar el modelo con los parámetros actuales, realizar una predicción sobre algunas muestras, comparar las predicciones con las muestras, calcular el error y propagar el error hacia atrás para actualizar de nuevo los pesos del modelo.

Una muestra es un dato simple, también llamada instancia, observación, vector de entrada o vector de características, que se suministra al algoritmo junto con una salida que se utiliza para comparar el error con respecto a la salida que se obtiene con ella y la esperada. Este planteamiento se engloba en lo que se denomina \textit{aprendizaje supervisado}. Un conjunto de muestras de entrenamiento engloba un cierto número, $n$, de muestras.

Como se ha indicado anteriormente, existe el concepto de \textit{lote} (\textit{batch}) que define el número de muestras utilizadas antes de llevar a cabo una actualización de los pesos de la red en el modelo. Considérese un lote, constando de una o más muestras dentro de un bucle y haciendo predicciones precisamente con esas muestras del lote. Una vez que han pasado todas las muestras del lote, se realizan las predicciones y se comparan con las variables de salida esperadas para obtener el error correspondiente. Con este error, el algoritmo de optimización actualiza los pesos que definen el modelo. Esto significa que hasta que no han pasado todas las muestras del lote no se actualizan dichos pesos. Un conjunto de datos de entrenamiento se puede dividir en uno o más lotes.

Cuando todas las muestras de entrenamiento se utilizan para crear un lote, el algoritmo de aprendizaje se denomina \textit{descenso del gradiente por lotes}. Cuando el lote es del tamaño de una muestra, el algoritmo de aprendizaje se denomina \textit{descenso de gradiente estocástico} (\textit{SGD}) o de otro tipo. Cuando el tamaño del lote posee más de una muestra y es menor que el tamaño del conjunto de datos de entrenamiento, el algoritmo de aprendizaje se denomina \textit{descenso de gradiente de mini-lote} (\textit{mini-batch}). Dimensiones habituales de los mini-lotes son 32, 64 y 128, lo que suele derivar en el hecho de que el último lote a veces tiene un tamaño menor que el resto, al no coincidir el número de muestras totales con esas dimensiones, cuando se reparten las muestras en los lotes. A veces se eliminan algunas muestras para que esto no ocurra.

Por otra parte, existe otro parámetro conocido como \textit{época} (\textit{epoch}), que define el número de veces que el conjunto total de muestras son procesadas por el algoritmo. Esto significa que cada muestra del conjunto $N$ ha tenido una oportunidad de actualizar los pesos en cada \textit{epoch}. Una \textit{epoch} puede tener uno o más lotes. Para mayor claridad, se puede pensar en un bucle sobre el número de \textit{epochs} donde cada iteración o ciclo opera sobre $N$. Dentro de esta iteración, existe otra iteración anidada que actúa sobre cada lote de muestras, teniendo en cuenta que un lote tiene el número especificado de muestras según el tamaño del lote. El número de \textit{epochs} depende del tipo de datos, siendo a veces alto, para garantizar una eficiencia suficiente. Se pueden encontrar aplicaciones con 10, 100, 1000 e incluso más, y naturalmente también con menos. Además, es común representar gráficamente, sobre un sistema de ejes bidimensional (x-y), el proceso de aprendizaje mostrando las \textit{epochs} sobre el eje horizontal $x$ junto con las iteraciones y el error del modelo en el eje $y$. Estas representaciones ayudan a diagnosticar la evolución del aprendizaje, y cómo el modelo se ajusta adecuadamente o no al conjunto de datos de entrenamiento.

En resumen, la dimensión del lote expresa el número de muestras procesadas antes de actualizar los pesos de la red. El número de \textit{epochs} es el número de pases completos de todo el conjunto de muestras de entrenamiento $N$. Se puede establecer que la razón de aprendizaje varíe cada cierto número de \textit{epochs}, por ejemplo, cada 5 \textit{epochs} disminuya un 10\% con respecto al valor inicial. Se puede ejecutar el algoritmo todo el tiempo que se desee e incluso detenerlo utilizando otros criterios además de un número fijo de \textit{epochs}, tal como que el error del modelo o que la función objetivo no cambia con el número de iteraciones. En estos casos se dice que se ha alcanzado la convergencia y los criterios así establecidos se denominan de convergencia. A modo de ejemplo, supóngase un conjunto de datos con 400 muestras, de forma que se define la dimensión del lote en 10 con 100 \textit{epochs}. Esto significa que el conjunto de datos se divide en 40 lotes, cada lote con 10 muestras. La actualización de los pesos del modelo se realizará tras el paso de cada lote con 10 muestras. Así, una \textit{epoch} involucra 40 lotes, es decir, 40 actualizaciones del modelo.

A veces ocurre que el conjunto de muestras de entrenamiento no es divisible por un determinado número de lotes, en este caso cabe la posibilidad de considerar el último lote con menos muestras que los otros, como se ha indicado previamente, o bien eliminar algunas muestras del conjunto de datos o también cambiar el tamaño del lote. Supóngase que se tiene un conjunto de muestras de entrenamiento de 208, y se establece la dimensión del mini-lote en 10 con 4 \textit{epochs}, esto significa que el paso de las 208 muestras determina una \textit{epoch}. El número de mini-lotes se obtiene calculando el resultado de la división $208/10 = 20.8$, lo que significa que se formarán 20 mini-lotes con 10 muestras cada uno, más uno adicional con 8 muestras. En total habrá 21 mini-lotes. Cada uno de estos mini-lotes define una iteración, por lo que en cada \textit{epoch} se realizan 21 iteraciones, de forma que tras cada una de ellas se actualizan los pesos.

Considerando la ecuación (2.9) simplificada $\Delta w = -\eta \nabla J_i(w(t-1))$ y un conjunto con $N$ muestras de entrenamiento, cuando no se definen mini-lotes, o lo que es lo mismo, la dimensión del mini-lote es 1, entonces, cada muestra actualiza los pesos según esta ecuación. Por el contrario, si las $N$ muestras forman un lote, entonces la actualización de los pesos se realiza una única vez, promediando los pesos obtenidos tras el paso de cada muestra como sigue:

\[
\Delta w = \frac{1}{N} \sum_{k=1}^{N} \Delta w(k)
\]

Si se establecen mini-lotes de dimensión $m$, la actualización de los pesos también se lleva a cabo de la misma manera que en el caso anterior, si bien promediando con $m$ en lugar de $N$ y realizando tantas actualizaciones por \textit{epoch} como número de mini-lotes existan.





\end{comment}
\section{Función de activación.}

Entre capa y capa de las redes neuronales, cada neurona suma ponderadamente las entradas provenientes de las neuronas de la capa anterior. Este resultado, se lo pasa a una función no lineal, llamada función de activación, que permiten que la red neuronal pueda aprender, representar patrones complejos y crear relaciones no lineales entre las características de entrada y la salida. El valor resultante de aplicarle la función es el valor que se le queda asociado a esta neurona, para que futuras capas puedan usarlo. En esta sección, describimos varias funciones de activación populares, sus propiedades y aplicaciones \citep{pajares2021aprendizaje}.


\subsubsection*{Softmax}


La función \textbf{Softmax} se utiliza comúnmente en la capa de salida de redes neuronales para problemas de clasificación multiclase. Convierte un vector de valores arbitrarios en un vector de probabilidades, donde la suma de todas las probabilidades es 1. La función Softmax está definida como:

\begin{equation}
    \text{Softmax}(z_i) = \frac{e^{z_i}}{\sum_{j=1}^{K} e^{z_j}}
\end{equation}

donde \(z_i\) es el valor de la \(i\)-ésima neurona de salida, y \(K\) es el número total de clases. La interpretación de esta función es que transforma las salidas en probabilidades, facilitando la toma de decisiones sobre la clase a la que pertenece una muestra en función de las probabilidades más altas \citep{pajares2021aprendizaje}. 




\subsubsection*{Sigmoide}


\begin{minipage}{0.6\textwidth}
    La función \textbf{sigmoide}, también conocida como la función logística, convierte entradas en el rango \(0, 1\). Está definida como:
\begin{equation}
    \sigma(x) = \frac{1}{1 + e^{-x}} 
\end{equation}
Sin embargo, tiene problemas como la saturación del gradiente (cuando los valores se aproximan a 0 o 1, el gradiente tiende a 0, lo que afecta en la actualización de los pesos) o los pesos tienen esperanza de la función de salida positiva, lo que puede llevar a una convergencia lenta \citep{pajares2021aprendizaje}.
\end{minipage}
\begin{minipage}{0.05\textwidth}
\textbf{ }
\end{minipage}
\begin{minipage}{0.35\textwidth}
    \includegraphics[width=1.1\textwidth]{img/Sigmoide.png}
    \captionof{figure}{Gráfica de la función sigmoide.}
    \label{img: sigmoide}
\end{minipage}



\subsubsection*{Sigmoide dura}


\begin{minipage}{0.35\textwidth}
    \includegraphics[width=1.1\textwidth]{img/Sigmoide Dura.png}
    \captionof{figure}{Gráfica de la función sigmoide dura.}
    \label{img: sigmoideDura}
\end{minipage}
\begin{minipage}{0.05\textwidth}
\textbf{ }
\end{minipage}
\begin{minipage}{0.6\textwidth}
    Para solucionar el problema de la saturación del gradiente en los extremos, está la función \textbf{sigmoide dura} (\textit{hard-sigmoid}), que es una aproximación simplificada de la función sigmoide. Su fórmula es la siguiente:

\begin{equation}
    f(x) = \max(0, \min(1, \frac{x+1}{2}))
\end{equation}

Esta función es computacionalmente más eficiente y evita los problemas de saturación en los extremos \citep{pajares2021aprendizaje}.
\end{minipage}



\subsubsection*{Función Tangente hiperbólica}
 

\begin{minipage}{0.6\textwidth}
    La función \textbf{tangente hiperbólica (tanh)} es similar a la sigmoide pero escala las salidas al rango \([-1, 1]\):

\begin{equation}
    \tanh(x) = \frac{\sinh(x)}{\cosh(x)} = \frac{e^x + e^{-x}}{e^x + e^{-x}}
\end{equation}

Aunque también sufre de saturación del gradiente, \textit{tanh} centra las salidas alrededor de cero, lo que puede acelerar el entrenamiento \citep{pajares2021aprendizaje}. 
\end{minipage}
\begin{minipage}{0.05\textwidth}
\textbf{ }
\end{minipage}
\begin{minipage}{0.35\textwidth}
    \includegraphics[width=1.1\textwidth]{img/tanh.png}
    \captionof{figure}{Gráfica de la función tangente hiperbólica.}
    \label{img: tanh}
\end{minipage}



\subsubsection*{Función ReLU}

\begin{minipage}{0.35\textwidth}
    \includegraphics[width=1.1\textwidth]{img/ReLU.png}
    \captionof{figure}{Gráfica de la función ReLU.}
    \label{img: relu}
\end{minipage}
\begin{minipage}{0.05\textwidth}
\textbf{ }
\end{minipage}
\begin{minipage}{0.6\textwidth}
    La \textbf{función ReLU} es popular debido a su simplicidad y efectividad para solventar el problema de la saturación del gradiente.

\begin{equation}
    f(x) = \max(0, x)
\end{equation}
Sin embargo, la función ReLU tiene algunos problemas, como el problema de la ``ReLU muerta'' \citep{maas2013rectifier} y la falta de diferenciabilidad en cero. Para este problema último, se suele asignar un valor arbitrario para este valor como 0, 0.5 o 1 citep{pajares2021aprendizaje}. Esto ocurre cuando se aprende un sesgo negativo grande, haciendo que la salida de la neurona sea siempre cero, sin importar la entrada \citep{apicella2021survey}. Por eso, a continuación, discutimos algunas variantes de ReLU.
\end{minipage}


\subsubsection*{Función Leaky ReLU (LReLU)}

\begin{minipage}{0.6\textwidth}
    Una de las primeras funciones de activación basadas en ReLU fue \textbf{LReLU} \citep{maas2013rectifier}. La función LReLU fue un intento de aliviar los problemas potenciales de ReLU mencionados anteriormente. Se define como:

\begin{equation}
    f(x) = \max(0.001x, x)
\end{equation}

La función de activación Leaky ReLU permite que la neurona tenga un pequeño gradiente cuando su salida es cero o negativa (es decir, cuando $x \leq 0)$. Sin embargo, se ha demostrado que Leaky ReLU funciona de manera casi idéntica a la ReLU original, por lo que no mejora significativamente el rendimiento de la red. Además, se propuso una versión aleatoria de ReLU, donde el valor de $x$ se elige aleatoriamente de una distribución uniforme \( U(l, u) \) con \( 0 \le l < u < 1 \) \citep{apicella2021survey}.
\end{minipage}
\begin{minipage}{0.05\textwidth}
\textbf{ }
\end{minipage}
\begin{minipage}{0.35\textwidth}
    \includegraphics[width=1.1\textwidth]{img/Leaky ReLU.png}
    \captionof{figure}{Gráfica de la función Leaky ReLU.}
    \label{img: LReLU}
\end{minipage}


\subsubsection*{Función paramétrica ReLU(PReLU)}

\begin{minipage}{0.35\textwidth}
    \includegraphics[width=1.1\textwidth]{img/PReLU.png}
    \captionof{figure}{Gráfica de la función Paramétrica ReLU.}
    \label{img: PReLU}
\end{minipage}
\begin{minipage}{0.05\textwidth}
\textbf{ }
\end{minipage}
\begin{minipage}{0.6\textwidth}
    Una forma de generalizar la función anterior es introduciendo un parámetro \(\alpha\) para ajustar la pendiente para entradas negativas. Esta función se llama PReLU y se define:
\begin{equation}
    f(x; \alpha) = 
    \begin{cases} 
      \alpha x & \text{si } x < 0 \\ 
      x & \text{si } x \geq 0 
    \end{cases}
    \label{eq:prelu}
\end{equation}
Cuando $\alpha = 0$, la función se corresponde con ReLU y cuando $\alpha > 0$, con la LReLU. La función se adapta durante el entrenamiento, permitiendo más flexibilidad que ReLU o LReLU. 
\end{minipage}

\subsubsection*{Función Softplus}

\begin{minipage}{0.6\textwidth}
    La función softplus puede considerarse una aproximación suave de la función ReLU. Se define como:
\begin{equation}
\text{softplus}(a) = \log(1 + \exp(a))
\end{equation}

La forma más suave de la función y la ausencia de puntos no diferenciables podrían sugerir un mejor comportamiento y un entrenamiento más fácil como función de activación. Sin embargo, los resultados experimentales obtenidos tienden a contradecir esta hipótesis, sugiriendo que las propiedades de ReLU pueden facilitar el entrenamiento supervisado mejor que las funciones softplus \citep{apicella2021survey}.
\end{minipage}
\begin{minipage}{0.05\textwidth}
\textbf{ }
\end{minipage}
\begin{minipage}{0.35\textwidth}
    \includegraphics[width=1.1\textwidth]{img/softplus.png}
    \captionof{figure}{Gráfica de la función Softplus.}
    \label{img:softplus}
\end{minipage}


\subsubsection*{Función ELU}

\begin{minipage}{0.35\textwidth}
    \includegraphics[width=1.1\textwidth]{img/elu.png}
    \captionof{figure}{Gráfica de la función ELU.}
    \label{img:elu}
\end{minipage}
\begin{minipage}{0.05\textwidth}
\textbf{ }
\end{minipage}
\begin{minipage}{0.6\textwidth}
    La \textbf{función ELU} es una función de activación que mantiene la identidad para argumentos positivos pero con valores no nulos para los negativos. Se define como:
    \[
    \text{ELU}(a) =
    \begin{cases} 
    a & \text{si } a > 0 \\
    \alpha \cdot (\exp(a) - 1) & \text{de lo contrario}
    \end{cases}
    \]
    donde \( \alpha \) controla el valor para entradas negativas. Los valores dados por las unidades ELU empujan la media de las activaciones más cerca de cero, permitiendo una fase de aprendizaje más rápida, a costa de un hiperparámetro extra (\( \alpha \)) que necesita ser establecido durante el entrenamiento \citep{apicella2021survey}.
\end{minipage}


\subsubsection*{Funciones Swish y SiLU}

\textbf{Swish y SiLU} combinan la función sigmoide con otros parámetros para proporcionar una activación suave y no lineal. Ambas funciones se han probado en tareas de aprendizaje por refuerzo y en otros contextos, demostrando su efectividad \citep{apicella2021survey}.

\begin{equation}
    \text{SiLU}(a) = a \cdot \sigma(a) = \frac{a}{1 + e^{-a}}
\end{equation}

La derivada de SiLU se define como:

\begin{equation}
    \text{dSiLU}(a) = \sigma(a) \left( 1 + a (1 - \sigma(a)) \right)
\end{equation}

Donde \(\sigma(a)\) es la función sigmoide.

La función Swish se define de manera similar, pero con un parámetro adicional \(\beta\):
\begin{equation}
    \text{Swish}(x) = x \cdot \sigma(\beta x) = \frac{x}{1 + e^{-\beta x}}
\end{equation}

Para \(\beta = 1\), Swish es equivalente a SiLU. La Figura \ref{fig:activf} muestra las gráficas de las funciones Swish (con \(\beta = 3\)) y SiLU.

\begin{figure}[h]
    \centering
    \begin{minipage}[b]{0.45\textwidth}
        \centering
        \includegraphics[width=\textwidth]{img/Swish.png}
        \caption{Gráfica de la función Swish con \(\beta = 3\).}
        \label{fig:swish}
    \end{minipage}
    \hfill
    \begin{minipage}[b]{0.45\textwidth}
        \centering
        \includegraphics[width=\textwidth]{img/SiLU.png}
        \caption{Gráfica de la función SiLU.}
        \label{fig:silu}
    \end{minipage}
    \caption{Comparación de las funciones de activación Swish y SiLU.}
    \label{fig:activf}
\end{figure}



Las funciones de activación juegan un papel esencial en la formación de redes neuronales al permitir que las redes aprendan y generen relaciones complejas. Desde funciones clásicas como la sigmoide y tanh, hasta variantes más recientes como ReLU, LReLU y ELU, cada función tiene sus propios beneficios y limitaciones. La elección de la función de activación depende del tipo de problema al que nos enfrentemos y de lo que se busque con el modelo, pero tiene que ser una elección acertada, ya que puede influir significativamente en el rendimiento del modelo. Además, permiten resolver el desvanecimiento del gradiente durante el proceso de aprendizaje, evitando que el gradiente se acerque a cero, cumpliendo la hipótesis de derivabilidad \citep{pajares2021aprendizaje}.


\section{Función de perdida}

En el aprendizaje supervisado, las funciones de pérdida son herramientas muy importantes para evaluar qué tan bien un modelo predice los resultados esperados. Su objetivo es cuantificar el grado de error en las predicciones realizadas por el modelo, lo cual es fundamental para el ajuste de sus parámetros durante el entrenamiento. En términos generales, estas funciones se dividen en dos grandes categorías: las utilizadas para problemas de clasificación y las utilizadas para problemas de regresión \citep{pajares2021aprendizaje}. Ambas buscan minimizar el error, pero lo hacen con diferentes enfoques dependiendo de la naturaleza de la variable de salida.

\subsection{Clasificación}

En los problemas de clasificación, la tarea del modelo es asignar una etiqueta a cada observación basada en las características de entrada. La etiqueta es una variable categórica que indica a cuál de varias categorías pertenece cada observación. Las funciones de pérdida en clasificación miden la discrepancia entre las etiquetas reales y las etiquetas predichas, o las probabilidades de las categorías predichas.

\subsubsection{Entropía Cruzada}

La entropía cruzada mide la diferencia entre la distribución de probabilidad verdadera de las etiquetas y la distribución de probabilidad predicha por el modelo. Existen dos variantes principales:

\begin{itemize}
	\item La \textbf{entropía cruzada categórica} se utiliza cuando se predicen múltiples categorías (multiclase). Se calcula como:

	\begin{equation}
    		H(p, q) = -\sum_{x} p(x) \log q(x)
	\end{equation}

	Aquí, \(p(x)\) es la distribución verdadera (generalmente representada como \textit{one-hot} para una categoría) y \(q(x)\) es la distribución predicha. Esta función penaliza las predicciones que se desvían de las probabilidades verdaderas, con valores más altos indicando mayores discrepancias \citep{pajares2021aprendizaje}. 

	\item La\textbf{ entropía cruzada binaria} se aplica a problemas de clasificación binaria, donde solo hay dos posibles etiquetas. Se define como:

	\begin{equation}
    		H(p, q) = - \left[ p \log q + (1 - p) \log (1 - q) \right]
	\end{equation}

	Aquí, \(p\) es la probabilidad de la etiqueta verdadera (1 para la clase positiva, 0 para la clase negativa), y \(q\) es la probabilidad predicha para la clase positiva. Esta variante es más sencilla y específica para problemas donde las etiquetas son binarias \citep{pajares2021aprendizaje}.
\end{itemize}



\subsubsection{Coeficiente de Dice}

El coeficiente de Dice es una métrica especialmente útil en problemas de segmentación de imágenes y clasificación. Esta métrica calcula la similitud entre el conjunto de píxeles predichos y el conjunto de píxeles verdaderos, y se define como:

\begin{equation}
    D = \frac{2 |P \cap G|}{|P| + |G|}
\end{equation}

donde \(P\) representa el conjunto de píxeles predichos y \(G\) representa el conjunto de píxeles verdaderos. La interpretación de esta función es que un valor más alto indica una mayor superposición entre las predicciones y la verdad del terreno, siendo 1 el valor ideal cuando hay coincidencia perfecta \cite{pajares2021aprendizaje}.

\subsubsection{Divergencia de Kullback-Leibler}

La divergencia de Kullback-Leibler (KL-divergence) mide la diferencia entre dos distribuciones de probabilidad: la distribución verdadera \(P\) y la distribución predicha \(Q\). Se expresa como:

\begin{equation}
    D_{KL}(P \| Q) = \sum_{x} P(x) \log \frac{P(x)}{Q(x)}
\end{equation}

Este valor indica cuánta información se pierde al usar \(Q\) para aproximar \(P\). En el contexto de modelos de clasificación, una menor divergencia KL indica que las distribuciones de probabilidades predichas se acercan más a las verdaderas, haciendo al modelo más preciso \cite{pajares2021aprendizaje}.

\subsection{Regresión}

En los problemas de regresión, la tarea del modelo es predecir una variable de salida continua basada en las características de entrada. Las funciones de pérdida en regresión cuantifican la discrepancia entre los valores continuos predichos y los valores reales.

\subsubsection{Error Cuadrático Medio (ECM)}

El error cuadrático medio (MSE, \textit{mean square error}) es una función de pérdida que calcula la media de los cuadrados de los errores entre las predicciones y los valores reales. Se define como:

\begin{equation}
    \text{MSE} = \frac{1}{n} \sum_{i=1}^{n} (y_i - \hat{y}_i)^2
\end{equation}

donde \(n\) es el número de muestras, \(y_i\) es el valor real y \(\hat{y}_i\) es el valor predicho. Esta función penaliza los errores grandes más severamente que los pequeños, debido al cuadrado del término de error. Un valor de MSE más bajo indica predicciones más cercanas a los valores reales \cite{pajares2021aprendizaje}.

\subsubsection{Error Absoluto Medio (EAM)}

El error absoluto medio (MAE, \textit{mean absolute error}) mide la media de los valores absolutos de las diferencias entre las predicciones y los valores reales. Está dado por:

\begin{equation}
    \text{MAE} = \frac{1}{n} \sum_{i=1}^{n} |y_i - \hat{y}_i|
\end{equation}

A diferencia del MSE, el MAE no penaliza tanto los errores grandes, ya que no los eleva al cuadrado. Esto hace que sea más robusto frente a los valores atípicos. Un MAE más bajo indica un modelo más preciso \cite{pajares2021aprendizaje}.

\subsubsection{Pérdida de Huber}

La función de pérdida de Huber combina las características del ECM y el EAM. Se comporta de manera cuadrática para errores pequeños y de manera lineal para errores grandes. Está definida como:

\begin{equation}
    L_\delta(y, \hat{y}) = 
    \begin{cases} 
      \frac{1}{2}(y - \hat{y})^2 & \text{si } |y - \hat{y}| \leq \delta \\ 
      \delta |y - \hat{y}| - \frac{1}{2}\delta^2 & \text{si } |y - \hat{y}| > \delta 
    \end{cases}
\end{equation}

El parámetro \(\delta\) controla el punto en el que la transición ocurre entre la penalización cuadrática y la lineal. Esto permite combinar la sensibilidad a los errores pequeños del ECM con la robustez frente a los errores grandes del EAM \cite{pajares2021aprendizaje}.

\subsubsection{Pérdida Log-Cosh}

La función de pérdida Log-cosh es otra opción para la regresión que suaviza los errores más que la pérdida cuadrática. Se define por:

\begin{equation}
    L = \sum_{i=1}^{n} \log(\cosh(\hat{y}_i - y_i))
\end{equation}

Aquí, \(\cosh\) es la función hiperbólica del coseno. La ventaja de esta función es que, para errores pequeños, se comporta como el ECM, mientras que para errores grandes se asemeja al EAM, lo que permite una suavización de los errores sin la severidad de la penalización cuadrática \cite{pajares2021aprendizaje}.

\subsubsection{Pérdida Basada en Cuantiles}

La función de pérdida basada en cuantiles (quantile loss) se utiliza en regresión cuando se quiere predecir un intervalo en lugar de un punto concreto. Se define como:

\begin{equation}
    L_{\tau}(y, \hat{y}) = \sum_{i=1}^{n} \left(\tau \max(y_i - \hat{y}_i, 0) + (1 - \tau) \max(\hat{y}_i - y_i, 0)\right)
\end{equation}

donde \(\tau\) es el cuantil elegido.














\section{Sobreajuste del modelo.}

El sobreajuste es uno de los principales desafíos a los que se enfrentan los expertos en machine learning a día de hoy. Para poder explicar y presentar diferentes soluciones a este problema, vamos a ver primero algunos conceptos.


El \textbf{sobreajuste} (también conocido como \textit{overfitting}) se produce cuando un modelo se ajusta demasiado bien a los datos de entrenamiento, capturando incluso el ruido y las fluctuaciones aleatorias. Este ajuste excesivo provoca que el modelo tenga un rendimiento excelente en los datos de entrenamiento pero un rendimiento deficiente en datos nuevos. El overfitting suele ocurrir cuando el modelo es demasiado complejo en comparación con la cantidad de datos disponibles.

El \textbf{subajuste} (también conocido como \textit{underfitting}) ocurre cuando el modelo es demasiado simple para capturar la estructura subyacente en los datos. Esto se manifiesta con un bajo ajuste del polinomio tanto a los datos de entrenamiento como a los de validación, indicando que el modelo no ha capturado patrones importantes en los datos.

En la primera gráfica de la Figura \ref{fig:overfitting}, se puede observar como el modelo no es capaz de captar la complejidad del problema, fallando en las predicciones tanto en los datos de entrenamiento como en los datos nuevos. En la segunda gráfica se puede ver que el modelo captura perfectamente todos los puntos de entrenamiento, pero fallará al predecir datos nuevos debido a que ha aprendido el ruido presente en los datos de entrenamiento.

\begin{figure}[h!]
    \centering
    \includegraphics[width=0.8\textwidth]{img/overfitting.png}
    \caption{Ejemplo de sobreajuste, subajuste y buen ajuste.}
    \label{fig:overfitting}
\end{figure}

Por último, la gráfica de más a la derecha presenta un ajuste idóneo para este conjunto de datos, ya que se puede observar como se ajusta a todos los puntos pero sin capturar ruido. Para llegar a este tipo de ajuste y evitar los dos primeros casos, vamos a explorar diferentes métodos para mitigar el sobreajuste, como la regularización $l1$ y $l2$, las capas dropout o la parada temprana.

\subsection{Regularización l1 y l2} \label{sec:regularizacion}

La regularización ayuda a evitar el sobreajuste, lo cual ocurre cuando un modelo es demasiado complejo y comienza a captar el ruido o los patrones irrelevantes en los datos de entrenamiento. En lugar de eso, la regularización favorece la creación de modelos más simples que pueden identificar los patrones importantes y generalizar mejor a datos nuevos. Esta técnica es especialmente útil cuando se tiene una cantidad limitada de datos, conjuntos de datos con muchas características o modelos con muchos parámetros \citep{pajares2021aprendizaje}.

Para implementarla, se añade un término de regularización a la función de pérdida durante el entrenamiento. Este término penaliza ciertos parámetros del modelo, ajustando el total de la pérdida. La intensidad de la regularización se controla mediante un parámetro que determina el equilibrio entre ajustar los datos y reducir el impacto de coeficientes muy grandes \citep{geron2022hands}.

\begin{equation}
loss_{regul}(\theta) = loss(\theta) + \beta \cdot f(\theta) 
\end{equation}

Aquí, \( loss_{regul}(\theta) \) es la función de pérdida final despues de aplicarle una regularización. \( loss(\theta) \) es la función de pérdida original que mide el ajuste del modelo a los datos, y \( \theta \) representa los parámetros del modelo (pesos y sesgos). El parámetro \(\beta\) controla la intensidad de la regularización, equilibrando entre ajustar el modelo a los datos y mantener los valores de los parámetros bajo control. Por último, el término de regularización \( f(\theta) \) penaliza los parámetros del modelo, siendo comúnmente la regularización $l2$ o la $l1$. 
 

La \textbf{regularización \(l2\)} (o regresión Ridge) añade una penalización proporcional a la suma de los pesos al cuadrado. La función de pérdida regularizada \(l2\) se define como:

\begin{equation}
\text{Loss} = \text{Original Loss} + \lambda \sum_{i} w_i^2
\end{equation}

donde \(\lambda\) es el parámetro de regularización que controla la importancia de la penalización y $w_i$ los pesos \citep{geron2022hands}. Este método tiende a producir soluciones más estables y reduce la complejidad de los modelos, tendiendo los coeficientes hacia valores más pequeños.


La \textbf{regularización \(l1\)} (o regresión Lasso) añade una penalización proporcional a la suma del valor absoluto de los pesos:
\begin{equation}
\text{Loss} = \text{Original Loss} + \lambda \sum_{i} |w_i|
\end{equation}
Esta técnica puede generar un modelo más sencillo, donde algunos coeficientes se anulan, seleccionando las características más importantes de los datos. Sin embargo, produce soluciones menos estables que aplicando $l2$ \citep{geron2022hands}. 


\subsection{Dropout} \label{sec:dropout}

Otro método para evitar el sobreajuste de nuestro modelo es el Dropout, que fue propuesto por Hinton et al. \citep{hinton2012improving} como una forma de regularización para capas de redes neuronales completamente conectadas. El dropout consiste en que cada elemento de la salida de una capa se mantiene en cada iteración con una probabilidad \(p\) (``tasa de dropout''), de lo contrario se establece en 0, con una probabilidad \((1 - p)\). Según \citep{geron2022hands}, este valor de $p$ suele variar entre 0.1 y 0.5, siendo este último uno de los más típicos \citep{srivastava2013improving}, aunque la red neuronal que se esté utilizando influye en el valor óptimo \citep{geron2022hands}. La elección de este parámetro es muy importnate, ya que si tomamos un valor muy alto, puede llevarnos a underfitting.
    
\begin{figure}[H]
    \centering
    \includegraphics[width=0.6\textwidth]{img/dropout.png}
    \captionof{figure}{Esquema de la técnica dropout. Fuente \citep{srivastava2014dropout}}
    \label{fig:dropout}
\end{figure}
    
Al eliminar una neurona, esta se elimina de la red con todas sus conexiones entrantes y salientes (ver Figura \ref{fig:dropout}). La principal ventaja de esto es que la red no depende excesivamente de ninguna neurona, mejorando de esta forma la generalización del modelo. La eliminación se aplica de manera independiente a cada capa oculta y a cada iteración del entrenamiento. 


Por lo tanto, aplicar dropout a una red neuronal equivale a seleccionar una submuestra ``reducida'' de la red original. Como cada neurona puede estar activa o inactiva, hay un total de $2^n$ redes posibles, donde $n$ es el número de neuronas que se pueden desactivar \citep{srivastava2013improving}. Puede interpretarse como entrenar una gran colección de redes neuronales diferentes y usar sus promedios para hacer predicciones.


Experimentos como los que se harán en el Capítulo \ref{Capitulo_3} muestran como el Dropout mejora la capacidad de generalización de la red, proporcionando un mejor rendimiento del modelo.


El uso del dropout en una red neuronal se implementa fácilmente en frameworks como Keras. Por ejemplo:

\lstset{language=Python}
\begin{lstlisting}
from tensorflow.keras.layers import Dropout

model = Sequential([
    Dense(128, activation='relu'),
    Dropout(0.5),
    Dense(10, activation='softmax') ])
\end{lstlisting}

En este código, se aplica dropout con una tasa del 50\% después de una capa densa con 128 unidades.



\subsection{Parada temprana}

Además de las técnicas de regularización $l1$ y $l2$ y el uso de dropout para prevenir el sobreajuste en redes neuronales, otra estrategia ampliamente utilizada es la parada temprana o \textit{Early Stopping}.

\begin{minipage}{0.35\textwidth}
\centering
    \includegraphics[width=0.8\textwidth]{img/earlyStopping.png}
    \captionof{figure}{Curvas de error de entrenamiento y validación con Early Stopping.}
    \label{fig:early_stopping}
\end{minipage}
\begin{minipage}{0.6\textwidth}
     El Early Stopping consiste en que el modelo deja de entrenar cuando el error en el conjunto de validación alcanza un mínimo y comienza a aumentar nuevamente. Esto se observa durante el proceso de entrenamiento, cuando el error de entrenamiento continúa disminuyendo mientras que el error de validación inicialmente disminuye, pero eventualmente comienza a incrementarse debido al sobreajuste \citep{geron2022hands}. La Figura \ref{fig:early_stopping} ilustra este comportamiento, donde la precisión del modelo en el conjunto de validación deja de mejorar después de cierto punto y comienza a empeorar.
     
\bigskip

Formalmente, si denotamos el error en el conjunto de entrenamiento después de la \( t \)-ésima época como \( E_{\text{tr}}(t) \) y el error en el conjunto de validación como \( E_{\text{val}}(t) \), Early Stopping interrumpe el entrenamiento cuando \( E_{\text{val}}(t) \) deja de mejorar durante un número definido de épocas consecutivas. 
\end{minipage}


Esta técnica no solo ayuda a prevenir el sobreajuste sino que también optimiza el uso de recursos al detener el entrenamiento cuando ya no se obtienen mejoras en el desempeño del modelo \citep{tian2022comprehensive}.

Para implementarlo en la práctica, frameworks como Keras ofrecen callbacks que facilitan este proceso. A continuación se presenta un ejemplo de código que utiliza \lstinline|EarlyStopping| para detener el entrenamiento cuando no se observa ninguna mejora en el error de validación:


\lstset{language=Python}
\begin{lstlisting}
from tensorflow.keras.callbacks import EarlyStopping

early_stopping_cb = keras.callbacks.EarlyStopping(patience=10,
                                           restore_best_weights=True)
history = model.fit(X_train, y_train, epochs=100,
                    validation_data=(X_valid, y_valid),
                    callbacks=[early_stopping_cb])
\end{lstlisting}

En este ejemplo, EarlyStopping detiene el entrenamiento después de un número definido de épocas sin mejora (\lstinline|patience = 10|), restaurando los pesos que lograron el mejor desempeño \citep{geron2022hands}.

Así, \textit{Early Stopping} se complementa eficazmente con las técnicas de regularización $l1$, $l2$ y dropout, proporcionando una capa adicional de protección contra el sobreajuste y mejorando la robustez del modelo.

\bigskip


El sobreajuste es uno de los principales desafíos en el entrenamiento de modelos de aprendizaje automático, incluyendo las redes neuronales. Técnicas como la regularización \(l2\), el dropout y la parada temprana son estrategias efectivas para mitigar este problema. Es crucial seleccionar y combinar adecuadamente estas técnicas según las características del problema y los datos disponibles. A través de estos métodos, podemos mejorar la capacidad de generalización de nuestros modelos y lograr un rendimiento más robusto en datos no vistos \citep{geron2022hands, pajares2021aprendizaje}.












\section{Métricas de evaluación de un modelo de clasificación.}

Para evaluar el rendimiento de un modelo de clasificación, se utilizan una serie de métricas que nos permiten entender el rendimiento y efectividad de un modelo. A continuación, se describen las principales métricas utilizadas, así como las representaciones gráficas asociadas.

\subsection{Matriz de Confusión}

La matriz de confusión es una herramienta que nos permite visualizar el rendimiento del modelo al mostrar los conteos de verdaderos positivos (TP), falsos positivos (FP), verdaderos negativos (TN) y falsos negativos (FN). Cada fila de la matriz representa las etiquetas reales mientras que cada columna representa las etiquetas predichas.

La matriz se define como:
\[
\begin{array}{c|cc}
 & \text{Predicción Positiva} & \text{Predicción Negativa} \\
\hline
\text{Real Positiva} & TP & FN \\
\text{Real Negativa} & FP & TN \\
\end{array}
\]

Para calcularla, utilizamos el conjunto de predicciones del modelo y las etiquetas reales. La matriz de confusión nos ofrece una visión clara de cómo se distribuyen los errores del modelo entre las diferentes clases.

\subsection{Métricas}

La matriz de confusión ofrece una gran cantidad de información, pero a veces te interesa una métrica más concreta para evaluar un modelo de clasificación, como la sensibilidad, la tasa de falsos negativos o su sensibilidad. A continuación vamos a definir y explicar las principales métricas a tener en cuenta según \citep{erickson2021magician} junto con su fórmula. 

La \textbf{sensibilidad} es la fracción de casos positivos que el modelo predice correctamente como positivos. También se conoce como recall o tasa de verdaderos positivos (TPR). Se calcula utilizando la fórmula:
\[
\text{Sensibilidad} = \frac{TP}{TP + FN}
\]

La \textbf{especificidad} es la fracción de casos negativos que el modelo predice correctamente como negativos. También se conoce como selectividad o tasa de verdaderos negativos (TNR). Se calcula utilizando la fórmula:
\[
\text{Especificidad} = \frac{TN}{TN + FP}
\]

La \textbf{tasa de falsos positivos} (FPR) es la fracción de casos negativos que el modelo predice incorrectamente como positivos. También se conoce como fall-out o probabilidad de alarma falsa. Se calcula utilizando la fórmula:
\[
\text{FPR} = \frac{FP}{TN + FP}
\]

La \textbf{tasa de falsos negativos} (FNR) es la fracción de casos positivos que el modelo predice incorrectamente como negativos. También se conoce como tasa de error tipo II o tasa de omisión. Se calcula utilizando la fórmula:
\[
\text{FNR} = \frac{FN}{TP + FN}
\]

El \textbf{valor predictivo positivo} (PPV) es la fracción de casos que el modelo predijo como positivos que realmente son positivos. También se conoce como precisión. Se calcula utilizando la fórmula:
\[
\text{PPV} = \frac{TP}{TP + FP}
\]

El \textbf{valor predictivo negativo} (NPV) es la fracción de casos que el modelo predijo como negativos que realmente son negativos. Se calcula utilizando la fórmula:
\[
\text{NPV} = \frac{TN}{TN + FN}
\]

El \textbf{accuracy} es la fracción de casos que el modelo predijo correctamente, ya sean positivos o negativos. Se calcula utilizando la fórmula:
\[
\text{Accuracy} = \frac{TP + TN}{TP + FN + TN + FP}
\]

Por último la \textbf{puntuación F1} es la media armónica del valor predictivo positivo y la sensibilidad. También se conoce como puntuación F, medida F o coeficiente de similitud de Dice. Se calcula utilizando la fórmula:
\[
\text{Puntuación F1} = \frac{2TP}{2TP + FP + FN}
\]



\subsection{Curva ROC y AUC}

\begin{minipage}{0.5\textwidth}
La curva ROC (Receiver Operating Characteristic) es otra herramienta utilizada en los modelos de clasificación \citep{geron2022hands}. Consiste en una representación gráfica que muestra la relación entre la tasa de verdaderos positivos (TPR) y la tasa de falsos positivos (FPR) para diferentes umbrales de clasificación. La linea discontinua de la figura \ref{img: roc} indica la curva ROC de un clasificador puramente aleatorio \citep{geron2022hands}. 

\bigskip

Otro término que está relacionado es el AUC (\textit{area under the curve}). El AUC es el área que se encuentra encerrada entre la curva ROC, la recta $y = 0$ y $x = 1$. Cuantifica la capacidad que tiene el modelo para distinguir entre las diferentes clases. Cuanto mayor valor, mejor rendimiento. En la Figura \ref{img: roc} se representa con rallas de intensidad baja. 
\end{minipage}
\begin{minipage}{0.05\textwidth}
\textbf{ }
\end{minipage}
\begin{minipage}{0.4\textwidth}
	\includegraphics[width=1.15\textwidth]{img/roc.png}
	\captionof{figure}{Interpretación de la curva ROC con el área bajo la curva (AUC).}
	\label{img: roc}
\end{minipage}

La interpretación de estos términos consiste en que una curva ROC más cercana al vértice superior izquierdo, o un AUC cercano a 1, indica un mejor rendimiento del modelo \citep{geron2022hands}.

\subsection{Curva de Precisión-Recall}

\begin{minipage}{0.4\textwidth}
	\includegraphics[width=1.15\textwidth]{img/pr.png}
	\captionof{figure}{Curva de Precisión-Recall.}
	\label{img: pr}
\end{minipage}
\begin{minipage}{0.05\textwidth}
\textbf{ }
\end{minipage}
\begin{minipage}{0.5\textwidth}
Por último, se estudiará la curva de Precisión-Recall (PR). Esta curva muestra la precisión frente a la sensibilidad para diferentes umbrales de decisión. Se identifica a partir de qué valor de sensibilidad comienza una disminución en la precisión, y viceversa. Idealmente, se busca una curva que se acerque lo más posible a la esquina superior derecha del gráfico, lo que representaría una combinación óptima de alta precisión y alto recall.

\bigskip

Esta curva es especialmente útil para ajustar un umbral de decisión que permita obtener una precisión tan alta como se desee, pero a costa de la sensibilidad. Por ejemplo, si se desea una precisión del 90\%, se puede aumentar el umbral has-
\end{minipage}

ta alcanzar este valor de precisión, aunque esto reducirá la sensibilidad. Este proceso se denomina trade-off entre precisión y sensibilidad.

Para calcular este umbral, se utilizan las puntuaciones de decisión obtenidas del clasificador. Primero, se obtienen estas puntuaciones usando la función \lstinline|decision_function()| del clasificador. Luego, se emplea la función \lstinline|precision_recall_curve()| de Scikit-Learn para calcular la precisión y la sensibilidad para todos los umbrales posibles. Finalmente, se selecciona el umbral que proporciona la precisión deseada. La Figura \ref{img: pr_vs_threshold} ilustra cómo se comportan la precisión y la sensibilidad en función del umbral \citep{geron2022hands}.

\begin{figure}[h]
	\centering
    \includegraphics[width=0.6\textwidth]{img/precision_recall_vs_threshold.png}
    \captionof{figure}{Precisión y Sensibilidad vs Umbral de decisión. Fuente \citep{geron2022hands}}
    \label{img: pr_vs_threshold}
\end{figure}











\section{Arquitecturas relevantes} \label{Subsec: 3_2}
Mini tabla resumen en Deep Cybersecurity: A Comprehensive Overview from Neural Network and Deep Learning Perspective y miniresumen de todos los tipos en review Deep Cybersecurity: A Comprehensive Overview from Neural Networkand Deep Learning Perspective y review

\subsection{Autoencoder}

Los autoencoders son una clase de redes neuronales artificiales utilizadas en aprendizaje no supervisado para aprender representaciones eficientes de los datos. Su funcionamiento consiste en codificar la entrada en una representación comprimida y significativa, y luego decodificarla de manera que la reconstrucción sea lo más similar posible a la entrada original \citep{lopes2022effective}. La arquitectura básica de un autoencoder consta de tres partes: el encoder, el cuello de bottela y el decoder (Figura \ref{fig:AE_architecture}). 

\begin{figure}[h]
    \centering
    \includegraphics[width=0.6\textwidth]{img/AE4.png}
    \caption{Arquitectura de un autoencoder. Fuente:\citep{autoencoderImage}.}
    \label{fig:AE_architecture}
\end{figure}

El encoder mapea los datos de entrada a una representación oculta de menor dimensión utilizando funciones principalmente no lineales, mientras que el decoder reconstruye los datos de entrada a partir de esta representación oculta. Durante el entrenamiento, los parámetros del autoencoder se optimizan para minimizar la diferencia entre la entrada y la salida reconstruida, utilizando una función de pérdida que mide esta discrepancia, como por ejemplo la pérdida de entropía cruzada. Esto concluye el proceso de entrenamiento de un autoencoder.


El problema, tal como se define formalmente en \cite{bank2023autoencoders}, es aprender las funciones
\[
A : \mathbb{R}^n \rightarrow \mathbb{R}^p \quad \text{(encoder)}
\]
y
\[
B : \mathbb{R}^p \rightarrow \mathbb{R}^n \quad \text{(decoder)}
\]
que satisfacen
\[
\arg \min_{A, B} \mathbb{E}[\Delta(x, B \circ A(x))],
\]
donde $\mathbb{E}$ es la esperanza sobre la distribución de $x$, y $\Delta$ es la función de pérdida de reconstrucción, que mide la distancia entre la salida del decodificador y la entrada.


Las ecuaciones para obtener la salida de un autoencoder serían:
\[
\left\{
\begin{aligned}
z^{(1)} &= W^{(1)} \cdot x + b^{(1)} \\
a^{(2)} &= f(z^{(1)})  \\
z^{(2)} &= W^{(2)} \cdot a^{(2)} + b^{(2)}  \\
y &= z^{(2)} 
\end{aligned}
\right.
\]

donde  $x$ es el input, \( b^{(1)} \) y  \( b^{(2)} \) son los sesgos,  \( W^{(1)} \) y  \( W^{(2)} \) son los pesos, \( z^{(1)} \) es la salida lineal de la primera capa, \( a^{(2)} \) es la activación de la segunda capa, \( z^{(2)} \) es la salida lineal de la segunda capa e \( y \) es la salida final del modelo \citep{martinez2017analisis}.


Los autoencoders se utilizan en una amplia variedad de aplicaciones, incluida la reducción de dimensionalidad, la extracción de características, la eliminación de ruido en los datos de entrada y la detección de anomalías. Su versatilidad y capacidad para aprender representaciones útiles de los datos los hacen herramientas poderosas.


Las principales capas que se utilizan en esta red neuronal son las capas densas y las de aplanamiento, aunque también se pueden utilizar capas convolucionales y de pooling para autocodificadores convolucionales\footnote{Autocodificador para imágenes de gran tamaño} o LSTM en el caso de autocodificadores recurrentes\footnote{Autocodificador específico para series temporales o secuencias.}\citep{geron2022hands}. 


COMENTAR LOS CONVOLUTIONAL AUTOENCODERS



\subsection{Deep Belief Networks}
\subsubsection{Red Neuronal Profunda}
\subsection{Red Neuronal Convolucional} \label{sec:2.CNN}


Las redes neuronales convolucionales son una de las métodos de machine learning más importantes y utilizados en el campo de la ciberseguridad. Estas redes neuronales están diseñadas para procesar entradas almacenadas en matrices, como las imágenes.  Son una parte de las redes profundas que procesa y analiza entradas de imágenes visuales, y están compuestas por neuronas con pesos y sesgos que aprenden a lo largo de su entrenamiento \citep{podder2021artificial}. La arquitectura de una CNN (Figura \ref{fig:cnn_architecture}) consta de tres tipos de capas: capas de convolución, capas de pooling y la capa de clasificación. 
 

\begin{figure}[h]
    \centering
    \includegraphics[width=0.6\textwidth]{img/convlayers.png}
    \caption{Arquitectura de una CNN con capas de convolución, pooling y clasificación. Fuente:\citep{phung2018deep}.}
    \label{fig:cnn_architecture}
\end{figure}

\subsubsection*{Capas convolucionales}

La capa convolucional es la capa más importante de una \acrshort{cnn}. En ella se extraen las características más significativas de la imagen de entrada, como los bordes, el color o la forma. Para ello se aplica una convolución a la imagen con un filtro. Esta operación matemática se representa como \( \int (x \star w)(t) \) donde $x$ representa la entrada y $w$ el núcleo de convolución \citep{pajares2021aprendizaje}.



En una CNN, la entrada de la convolución es una matriz multidimensional mientras que $w$ es una matriz de parámetros, llamada núcleo o filtro, que se ajusta durante el aprendizaje.
Cada píxel de la capa de convolución tiene una neurona, que se conecta con la capa anterior aplicando la convolución con las neuronas de su campo receptivo\footnote{Región de entrada que contribuye a la salida generada por el filtro} \cite{geron2022hands}. Esta convolución se realiza con un solapamiento total del filtro, lo que resulta en una imagen de menor dimensión (Figura \ref{fig:convolucion}). Si se desea mantener la misma dimensión, se puede aplicar zero-padding, que consiste en rellenar con ceros la matriz para obtener las dimensiones deseadas (Figura \ref{fig:convolucionPadding}). En la figura \ref{fig:convolu2} se muestran sendos campos receptivos de la imagen \( I \) que contribuyen a las salidas \( P \) y \( Q \) generadas por el filtro \( K \). La matriz de respuesta al aplicarle el kernel se llama mapa de características.

\begin{figure}[h] 
     \centering
     \begin{subfigure}[b]{0.45\textwidth}
         \centering
         \includegraphics[width=\textwidth]{img/convolucion.png}
         \caption{Convolución reduciendo tamaño.}
         \label{fig:convolucion}
     \end{subfigure}
     \hfill
     \begin{subfigure}[b]{0.45\textwidth}
         \centering
         \includegraphics[width=1\textwidth]{img/convolucionPadding.png}
         \caption{Convolución con zero-padding.}
         \label{fig:convolucionPadding}
     \end{subfigure}
     \caption{Convolución en 2D.}
     \label{fig:convolu2}
\end{figure} 

Se observa que el filtro se desplaza por la matriz \textit{I} con paso unitario en vertical y horizontal. Este parámetro se llama stride y su valor depende de el objetivo que se quiera lograr con esta capa convolucional.

\begin{figure}[h]
    \centering
    \includegraphics[width=0.6\textwidth]{img/stride.png}
    \caption{Campos receptivos en una convolución. Adaptación de imagen de \citep{yepez2020stride}}
    \label{fig:receptive_field}
\end{figure}

Una longitud de paso de 1 se utiliza normalmente para extraer el máximo número de características, ya que proporciona el máximo solapamiento entre el núcleo y la entrada. Por otro lado, cuando la longitud de paso es mayor que 1, los campos receptivos se solapan menos y producen una salida más pequeña. Si la longitud de paso fuera 3, habría problemas con el espaciado, ya que el campo receptivo no encajaría alrededor de la entrada como un número entero \citep{yepez2020stride}.



Por simplicidad, se ha usado siempre un único kernel, pero se puede generalizar a varios filtros, creando un mapa de características por cada uno. En cada uno de estos mapas hay una neurona por pixel y todas ellas comparten los mismos parámetros, lo que reduce considerablemente el número de parámetros del modelo. El campo receptivo de una neurona ahora se extiende por los mapas de características de todas las capas anteriores \citep{geron2022hands}.



Toda la información anterior se resume en la siguiente ecuación \citep{pajares2021aprendizaje}:

\begin{equation}
z_{i,j,k} = b_k + \sum_{u=0}^{f_h-1} \sum_{v=0}^{f_w-1} \sum_{k'=0}^{f_n'-1} x_{i',j',k'} \cdot w_{u,v,k',k} \hspace{5mm} \textup{con }
\left\{
\begin{array}{l}
i' = i \cdot s_h + u \\
j' = j \cdot s_w + v
\end{array}
\right.
\end{equation}

donde:
\begin{itemize}
    \item \( z_{i,j,k} \) es la salida de la neurona ubicada en la fila \(i\), columna \(j\) en el mapa de características \(k\) de la capa convolucional (capa \(l\)).
    \item \( s_h \) y \( s_w \) son los pasos de avance vertical y horizontal.
    \item \( f_h \) y \( f_w \) son la altura y la anchura del campo receptivo y \( f_{n'} \) es el número de mapas de características de la capa anterior (capa \(l-1\)).
    \item \( x_{i',j',k'} \) es la salida de la neurona situada en la fila \(i'\), columna \(j'\), mapa de características \(k'\).
    \item \( b_k \) es el sesgo para el mapa de características \(k\) (en la capa \(l\)).
    \item \( w_{u,v,k',k} \) es el peso de conexión entre cualquier neurona del mapa de características \(k\) de la capa \(l\) y su entrada situada en la fila \(u\), columna \(v\) (relativa al campo receptivo de la neurona) y el mapa de características \(k'\).
\end{itemize}





\subsubsection*{Capas de pooling}

El siguiente tipo de capa de las \acrshort{cnn} son las pooling, cuyo objetivo es reducir la imagen de entrada para disminuir la carga computacional, el uso de memoria y el número de parámetros, limitando así el riesgo de sobreajuste y proporcionando robustez contra el ruido y las distorsiones. Esta capa se suele colocar entre las capas de convolución, permitiendo reducir el tamaño de las imágenes mientras se preservan las características más importantes \citep{podder2021artificial}. Al igual que en las capas convolucionales, sus neuronas están conectadas a un pequeño grupo de neuronas de la capa anterior a las que se le aplica una función de agregación\footnote{Las funciones de agregación devuelven un valor único de un conjunto de registros.}. Las tres funciones más comunes son el promedio, la suma y el máximo. La Figura \ref{fig:maxpooling} muestra una capa de max pooling, que es el tipo más común \citep{geron2022hands}. 



\begin{figure}[h!]
\centering
\includegraphics[width=0.4\textwidth]{img/maxpooling.png}
\caption{Capa de max pooling con un kernel de \( 2 \times 2 \), stride 2 y sin padding.}
\label{fig:maxpooling}
\end{figure}

Además de reducir el número de operaciones, el número de parámetros y ayudar con el overfitting, una capa de max pooling introduce cierto nivel de invarianza a pequeñas translaciones, ya que si un pixel se traslada hacia la derecha, la salida también debería trasladarse un pixel hacia la derecha, como se ilustra en la Figura \ref{fig:translacionPooling}. Esto significa que pequeñas variaciones en la posición de las características dentro de la imagen no afectan significativamente la salida.


\begin{figure}[h!]
\centering
\includegraphics[width=0.4\textwidth]{img/translacionPooling.png}
\caption{Invarianza a translaciones pequeñas mediante una capa de max pooling. Fuente \citep{geron2022hands}}
\label{fig:translacionPooling}
\end{figure}



\subsubsection*{Capas Totalmente Conectadas (Fully Connected)}

Por último están las capas totalmente conectadas (\textit{fully connected}), que realizan la clasificación sobre la salida generada por las capas de convolución y pooling. Como el input de una capa densa debe ser un vector, primero se debe aplanar la salida de la última capa para poder utilizar después esta capa. Cada una de sus neuronas está conectada a todas las de la capa anterior, estableciendo una red densa de conexiones. Este tipo de neuronas suele ir seguido de una capa Dropout para mejorar la generalización del modelo. Este diseño permite a las CNN manejar datos complejos y variados, aprovechando la jerarquía de características aprendidas durante el entrenamiento. Este tipo de capa suele ir seguido de una capa de Dropout que mejora la capacidad de generalización del modelo al prevenir el sobreajuste, un problema común en el ámbito del aprendizaje profundo \citep{hossain2019classification}.


\begin{comment}
%Adicionalmente, las CNN pueden utilizar técnicas de regularización que ayudan a reducir el sobreajuste. Una de las técnicas más exitosas se llama \textit{dropout} \citep{srivastava2014dropout}. Al entrenar un modelo utilizando \textit{dropout}, durante cada iteración de entrenamiento, un porcentaje especificado de nodos en una capa dada y sus conexiones entrantes y salientes se eliminan aleatoriamente. Incluir \textit{dropout} típicamente mejora la precisión y la capacidad de generalización de un modelo porque aumenta la probabilidad de que un nodo sea útil.

\subsubsection{Aplicaciones y Éxitos}

Los usos de las CNN son significativamente variados. El mayor éxito se ha logrado en tareas de visión por computadora, como la detección y reconocimiento de objetos y escenas \citep{krizhevsky2012imagenet}. Las aplicaciones van desde la biología \citep{ronneberger2015u} hasta el reconocimiento facial \citep{parkhi2015deep}. El mejor ejemplo del éxito de las CNN tuvo lugar en 2012 en la competencia ImageNet, donde una CNN superó el rendimiento de otros métodos y luego la precisión humana en 2015 mediante el uso de GPUs, ReLUs, \textit{dropout} y la generación de imágenes adicionales \citep{he2015delving}. Además, las CNN se han utilizado con éxito en modelos de lenguaje para la detección de fonemas \citep{hinton2012deep}, reconocimiento de letras \citep{lecun1998gradient}, reconocimiento de voz \citep{graves2013speech} y construcción de modelos de lenguaje \citep{collobert2008unified}.



Una de las redes neuronales más importantes y más utilizadas en el campo de la ciberseguridad son la \acrfull{cnn}. Las CNN tienen su origen en el estudio de la corteza visual del cerebro y han sido utilizadas en el reconocimiento de imágenes desde la década de los 80. Actualmente, debido al aumento en la capacidad computacional, la disponibilidad de grandes volúmenes de datos de entrenamiento y las técnicas avanzadas de redes profundas, han permitido que las CNN alcancen un rendimiento excepcional en tareas visuales complejas. Estas redes son la base de servicios como la búsqueda de imágenes o los vehículos autónomos. Además, las CNN no se limitan únicamente a la percepción visual, también han demostrado ser efectivas en tareas como el reconocimiento de voz y el procesamiento del lenguaje natural \citep{geron2022hands}. No obstante, nos enfocaremos únicamente en sus aplicaciones visuales.

\end{comment}

\subsection{Red Neuronal Recurrente}
\subsubsection{Restricted Boltzmann Machine}









\section{Bibliotecas utilizadas en Python} \label{Subsec: 3_3}

Para nuestros experimentos, utilizaremos Python debido a su popularidad y versatilidad en el ámbito del aprendizaje automático y la inteligencia artificial. Python ofrece una amplia gama de bibliotecas especializadas que facilitan la creación, entrenamiento y evaluación de modelos, así como el análisis y visualización de datos. A continuación, se describen las principales bibliotecas y frameworks que emplearemos en este trabajo, destacando sus características y ventajas.


\subsection{Principales frameworks. Keras}

Como las técnicas de aprendizaje profundo han ido ganando popularidad, muchas organizaciones académicas e industriales se han centrado en desarrollar marcos para facilitar la experimentación con redes neuronales profundas. En esta sección, ofrecemos una visión general de los marcos de trabajo más importantes que se pueden usar en Python, concluyendo con nuestra elección.


\textbf{TensorFlow} \citep{tensorflow} es una biblioteca de código abierto desarrollada por el equipo de Google Brain para la computación numérica y el aprendizaje automático a gran escala. Diseñada para ser altamente flexible, TensorFlow soporta computación distribuida y permite la optimización de gráficos computacionales, lo que mejora significativamente la velocidad y el uso de memoria de las operaciones. En su núcleo, TensorFlow es similar a NumPy pero con soporte para GPU, lo que acelera considerablemente los cálculos. Además, incluye herramientas avanzadas como TensorBoard para la visualización de modelos y TensorFlow Extended para la producción de modelos de aprendizaje automático. Gracias a estas capacidades, TensorFlow se ha convertido en una herramienta esencial en la industria y la investigación, siendo utilizada en aplicaciones que van desde la clasificación de imágenes y el procesamiento de lenguaje natural hasta los sistemas de recomendación y la previsión de series temporales.


\textbf{Keras} \citep{keras} es una API de alto nivel para redes neuronales que ahora es parte integral de TensorFlow. Fue desarrollada por François Chollet y ganó popularidad rápidamente gracias a su simplicidad y diseño elegante. Inicialmente, Keras soportaba múltiples backends, pero desde la versión 2.4, funciona exclusivamente con TensorFlow \citep{muller2016introduction}. Keras permite a los usuarios construir, entrenar y evaluar modelos de aprendizaje profundo de manera rápida y eficiente. Su facilidad de uso y extensa documentación la convierten en una herramienta valiosa tanto para la investigación como para la implementación de aplicaciones de inteligencia artificial.



\textbf{PyTorch} \citep{pytorch}, desarrollado por el equipo de investigación de IA de Facebook, es una biblioteca de aprendizaje profundo que destaca por su enfoque en la computación dinámica, lo que permite una mayor flexibilidad en la creación de modelos complejos. A diferencia de TensorFlow, que utiliza gráficos computacionales estáticos, PyTorch permite que la topología de la red neuronal cambie durante la ejecución del programa \citep{mahmoud2019dlbench}. Esto, junto con su capacidad de auto-diferenciación en modo inverso\footnote{Técnica en la que PyTorch calcula automáticamente las derivadas de las funciones de pérdida con respecto a los parámetros del modelo.}, hace que PyTorch sea popular entre los investigadores y desarrolladores. Su facilidad de uso y robusta comunidad de apoyo han llevado a su adopción por parte de importantes organizaciones como Facebook, Twitter y NVIDIA.



Para escoger con cuál de estas librerías se realizará la parte práctica de este trabajo, vamos a utilizar, además de las características previamente vistas, los resultados de \citep{mahmoud2019dlbench}. En él se hace un estudio de eficiencia, convergencia, tiempo de entrenamiento y uso de memoria de los diferentes frameworks con varios datasets. Entre sus resultados podemos observar como Keras destaca por encima de las demás en el entorno de la CPU. No solo logra el mejor accuracy en los tres datasets (MNIST, CIFAR-10, CIFAR-100), sino que además también tiene los tiempos de ejecución más bajos y una de las mejores tasas de convergencia. En cuento al entorno de la GPU, las tres librerías obtienen unos resultados semejantes. En conclusión, podemos afirmar que estos resultados junto con su facilidad de uso, accesibilidad y documentación bien estructurada, han sido determinantes para optar por usar Keras en vez de PyTorch o TensorFlow en nuestros estudios posteriores. Aakash Nain resume perfectamente las ventajas de Keras \citep{keraswebsite2} al señalar que:

\begin{quote} 
``Keras is that sweet spot where you get flexibility for research and consistency for deployment. Keras is to Deep Learning what Ubuntu is to Operating Systems.'' 
\end{quote}

De manera similar, Matthew Carrigan destaca la intuitividad y facilidad de uso de Keras \citep{keraswebsite}, afirmando:

\begin{quote}
``The best thing you can say about any software library is that the abstractions it chooses feel completely natural, such that there is zero friction between thinking about what you want to do and thinking about how you want to code it. That's exactly what you get with Keras.''
\end{quote}


\subsection{Librerías y herramientas esenciales.} \label{sec:2.3.2}

De forma complementaria, también es importante conocer y utilizar diversas librerías y herramientas esenciales que facilitan el desarrollo y análisis de los modelos de Keras. Estas incluyen herramientas para la manipulación, visualización y análisis de datos.

\textbf{Scikit-Learn} \citep{scikitlearn} es una librería de código abierto con herramientas simples y eficientes para el análisis predictivo de datos. Contiene varios algoritmos de aprendizaje automático, desde clasificación y regresión hasta clustering y reducción de dimensionalidad, con la documentación completa sobre cada algoritmo. Está construida sobre otras librerías que veremos más adelante como Numpy, SciPy y matplotlib. Aunque no se aprovecharán todas estas funcionalidades de scikit-learn, si que se va a utilizar una de sus funciones más populares, \lstinline|train\_test\_split()| \citep{traintestsplit}. Esta función divide el dataset en dos subconjuntos de forma aleatoria, manteniendo la correspondencia en caso de que el dataset contenga dos o más partes. Usualmente, a estos subconjuntos se les llama conjunto de prueba y conjunto de entrenamiento, cuyo tamaño se indica con un valor entre 0 y 1 (\lstinline|test\_size|). Además, también se suele asignar una semilla a esa división para que cada vez que se quieran reproducir los experimentos, pueda usarse la misma partición. Esa semilla es un número natural que se introduce como parámetro de entrada en la variable \lstinline|random\_state|. Veamos un ejemplo de como utilizar esta función.


\lstset{language=Python}
\begin{lstlisting}
# Ejemplo de codigo en Python
from sklearn.model_selection import train_test_split

X_train, X_test, y_train, y_test = train_test_split(data, labels,
                                        test_size=0.25, random_state=42)
\end{lstlisting}


Las variables X\_train, X\_ test y compañía son numpy arrays. \textbf{NumPy} \citep{numpy} es el paquete fundamental de Python para la computación científica. Es una biblioteca general de estructuras de datos, álgebra lineal y manipulación de matrices para Python, cuya sintaxis y manejo de estructuras de datos y matrices es comparable al de MATLAB \citep{bloice2016tutorial}. En NumPy, se pueden crear arrays y realizar operaciones rápidas y eficientes sobre ellos. Se utilizarán estas estructuras de datos para almacenar los datos y entrenar las redes neuronales con ellas. Aunque también se pueden utilizar tensores \citep{modeltraining}, se ha decidido utilizar numpy arrays por su alta eficiencia operacional y por su uso en la industria.


Otro paquete que se va a utilizar durante los experimentos y que Scikit-Learn utiliza es \textbf{matplotlib} \citep{matplotlib}. Es la principal biblioteca de gráficos científicos en Python y proporciona funciones para crear visualizaciones de calidad como gráficos de barras, histogramas, gráficos de dispersión, etc. Se utilizará este paquete para representar gráficamente los datos de cada dataset para poder obtener bastante información con un simple vistazo. 










\begin{comment}


\section{perdida}
Como se ha comentado anteriormente, la función de pérdida se usa como medida de la precisión de un modelo de predicción en términos de predecir el resultado esperado. Es importante distinguir entre problemas de clasificación y regresión, ambas categorías bajo el mismo paraguas del aprendizaje supervisado. La tarea consiste en aprender (aproximar), con la máxima precisión posible, una función de proyección \(f\) desde una variable de entrada \(x\) a una variable de salida discreta \(y\), de forma que \(y = f(x)\), permitiendo que nuevos datos de entrada \(x\) puedan predecir la salida para un conjunto de datos nuevo. Clasificación y regresión comparten el mismo concepto de utilizar un conjunto de datos conocidos, llamados datos de entrenamiento, para realizar predicciones. La principal diferencia entre ellos es que la variable de salida en la regresión es numérica (continua) mientras que en la clasificación es categórica (discreta).


Algunas funciones de pérdida que distinguen entre clasificación y regresión se introducen a continuación.

\subsection{Clasificación}

\subsubsection{Entropía Cruzada}

Utilizada generalmente en clasificación, la entropía cruzada (cross-entropy), también conocida como \textit{log loss}, determina el desempeño del modelo de clasificación cuya salida es un valor de probabilidad entre 0 y 1. El valor de la función de entropía cruzada aumenta a medida que la probabilidad predicha diverge de la etiqueta actual. Se define como sigue en términos de probabilidad:

\begin{equation}
    H(p, q) = -\sum_{x} p(x) \log q(x)
\end{equation}

Por ejemplo, supongamos que se tiene un problema con tres categorías, A, B y C, de forma que una muestra de entrenamiento dada pertenece a la categoría B, para la que se ha asignado lo que se conoce como \textit{one-hot distribution}, con probabilidades \(p(A) = 0.0\), \(p(B) = 1.0\) y \(p(C) = 0.0\). Si el algoritmo predice la siguiente distribución de probabilidad \(p(A) = 0.2\), \(p(B) = 0.7\) y \(p(C) = 0.1\), el valor de la entropía cruzada \(H = - (0.0 \cdot \log(0.2) + 1.0 \cdot \log(0.7) + 0.0 \cdot \log(0.1)) = 0.357\).

\subsubsection{Coeficiente Dice}

Una función de pérdida utilizada en ocasiones para segmentación de imágenes y clasificación es el coeficiente Dice, propuesto inicialmente por Milletari et al. (2016). Se define como sigue:

\begin{equation}
    D = \frac{2 |P \cap G|}{|P| + |G|}
\end{equation}

donde \(P\) es el conjunto de píxeles predichos y \(G\) es el conjunto de píxeles del \textit{ground-truth}. 

\subsubsection{Divergencia de Kullback-Leibler}

Conocida como KL-divergence (Kullback y Leibler, 1951), mide la diferencia entre dos distribuciones de probabilidad \(P\) y \(Q\). Se define como sigue:

\begin{equation}
    D_{KL}(P \| Q) = \sum_{x} P(x) \log \frac{P(x)}{Q(x)}
\end{equation}

\subsection{Regresión}

\subsubsection{Error Cuadrático Medio (ECM)}

El error cuadrático medio (MSE, mean square error), también conocido como \textit{quadratic loss} o \textit{L2 loss}, se define como:

\begin{equation}
    \text{MSE} = \frac{1}{n} \sum_{i=1}^{n} (y_i - \hat{y}_i)^2
\end{equation}

donde \(n\) es el número de muestras de entrenamiento, \(y_i\) es el valor deseado y \(\hat{y}_i\) es el valor predicho.

\subsubsection{Error Absoluto Medio (EAM)}

El error absoluto medio (MAE, mean absolute error), también conocido como \textit{absolute loss} o \textit{L1 loss}, se define como:

\begin{equation}
    \text{MAE} = \frac{1}{n} \sum_{i=1}^{n} |y_i - \hat{y}_i|
\end{equation}

\subsubsection{Huber Loss}

La función de pérdida Huber combina ECM y EAM. Es cuadrática cuando el error es pequeño y lineal de otro modo. Se define como:

\begin{equation}
    L_\delta(y, \hat{y}) = 
    \begin{cases} 
      \frac{1}{2}(y - \hat{y})^2 & \text{si } |y - \hat{y}| \leq \delta \\ 
      \delta |y - \hat{y}| - \frac{1}{2}\delta^2 & \text{si } |y - \hat{y}| > \delta 
    \end{cases}
\end{equation}

\subsubsection{Log-Cosh Loss}

Log-cosh es otra de las funciones de pérdida utilizadas en regresión. Es más suave que \(L2\). Se define como:

\begin{equation}
    L = \sum_{i=1}^{n} \log(\cosh(\hat{y}_i - y_i))
\end{equation}



\subsection{Funciones de Activación No Lineales}
AÑADIR SOFTMAX
Una función de activación clásica es la función sigmoide definida como:

\begin{equation}
    f(a, x, c) = \frac{1}{1 + e^{-a(x-c)}}
\end{equation}

Dependiendo del signo del parámetro $a$, la función sigmoide se abre hacia la izquierda o hacia la derecha, siendo apropiada para representar conceptos tales como ``muy grande'' o ``muy negativo''. La Figura \ref{fig:sigmoid} muestra la representación de sendas funciones sigmoide: en (a) con los siguientes parámetros $a = 2$ y $c = 4$; y en (b) con $a = -2$ y $c = 4$.

\begin{figure}[h]
    \centering
    \includegraphics[width=0.8\textwidth]{fig2-4.png}
    \caption{Funciones sigmoide: (a) con $a=2$ y $c = 4$; (b) con $a = -2$ y $c = 4$.}
    \label{fig:sigmoid}
\end{figure}

La función sigmoide que proyecta salidas de números reales de entrada al intervalo $[0, 1]$ posee dos problemas:
\begin{enumerate}
    \item Saturación del gradiente. Cuando el valor de la función de activación se aproxima a los extremos 0 o 1, el gradiente de la función tiende a 0, lo que repercute en el ajuste de los pesos de las redes.
    \item Pesos positivos de forma continua. El valor medio de la función de salida no es 0, lo que origina que los pesos tiendan a ser positivos.
\end{enumerate}

Estas dos cuestiones provocan una convergencia lenta de los parámetros, afectando a la eficiencia del entrenamiento.

En \cite{Courbariaux2015}, se define lo que denominan función sigmoide dura (\textit{hard-sigmoid}) como:

\begin{equation}
    f(x) = \max(0, \min(1, \frac{x+1}{2}))
\end{equation}

La función tangente hiperbólica \textit{tanh} proporciona salidas reales en el rango definido, siendo una variante de la función sigmoide, definida exactamente como:

\begin{equation}
    \text{tanh}(x) = 2\cdot\sigma(2x) - 1
\end{equation}

presentando el mismo problema de la saturación del gradiente. La Figura \ref{fig:tanh} muestra la representación de la función \textit{tanh}.

\begin{figure}[h]
    \centering
    \includegraphics[width=0.8\textwidth]{fig2-5.png}
    \caption{Función tanh.}
    \label{fig:tanh}
\end{figure}

La función Unidad Lineal Rectificada (ReLU, \textit{Rectified Linear Unit}), $f(x) = \max(0, x)$, representada en la Figura \ref{fig:relu}(a) tiene las siguientes características:

\begin{enumerate}
    \item Gradiente no saturado. Por el hecho de que $x > 0$, el problema de la dispersión del gradiente en el proceso de propagación inversa se ve aliviado, y los parámetros en la primera capa de la red neuronal pueden actualizarse rápidamente. En $x = 0$ no es derivable, por lo que es habitual asignar un valor arbitrario en este caso, por ejemplo 0, 0.5 o bien 1.0.
    \item Baja complejidad computacional. Dada su propia definición. No obstante, posee la desventaja de que la neurona ReLU puede morir cuando recibe un gradiente negativo alto durante la retropropagación, lo que impide aprender más porque su derivada es cero cuando su entrada es menor que cero, por lo que el gradiente será finalmente cero. Esto se puede evitar al inicializar cuidadosamente los pesos o utilizar ReLU con ``fugas'', similar a ReLU, pero donde su salida es lineal multiplicada por un valor pequeño (aproximadamente 0.001) cuando la entrada es negativa, esto es $f(x) = \max(0.001x, x)$, tal y como se muestra en la Figura \ref{fig:relu}(b), conocida en ocasiones como \textit{Leaky ReLU} (LReLU, \textit{Leaky ReLU}).
\end{enumerate}

\begin{figure}[h]
    \centering
    \includegraphics[width=0.8\textwidth]{fig2-6.png}
    \caption{Funciones: (a) ReLU; (b) LReLU.}
    \label{fig:relu}
\end{figure}

En algunos tipos de redes como \textit{MobileNet}, que se estudiarán posteriormente, se define una variante de ReLU como sigue (es la función ReLU6), y cuya representación se muestra en la Figura \ref{fig:relu6}(a). A partir de ella se define \textit{hard-swish} o \textit{h-swish} (Hs) representada en la Figura \ref{fig:relu6}(b):

\begin{equation}
    \text{ReLU6}(x) = \min(\max(x, 0), 6)
\end{equation}

\begin{equation}
    \text{HS}(x) = x \cdot \frac{\text{ReLU6}(x + 3)}{6}
\end{equation}

\begin{figure}[h]
    \centering
    \includegraphics[width=0.8\textwidth]{fig2-7.png}
    \caption{Funciones: (a) ReLU6; (b) Hard-Swish.}
    \label{fig:relu6}
\end{figure}

La función Paramétrica ReLU (PReLU) se define según la ecuación \eqref{eq:prelu} \cite{He2015}:

\begin{equation}
    f(x; \alpha) = 
    \begin{cases} 
      \alpha x & \text{si } x < 0 \\ 
      x & \text{si } x \geq 0 
    \end{cases}
    \label{eq:prelu}
\end{equation}

de forma que si el parámetro $\alpha = 0$, la función es exactamente ReLU; si $\alpha > 0$, se trata de la función LReLU. Es cuando el parámetro $\alpha$ se incluye como un parámetro a aprender durante el proceso de entrenamiento, cuando la función toma su verdadero significado, de ahí su nombre.

Por otra parte, existe la Unidad lineal exponencial (ELU, \textit{Exponential Linear Unit}) definida en \cite{Clevert2016} como sigue con $\alpha > 0$:

\begin{equation}
    f(x; \alpha) = 
    \begin{cases} 
      \alpha (e^x - 1) & \text{si } x < 0 \\ 
      x & \text{si } x \geq 0 
    \end{cases}
\end{equation}

El parámetro $\alpha$ controla el valor para el cual se produce la saturación para valores de $x$ negativos. En la Figura \ref{fig:elu} se muestran sendas funciones ELU con valores de $\alpha = 0.1$ en (a) y 1.0 en (b), respectivamente.

\begin{figure}[h]
    \centering
    \includegraphics[width=0.8\textwidth]{fig2-8.png}
    \caption{Funciones ELU: (a) con $\alpha = 0.1$; (b) con $\alpha = 1.0$.}
    \label{fig:elu}
\end{figure}

La función SELU hace referencia a Unidad lineal exponencial escalada (SELU, \textit{Scaled Exponential Linear Unit}), siendo una versión ligeramente modificada de ELU por \cite{Klambauer2017} y definida como sigue:

\begin{equation}
    f(x; \alpha) = 
    \begin{cases} 
      \lambda x & \text{si } x \geq 0 \\ 
      \lambda \alpha (e^x - 1) & \text{si } x < 0 
    \end{cases}
\end{equation}

En la Figura \ref{fig:selu} se muestra la representación gráfica de la función SELU con valores $\alpha = 1.0$ y $\lambda = 1.0507$.

\begin{figure}[h]
    \centering
    \includegraphics[width=0.8\textwidth]{fig2-9.png}
    \caption{Función SELU}
    \label{fig:selu}
\end{figure}

Las funciones ReLU tienen la ventaja de acelerar el entrenamiento, ya que el cálculo del gradiente es más sencillo y la actualización de los pesos más rápida que con funciones sigmoides o \textit{tanh}. No obstante, en algunos casos pueden ser menos precisas que las funciones tradicionales y pueden llevar a neuronas muertas en redes profundas. Las funciones LReLU y PReLU abordan este problema permitiendo gradientes pequeños cuando la entrada es negativa. ELU y SELU mejoran aún más la estabilidad del gradiente en redes profundas al permitir que las activaciones negativas tengan valores ajustables en el rango negativo, lo que puede ayudar en la normalización de las entradas.


La función Swish (Ramachandran et al., 2017) se define como el producto del argumento \(x\) por la función sigmoide parametrizada con \(\beta = 3\) y, si \(\beta = 1\), se define como la Unidad Lineal Sigmoide Ponderada (SiLU, Sigmoid-weighted Linear Unit) (Elfwing et al., 2017), existiendo también su versión derivada (dSiLU). Las ecuaciones son las siguientes:

\begin{equation}
    \text{SiLU}: f(x) = x \cdot \sigma(\beta x) = \frac{x}{1 + e^{-\beta x}}
\end{equation}

\begin{equation}
    \text{dSiLU}: f'(x) = \sigma(x) \left(1 + x (1 - \sigma(x))\right)
\end{equation}

donde \(\sigma(x)\) es la función sigmoide definida como:

\begin{equation}
    \sigma(x) = \frac{1}{1 + e^{-x}}
\end{equation}

La Figura \ref{fig:swish_silu} muestra las funciones Swish con \(\beta = 3\) y SiLU con \(\beta = 1\).

\begin{figure}[H]
    \centering
    \includegraphics[width=0.8\textwidth]{fig2-11.png}
    \caption{Funciones: (a) Swish con \(\beta = 3\); (b) SiLU con \(\beta = 1\).}
    \label{fig:swish_silu}
\end{figure}

La función Mish se define como sigue (Misra, 2019):

\begin{equation}
    f(x) = x \cdot \tanh(\ln(e^x + 1))
\end{equation}

La Figura \ref{fig:mish} muestra la representación de la función Mish.

\begin{figure}[H]
    \centering
    \includegraphics[width=0.8\textwidth]{fig2-12.png}
    \caption{Función Mish.}
    \label{fig:mish}
\end{figure}
\end{comment}

\begin{comment}
¿QUITAR?
La ecuación \ref{eq:conv} resume las explicaciones anteriores en una gran ecuación matemática \citep{pajares2021aprendizaje}:

\begin{equation}
z_{i,j,k} = b_k + \sum_{u=0}^{f_h-1} \sum_{v=0}^{f_w-1} \sum_{k'=0}^{f_n'-1} x_{i',j',k'} \cdot w_{u,v,k',k} \hspace{5mm} \textup{con }
\left\{
\begin{array}{l}
i' = i \cdot s_h + u \\
j' = j \cdot s_w + v
\end{array}
\right.
\end{equation} \label{eq:conv}

donde \( z_{i,j,k} \) es la salida de la neurona ubicada en \((i,j)\) en el mapa de características \(k\) de la capa convolucional \(l\). \( s_h \) y \( s_w \) es el stride vertical y horizontal. \( f_h \) y \( f_w \) son las dimensiones del campo receptivo y \( f_{n'} \) es el número de mapas de características de la capa anterior. \( x_{i',j',k'} \) es la salida de la neurona situada en \((i',j')\) y en mapa de características \(k'\). \( b_k \) es el término de sesgo para el mapa de características \(k\) en la capa \(l\) y \( w_{u,v,k',k} \) es el peso de conexión entre cualquier neurona del mapa de características \(k\) de la capa \(l\) y su entrada en \((u,v)\) y el mapa de características \(k'\).
\end{comment}



\begin{comment}
Una de las redes neuronales más importantes y más utilizadas en el campo de la ciberseguridad son las \acrfull{cnn}. Las CNN tienen su origen en el estudio de la corteza visual del cerebro y han sido utilizadas en el reconocimiento de imágenes desde la década de los 80. Actualmente, debido al aumento en la capacidad computacional, la disponibilidad de grandes volúmenes de datos de entrenamiento y las técnicas avanzadas de redes profundas, han permitido que las CNN alcancen un rendimiento excepcional en tareas visuales complejas. Estas redes son la base de servicios como la búsqueda de imágenes o los vehículos autónomos \citep{geron2022hands}.


Una \acrfull{cnn} es una red neuronal diseñada para procesar entradas almacenadas en matrices. Un ejemplo de entrada es una imagen en escala de grises, que es una matriz bidimensional (2D) de píxeles. Aunque estas redes se utilicen principalmente en la clasificación visual de imágenes, también  se han demostrado que son efectivas en tareas como el reconocimiento de voz (matrices 2D de imágenes o espectrogramas de audio) \citep{kim2023bilstm}, clasificación visual de videos o imágenes volumétricas (matrices tridimensionales 3D) \citep{diba2017temporal} y procesamiento de lenguaje natural(matriz 2D) \citep{wang2017combining}. Independientemente de la dimensionalidad, las CNN se utilizan donde hay un ordenamiento espacial o temporal \citep{berman2019survey}. No obstante, nos enfocaremos únicamente en sus aplicaciones visuales.

La arquitectura de una CNN (ver Figura ????????) consta de tres tipos distintos de capas: capas de convolución, capas de pooling y la capa de clasificación. 

IMAGEN CAPAS: \citep{phung2018deep}


\subsubsection*{Capas convolucionales}

La capa convolucional es el bloque de construcción más importante de una red neuronal convolucional (CNN). En ella se aplica la operación de convolución, que involucra dos funciones y produce una tercera función que representa la cantidad de superposición de una función a medida que se desplaza sobre otra. Supóngase que se tiene una fuente de luz variable cuya intensidad se recibe mediante un sensor. Este sensor proporciona una salida en una determinada posición \( x \) en el tiempo \( t \), esto es, \( x(t) \). Tanto \( x \) como \( t \) son valores reales, de forma que debido a la variabilidad de la fuente mencionada, se pueden obtener diferentes lecturas en diferentes instantes de tiempo. La captura de la señal por el sensor puede estar contaminada con cierto ruido. Para obtener una señal más limpia, lo acertado es realizar un promediado de la salida con varias medidas. Si además tenemos en cuenta que las medidas más recientes son más relevantes que las alejadas en el tiempo, el promediado puede ponderarse concediendo más relevancia a las medidas recientes. Esto puede hacerse mediante una función de promediado \( w(a) \), donde \( a \) representa el alejamiento de la medida en el tiempo. Si se realiza esta operación de promediado ponderado en cada instante de tiempo \( t \), se obtiene una nueva función promediada como sigue:

\begin{equation}
s(t) = \int x(a) w(t-a) \, da
\end{equation}

Esta operación se denomina convolución, y se denota como:

\begin{equation}
s(t) = \int (x \star w)(t)
\end{equation}

Para que el promedio ponderado sea válido, la función de promediado \( w(t) \) necesita ser una función de densidad de probabilidad. Además, \( w(t) \) debe ser cero para todos los valores negativos de \( t \) para evitar tomar valores futuros, lo cual no es físicamente posible \citep{pajares2021aprendizaje}.


Para aplicaciones prácticas como el procesamiento de imágenes, las medidas suelen ser discretas, es decir, se toman en intervalos de tiempo específicos. En este caso, el tiempo \( t \) toma valores enteros, y la convolución discreta se define como:

\begin{equation}
s(t) = (x \star w)(t) = \sum_{a=-\infty}^{\infty} x(a) w(t-a)
\end{equation}

En el contexto de las CNN, la entrada es un vector o matriz multidimensional de datos (la imagen), y el núcleo es una matriz multidimensional de parámetros que se ajustan durante el proceso de aprendizaje. Sea \( I \) la imagen de entrada y \( K \) un núcleo de convolución. Se define la convolución discreta en dos dimensiones como:

\begin{equation}
S(i, j) = (K \star I)(i, j) = \sum_{m=0}^{M-1} \sum_{n=0}^{N-1} I(i+m, j+n) K(m, n)
\end{equation}

donde \( M \) y \( N \) son las dimensiones del núcleo. La figura ????? muestra un ejemplo de convolución con el núcleo \( K \), aplicado sobre una imagen \( I \). 


En una CNN, esta construcción es la mas importante. Las neuronas de la primera capa convolucional se concectan a todos y cada uno de los píxeles de la imagen de entrada, sino solo a píxeles en sus campos receptivos. A su vez, cada neurona de la segunda capa convolucional está conectada a las neuronas ubicadas dentro de un rectángulo pequeño en la primera capa. Esta arquitectura permite a la red concentrarse en cracterísticas pequeñas de bajo nivel en la primera capa oculta, juntarlas después para crear características más grandes de nivel superior en la siguiente capa oculta, y así sucesivamente. La convolución se realiza con solapamiento total del núcleo, lo que resulta en una imagen resultante de menor dimensión (REFERENCIA IMAGEN A ?????). Si se desea mantener la misma dimensión, se puede aplicar relleno con ceros o zero-padding (REFERENCIA IMAGEN A ?????). Este relleno se indica como "same" en términos de implementación. Si no se realiza relleno con ceros, se indica como "valid". Por otra parte, en las convoluciones, el campo receptivo se define como la región de entrada que contribuye a la salida generada por el filtro. En la figura REFERENCIA IMAGEN A ?????) se muestran sendos campos receptivos de la imagen \( I \) que contribuyen a las salidas \( P \) y \( \Q \) generadas por el filtro \( K \).


IMAGENES PADDING

Por último, se define el stride. Stride es la longitud de paso de desplazamiento del núcleo alrededor de la entrada. Cuando la longitud de paso es 1, el núcleo se desplaza sobre la entrada un elemento a la vez. Normalmente, la longitud de paso se establece de manera que el volumen de salida sea un número entero.

FOTOOO stride Fuente:\citep{yepez2020stride}

La Figura de la izquierda muestra una entrada de \(5 \times 5\) con un núcleo de \(3 \times 3\). Con un stride de paso 1, se genera una matriz de salida de tamaño \(3 \times 3\). Una longitud de paso de 1 se utiliza normalmente para extraer el máximo número de características, ya que proporciona el máximo solapamiento entre el núcleo y la entrada, pero a máxima complejidad computacional. Por otro lado, la imagen de la derecha muestra el núcleo desplazándose dos unidades sobre la entrada, generando una matriz de salida de \(2 \times 2\). Generalmente, cuando la longitud de paso es mayor que 1, los campos receptivos se solapan menos y producen una salida más pequeña. Si la longitud de paso fuera 3, habría problemas con el espaciado, ya que el campo receptivo no encajaría alrededor de la entrada como un número entero \citep{yepez2020stride}.









cada neurona en la capa convolucional está conectada solo a una pequeña región de la imagen de entrada (campo receptivo). Estas conexiones se realizan utilizando filtros (o núcleos de convolución), y la salida se conoce como mapa de características. La ecuación \ref{eq:conv} resume las explicaciones anteriores en una gran ecuación matemática \citep{pajares2021aprendizaje}:

\begin{equation}
z_{i,j,k} = b_k + \sum_{u=0}^{f_h-1} \sum_{v=0}^{f_w-1} \sum_{k'=0}^{f_n'-1} x_{i',j',k'} \cdot w_{u,v,k',k} \hspace{5mm} \textup{con }
\left\{
\begin{array}{l}
i' = i \cdot s_h + u \\
j' = j \cdot s_w + v
\end{array}
\right.
\end{equation}

donde:
\begin{itemize}
    \item \( z_{i,j,k} \) es la salida de la neurona ubicada en la fila \(i\), columna \(j\) en el mapa de características \(k\) de la capa convolucional (capa \(l\)).
    \item \( s_h \) y \( s_w \) son los pasos de avance vertical y horizontal.
    \item \( f_h \) y \( f_w \) son la altura y la anchura del campo receptivo y \( f_{n'} \) es el número de mapas de características de la capa anterior (capa \(l-1\)).
    \item \( x_{i',j',k'} \) es la salida de la neurona situada en la fila \(i'\), columna \(j'\), mapa de características \(k'\) (o canal \(k'\) si la capa anterior es la capa de entrada).
    \item \( b_k \) es el término de sesgo para el mapa de características \(k\) (en la capa \(l\)). Puedes imaginarlo como una rueda que ajusta el brillo general del mapa de características \(k\).
    \item \( w_{u,v,k',k} \) es el peso de conexión entre cualquier neurona del mapa de características \(k\) de la capa \(l\) y su entrada situada en la fila \(u\), columna \(v\) (relativa al campo receptivo de la neurona) y el mapa de características \(k'\).
\end{itemize}









\subsection{Red Neuronal Convolucional}

Una de las redes neuronales más importantes y más utilizadas en el campo de la ciberseguridad son las \acrfull{cnn}. Las CNN tienen su origen en el estudio de la corteza visual del cerebro y han sido utilizadas en el reconocimiento de imágenes desde la década de los 80. Actualmente, debido al aumento en la capacidad computacional, la disponibilidad de grandes volúmenes de datos de entrenamiento y las técnicas avanzadas de redes profundas, han permitido que las CNN alcancen un rendimiento excepcional en tareas visuales complejas. Estas redes son la base de servicios como la búsqueda de imágenes o los vehículos autónomos \citep{geron2022hands}.

Una \acrfull{cnn} es una red neuronal diseñada para procesar entradas almacenadas en matrices. Un ejemplo de entrada es una imagen en escala de grises, que es una matriz bidimensional (2D) de píxeles. Aunque estas redes se utilicen principalmente en la clasificación visual de imágenes, también se ha demostrado que son efectivas en tareas como el reconocimiento de voz (matrices 2D de imágenes o espectrogramas de audio) \citep{kim2023bilstm}, clasificación visual de videos o imágenes volumétricas (matrices tridimensionales 3D) \citep{diba2017temporal} y procesamiento de lenguaje natural (matriz 2D) \citep{wang2017combining}. Independientemente de la dimensionalidad, las CNN se utilizan donde hay un ordenamiento espacial o temporal \citep{berman2019survey}. No obstante, nos enfocaremos únicamente en sus aplicaciones visuales.

La arquitectura de una CNN (ver Figura ????????) consta de tres tipos distintos de capas: capas de convolución, capas de pooling y la capa de clasificación.

\begin{figure}[h]
    \centering
    \includegraphics[width=0.8\textwidth]{path_to_image}
    \caption{Arquitectura de una CNN con capas de convolución, pooling y clasificación \citep{phung2018deep}.}
    \label{fig:cnn_architecture}
\end{figure}

\subsubsection*{Capas convolucionales}

La capa convolucional es el bloque de construcción más importante de una \fullacr{cnn}. En ella se aplica la operación de convolución, que involucra dos funciones y produce una tercera función que representa la cantidad de superposición de una función a medida que se desplaza sobre otra. Supóngase que se tiene una fuente de luz variable cuya intensidad se recibe mediante un sensor. Este sensor proporciona una salida en una determinada posición \( x \) en el tiempo \( t \), esto es, \( x(t) \). Tanto \( x \) como \( t \) son valores reales, de forma que debido a la variabilidad de la fuente mencionada, se pueden obtener diferentes lecturas en diferentes instantes de tiempo. La captura de la señal por el sensor puede estar contaminada con cierto ruido. Para obtener una señal más limpia, lo acertado es realizar un promediado de la salida con varias medidas. Si además tenemos en cuenta que las medidas más recientes son más relevantes que las alejadas en el tiempo, el promediado puede ponderarse concediendo más relevancia a las medidas recientes. Esto puede hacerse mediante una función de promediado \( w(a) \), donde \( a \) representa el alejamiento de la medida en el tiempo. Si se realiza esta operación de promediado ponderado en cada instante de tiempo \( t \), se obtiene una nueva función promediada como sigue:

\begin{equation}
s(t) = \int x(a) w(t-a) \, da
\end{equation}

Esta operación se denomina convolución, y se denota como:

\begin{equation}
s(t) = \int (x \star w)(t)
\end{equation}

Para que el promedio ponderado sea válido, la función de promediado \( w(t) \) necesita ser una función de densidad de probabilidad. Además, \( w(t) \) debe ser cero para todos los valores negativos de \( t \) para evitar tomar valores futuros, lo cual no es físicamente posible \citep{pajares2021aprendizaje}.


Para aplicaciones prácticas como el procesamiento de imágenes, las medidas suelen ser discretas, es decir, se toman en intervalos de tiempo específicos. En este caso, el tiempo \( t \) toma valores enteros, y la convolución discreta se define como:

\begin{equation}
s(t) = (x \star w)(t) = \sum_{a=-\infty}^{\infty} x(a) w(t-a)
\end{equation}

En el contexto de las CNN, la entrada es un vector o matriz multidimensional de datos (la imagen), y el núcleo es una matriz multidimensional de parámetros que se ajustan durante el proceso de aprendizaje. Sea \( I \) la imagen de entrada y \( K \) un núcleo de convolución. Se define la convolución discreta en dos dimensiones como:

\begin{equation}
S(i, j) = (K \star I)(i, j) = \sum_{m=0}^{M-1} \sum_{n=0}^{N-1} I(i+m, j+n) K(m, n)
\end{equation}

donde \( M \) y \( N \) son las dimensiones del núcleo. La figura \ref{fig:conv_example} muestra un ejemplo de convolución con el núcleo \( K \), aplicado sobre una imagen \( I \).

\begin{figure}[h]
    \centering
    \includegraphics[width=0.8\textwidth]{path_to_image}
    \caption{Ejemplo de convolución con un núcleo \( K \) aplicado sobre una imagen \( I \).}
    \label{fig:conv_example}
\end{figure}

En una CNN, esta construcción es la más importante. Las neuronas de la primera capa convolucional no se conectan a todos y cada uno de los píxeles de la imagen de entrada, sino solo a píxeles en sus campos receptivos. A su vez, cada neurona de la segunda capa convolucional está conectada a las neuronas ubicadas dentro de un rectángulo pequeño en la primera capa. Esta arquitectura permite a la red concentrarse en características pequeñas de bajo nivel en la primera capa oculta, juntarlas después para crear características más grandes de nivel superior en la siguiente capa oculta, y así sucesivamente. La convolución se realiza con solapamiento total del núcleo, lo que resulta en una imagen resultante de menor dimensión (ver Figura \ref{fig:padding_valid}). Si se desea mantener la misma dimensión, se puede aplicar relleno con ceros o zero-padding (ver Figura \ref{fig:padding_same}). Este relleno se indica como "same" en términos de implementación. Si no se realiza relleno con ceros, se indica como "valid". Por otra parte, en las convoluciones, el campo receptivo se define como la región de entrada que contribuye a la salida generada por el filtro. En la figura \ref{fig:receptive_field} se muestran sendos campos receptivos de la imagen \( I \) que contribuyen a las salidas \( P \) y \( Q \) generadas por el filtro \( K \).

\begin{figure}[h]
    \centering
    \includegraphics[width=0.8\textwidth]{path_to_image}
    \caption{Convolución con relleno "valid".}
    \label{fig:padding_valid}
\end{figure}

\begin{figure}[h]
    \centering
    \includegraphics[width=0.8\textwidth]{path_to_image}
    \caption{Convolución con relleno "same".}
    \label{fig:padding_same}
\end{figure}

\begin{figure}[h]
    \centering
    \includegraphics[width=0.8\textwidth]{path_to_image}
    \caption{Campos receptivos en una convolución.}
    \label{fig:receptive_field}
\end{figure}

Por último, se define el stride. Stride es la longitud de paso de desplazamiento del núcleo alrededor de la entrada. Cuando la longitud de paso es 1, el núcleo se desplaza sobre la entrada un elemento a la vez. Normalmente, la longitud de paso se establece de manera que el volumen de salida sea un número entero.

\begin{figure}[h]
    \centering
    \includegraphics[width=0.8\textwidth]{path_to_image}
    \caption{Ejemplo de convolución con diferentes strides. Fuente: \citep{yepez2020stride}.}
    \label{fig:stride_example}
\end{figure}

La Figura \ref{fig:stride_example} de la izquierda muestra una entrada de \(5 \times 5\) con un núcleo de \(3 \times 3\). Con un stride de paso 1, se genera una matriz de salida de tamaño \(3 \times 3\). Una longitud de paso de 1 se utiliza normalmente para extraer el máximo número de características, ya que proporciona el máximo solapamiento entre el núcleo y la entrada, pero a máxima complejidad computacional. Por otro lado, la imagen de la derecha muestra el núcleo desplazándose dos unidades sobre la entrada, generando una matriz de salida de \(2 \times 2\). Generalmente, cuando la longitud de paso es mayor que 1, los campos receptivos se solapan menos y producen una salida más pequeña. Si la longitud de paso fuera 3, habría problemas con el espaciado, ya que el campo receptivo no encajaría alrededor de la entrada como un número entero.



La ecuación \ref{eq:conv} resume las explicaciones anteriores en una gran ecuación matemática \citep{pajares2021aprendizaje}:

\begin{equation}
z_{i,j,k} = b_k + \sum_{u=0}^{f_h-1} \sum_{v=0}^{f_w-1} \sum_{k'=0}^{f_n'-1} x_{i',j',k'} \cdot w_{u,v,k',k} \hspace{5mm} \textup{con }
\left\{
\begin{array}{l}
i' = i \cdot s_h + u \\
j' = j \cdot s_w + v
\end{array}
\right.
\end{equation}

donde:
\begin{itemize}
    \item \( z_{i,j,k} \) es la salida de la neurona ubicada en la fila \(i\), columna \(j\) en el mapa de características \(k\) de la capa convolucional (capa \(l\)).
    \item \( s_h \) y \( s_w \) son los pasos de avance vertical y horizontal.
    \item \( f_h \) y \( f_w \) son la altura y la anchura del campo receptivo y \( f_{n'} \) es el número de mapas de características de la capa anterior (capa \(l-1\)).
    \item \( x_{i',j',k'} \) es la salida de la neurona situada en la fila \(i'\), columna \(j'\), mapa de características \(k'\) (o canal \(k'\) si la capa anterior es la capa de entrada).
    \item \( b_k \) es el término de sesgo para el mapa de características \(k\) (en la capa \(l\)).
    \item \( w_{u,v,k',k} \) es el peso de conexión entre cualquier neurona del mapa de características \(k\) de la capa \(l\) y su entrada situada en la fila \(u\), columna \(v\) (relativa al campo receptivo de la neurona) y el mapa de características \(k'\).
\end{itemize}
\end{comment}



\begin{comment}
Dentro del AA se encuadra el conocido como Aprendizaje Profundo (AP). En la literatura especializada a nivel internacional es muy común referirse al AP por sus términos en inglés, esto es, Deep Learning, que es el núcleo central del presente libro. Por otra parte, y al hilo de esta cuestión, conviene reseñar que son muchos y diversos los términos en inglés utilizados para definir y describir los conceptos involucrados bajo el paradigma del AP. Muchos de los cuales no poseen una clara traducción conceptual al español, razón por la cual los considerados bajo esta situación se mantienen a lo largo del libro con el fin de que el lector pueda fácilmente identificarlos en la literatura especializada escrita en inglés. Solo se han traducido aquellos conceptos que no admiten discusión, manteniendo en todo caso su expresión original en inglés.

Bien es cierto que desde los años 50 del siglo pasado, la IA a veces se ha sobrevalorado y se ha considerado como muy prometedora en diversas ocasiones, eso pese a que no se han llegado a alcanzar las perspectivas iniciales. Por otra parte, no es menos cierto que en los últimos años se están viendo avances importantes gracias al AP. Ello a pesar de que todavía es relativamente frágil de cara a su generalización y adaptación a entornos o escenarios cambiantes, principalmente por falta de datos suficientes que capturen los cambios de dicho entorno, pudiendo aparecer ciertos sesgos por falta de la información necesaria extraíble de los datos. Algunos autores como Marcus y Davis (2019), expresan, no sin cierta razón, algunos aspectos relativos a las ventajas e inconvenientes de los procesos de AP, achacándoles que muchas arquitecturas basadas en redes neuronales hacen cosas increíbles, sin ser conscientes por parte de quien las aplica del conocimiento real sobre lo que están haciendo, y por ello, no son sistemas totalmente inteligentes.

Aunque en parte, esto último puede verse desde esta perspectiva, no es menos cierto que los desarrollos basados en AP son capaces de conseguir resultados importantes, siendo esta la perspectiva desde la que se abordan y plantean los temas del presente libro.

Como sostienen Goodfellow y col. (2016), en los comienzos de la IA, se abordaron rápidamente problemas intelectualmente difíciles para los seres humanos, pero relativamente sencillos para las computadoras, todo ello mediante una lista de reglas matemáticas formales. A partir de ahí, el verdadero desafío para la IA se tornó en resolver tareas fáciles de realizar para las personas, pero difíciles de describir formalmente. Aquí se incluyen tareas tales como el reconocimiento de objetos en imágenes, palabras o acciones en los movimientos. No cabe duda de que en este aspecto el AP ha conseguido ya logros muy relevantes. Es en este rasgo, y más concretamente en la exposición de una serie de técnicas orientadas a tal fin, donde se centra el presente libro.

En definitiva, se trata de exponer una serie de técnicas para resolver problemas, por decirlo de alguna manera, intuitivos para el ser humano, con el uso de las computadoras, que aprenden mediante los métodos y algoritmos diseñados a partir de los datos suministrados y sin necesidad de que los humanos especifiquen formalmente todo el conocimiento requerido por la computadora. En cualquier caso, y siguiendo también la teoría expuesta en Goodfellow y col. (2016), la jerarquía de conceptos permite que la computadora aprenda conceptos complicados al construirlos a partir de otros más simples, todos ellos estructurados en múltiples capas, razón por la cual a este enfoque se le denomina con el término ya indicado de AP. En cualquier caso, como una técnica específica del AA, estos procedimientos se encaminan a extraer patrones determinados a partir de los datos.

Existe una diferencia fundamental en lo que respecta a la extracción de las mencionadas características entre las técnicas clásicas, por llamarlas de alguna manera, de AA y las específicas del AP. Por ejemplo, considérese un ejemplo sencillo biclase, en el que se trata de separar el cielo y la hierba en una imagen de color de un paisaje de campo. Los datos disponibles en este caso son valores de color de los píxeles, de forma que en el caso del cielo predominan las tonalidades azules, mientras que en la hierba son las verdes. Un método simple tal como naive Bayes puede separar los patrones en dos clases diferentes teniendo en cuenta que los mismos están definidos, por lo que se conoce como características. La extracción de características en este caso es esencial. Por otro lado, y siguiendo en el ámbito de las imágenes, estas se caracterizan por poseer información espacial, y en vídeos, también temporal. Las redes neuronales profundas pueden captar perfectamente ambos tipos de información. En el primer caso, los filtros de convolución son responsables de la captura espacial, a diferencia de lo que ocurre con otros modelos de red, tal como las de retropropagación, en las que las características de las imágenes se transforman en vectores que se suministran a la entrada, perdiendo las relaciones espaciales. Por ejemplo, una imagen de dimensión 3x3 se transforma en un vector con 9 componentes. En el caso de las características temporales las redes recurrentes tienen la habilidad de realizar tal captura.

No obstante, para muchas tareas no resulta fácil extraer características. Por ejemplo, supóngase que a partir de una imagen se quieren identificar peatones cuando un vehículo autónomo navega en un entorno urbano. Una persona puede identificarse por poseer cabeza, tronco y extremidades. Se podría pensar en detectar la presencia de extremidades o del cuerpo y la cabeza o todas, lo cual no resulta trivial debido a que no es fácil establecer las características de dichas partes. Los brazos y las piernas son alargados, el tronco tiene una forma más rectangular, pero en todos los casos nunca están exentos de elementos perturbadores, tales como el uso de distintos tipos de ropa, sombras, oclusiones totales o parciales, entre otros.

Una solución a este problema consiste en utilizar el AA para descubrir no solo la proyección de la representación a la salida, sino también la representación misma. Este enfoque se conoce como aprendizaje de representación según Goodfellow y col. (2016). Las representaciones aprendidas a menudo resultan en un rendimiento mucho mejor que el que se puede obtener con representaciones diseñadas a mano. También permiten que los sistemas de IA se adapten rápidamente a nuevas tareas, con una mínima intervención humana. Un algoritmo de aprendizaje de representación puede descubrir un buen conjunto de características para una tarea simple en minutos o para una tarea compleja en horas y meses. El diseño manual de características para una tarea compleja requiere una gran cantidad de tiempo y esfuerzo humanos, pudiendo llevar incluso décadas. Un ejemplo por excelencia de un algoritmo de aprendizaje de representación es el \textit{autoencoder} (autocodificador), que convierte los datos de entrada en una representación diferente para luego poder devolverla a la representación original mediante el correspondiente \textit{decoder} (decodificador).

Cuando se diseñan algoritmos para aprender las características, el objetivo consiste en separar los factores de variación que expliquen los datos observados. El concepto factor se refiere a abstracciones que ayudan a distinguir entre la alta variabilidad de los datos observados; así, en el reconocimiento de los peatones los factores de variación hacen referencia a la posición de las extremidades con respecto al tronco, la posición con respecto a la cámara, la ropa con la que van vestidos, las oclusiones de las extremidades, las posibles sombras proyectadas sobre sus cuerpos o la intensidad de la luz con la que se ha obtenido la imagen, entre otros. La mayoría de las aplicaciones exigen separar los factores de variación descartando aquellos que no interesan. A la vista de lo cual, resulta francamente difícil obtener una representación tal que permita resolver el problema. Es precisamente aquí donde entra en acción el aprendizaje profundo, ya que permite introducir representaciones que se expresan en términos de otras representaciones a distintos niveles, que van estructurando convenientemente la información. Por ejemplo, la figura \ref{fig:modelo_aprendizaje_profundo} muestra un sistema basado en AP, concretamente una Red Neuronal Convolucional (RNC) o en terminología inglesa \textit{Convolutional Neural Network} (CNN), donde se representa el concepto de una imagen de una taza combinando conceptos más simples, tales como bordes, contornos o partes de los objetos hasta llegar a su clasificación, en este caso como taza. La idea de representación en múltiples capas es lo que determina una de las perspectivas del AP. De esta forma, se puede decir con carácter general, que las primeras capas de las redes profundas extraen características de bajo nivel, de modo que estas se van tornando en más complejas con características de mayor nivel hasta llegar a las capas superiores, en las que las características extraídas de la imagen son del más alto nivel. Esta información concatenada permite identificar un objeto (taza), a pesar de que pueda presentar diferentes características tales como, por ejemplo, color, forma o tamaño, lo que permite claramente diferenciar el AP del AA.

\begin{figure}[h]
    \centering
    \includegraphics[width=0.5\textwidth]{ruta_a_la_imagen}
    \caption{Modelo de aprendizaje profundo}
    \label{fig:modelo_aprendizaje_profundo}
\end{figure}

Otra idea para determinar el concepto de profundidad es la también establecida por Goodfellow y col. (2016), en el sentido de que la profundidad se determina como el estado del computador para aprender un programa computacional multipaso, de forma que cada capa de la representación puede verse como el estado de la memoria del computador después de ejecutar otro conjunto de instrucciones en paralelo. Las redes con mayor profundidad pueden ejecutar más instrucciones en secuencia. Las instrucciones secuenciales ofrecen un gran poder porque las instrucciones posteriores pueden referirse a los resultados de instrucciones anteriores. Según esta visión del AP, no toda la información en las activaciones de una capa codifica necesariamente factores de variación que explican la entrada. La representación también almacena información de estado que ayuda a ejecutar un programa que puede dar sentido a la entrada. Esta información de estado podría ser análoga a un contador o puntero en un programa de computación tradicional. No tiene nada que ver con el contenido de la entrada específicamente, pero ayuda al modelo a organizar su procesamiento.

Existen dos formas principales de medir la profundidad de un modelo. La primera se basa en el número de instrucciones secuenciales que deben ejecutarse para evaluar la arquitectura. Se puede pensar en esto como la longitud de la ruta más larga a través del diagrama de flujo que describe cómo calcular cada una de las salidas del modelo dadas sus entradas. Otro enfoque, utilizado por modelos probabilísticos profundos, considera que la profundidad de un modelo no es la profundidad del gráfico computacional sino la profundidad del gráfico que describe cómo se relacionan los conceptos entre sí. En este caso, la profundidad del diagrama de flujo de los cálculos necesarios para computar la representación de cada concepto puede ser mucho más profunda que la gráfica de los conceptos en sí mismos. Esto se debe a que la comprensión del sistema de los conceptos más simples puede refinarse dando información sobre los conceptos más complejos. Por ejemplo, siguiendo también a Goodfellow y col. (2016), un sistema inteligente que observa una imagen de una cara con un ojo en la sombra puede ver inicialmente únicamente un ojo. Después de detectar la presencia de una cara, el sistema puede inferir que probablemente también esté presente un segundo ojo. En este caso, la gráfica de conceptos incluye sólo dos capas, una capa para ojos y una capa para caras, pero la gráfica de cálculos incluye dos capas si se refina la estimación de cada concepto dadas las otras n veces. Debido a que no siempre está claro cuál de estos dos modelos (la profundidad del gráfico computacional o la profundidad del gráfico de modelado probabilístico) es más relevante, y debido a que distintas personas eligen diferentes conjuntos de elementos más pequeños a partir de los cuales construir sus gráficos, no existe un único valor correcto para la profundidad de una arquitectura, así como tampoco hay un valor correcto único para la duración de un programa computacional. Tampoco existe un consenso acerca de la profundidad que un modelo requiere para calificarlo como “profundo”. Sin embargo, el aprendizaje profundo puede considerarse, con seguridad, como el estudio de modelos que implican una mayor cantidad de composición de funciones aprendidas o conceptos aprendidos que el aprendizaje automático tradicional.

En definitiva, el aprendizaje profundo es un tipo particular de aprendizaje automático que consigue un gran poder y flexibilidad al representar al mundo como una jerarquía anidada de conceptos, donde cada concepto se define en relación con conceptos más simples y representaciones más abstractas calculadas en términos de conceptos menos abstractos.


\section{Funciones de Unidades Lineales de Activación No Lineales}

Una función de activación clásica es la función sigmoide sigmoidal definida como sigue:

\begin{equation}
f(a, x, c) = \frac{1}{1 + e^{-a(x-c)}}
\end{equation}

Dependiendo del signo del parámetro \(a\), la función sigmoide se abre hacia la izquierda o hacia la derecha, siendo apropiada para representar conceptos tales como "muy grande" o "muy negativo". La figura 2-4 muestra la representación de sendas funciones sigmoide: en (a) con los siguientes parámetros \(a = 2\) y \(c = 4\); y en (b) con \(a = -2\) y \(c = 4\).

\begin{figure}[h]
    \centering
    \includegraphics[width=0.45\textwidth]{sigmoid_a_2_c_4.png}
    \includegraphics[width=0.45\textwidth]{sigmoid_a_-2_c_4.png}
    \caption{Funciones sigmoide: (a) con \(a=2\) y \(c = 4\); (b) con \(a = -2\) y \(c = 4\).}
\end{figure}

La función sigmoide que proyecta salidas de números reales de entrada al intervalo [0, 1] posee dos problemas:
\begin{enumerate}
    \item Saturación del gradiente. Cuando el valor de la función de activación se aproxima a los extremos 0 o 1, el gradiente de la función tiende a 0, lo que repercute en el ajuste de los pesos de las redes.
    \item Pesos positivos de forma continua. El valor medio de la función de salida no es 0, lo que origina que los pesos tiendan a ser positivos.
\end{enumerate}

Estas dos cuestiones provocan una convergencia lenta de los parámetros afectando a la eficiencia del entrenamiento.

\section{Capítulo 2: Computación Numérica}

En Courbariaux y col. (2015) se define lo que denominan función sigmoide dura (hard-sigmoid) como sigue:

\begin{equation}
f(x) = \max(0, \min(1, 0.5(x+1)))
\end{equation}

La función tangente hiperbólica tanh proporciona salidas reales en el rango definido, siendo una variante de la función sigmoide, definida exactamente como

\begin{equation}
\tanh(x) = 2 \cdot \text{sigmoid}(2x) - 1
\end{equation}

presentando el mismo problema de la saturación del gradiente. La figura 2-5 muestra la representación de la función tanh.

\begin{figure}[h]
    \centering
    \includegraphics[width=0.45\textwidth]{tanh.png}
    \caption{Función tanh}
\end{figure}

La función Unidad Lineal Rectificada (ReLU, Rectified Linear Unit),

\begin{equation}
f(x) = \max(0, x)
\end{equation}

representada en la figura 2-6(a) tiene las siguientes características:
\begin{enumerate}
    \item Gradiente no saturado. Por el hecho de que \(x > 0\), el problema de la dispersión del gradiente en el proceso de propagación inversa se ve aliviado, y los parámetros en la primera capa de la red neuronal pueden actualizarse rápidamente. En \(x = 0\) no es derivable, por lo que es habitual asignar un valor arbitrario en este caso, por ejemplo 0, 0.5 o bien 1.0.
    \item Baja complejidad computacional. Dada su propia definición.
\end{enumerate}

No obstante, posee la desventaja de que la neurona ReLU puede morir cuando recibe un gradiente negativo alto durante la retropropagación que le permite aprender más porque su derivada es cero cuando su entrada es menor que cero, por lo que el gradiente será finalmente cero. Esto se puede evitar al inicializar cuidadosamente los pesos o utilizar ReLU con "fugas", similar a ReLU, pero donde su salida es lineal multiplicada por un valor pequeño (aproximadamente 0.001) cuando la entrada es negativa, esto es

\begin{equation}
f(x) = \max(0.01x, x)
\end{equation}

tal y como se muestra en la figura 2-6(b), conocida en ocasiones como ReLU con fugas (LReLU, Leaky ReLU).

\begin{figure}[h]
    \centering
    \includegraphics[width=0.45\textwidth]{relu.png}
    \includegraphics[width=0.45\textwidth]{lrelu.png}
    \caption{Funciones: (a) ReLU; (b) LReLU}
\end{figure}

En algunos tipos de redes como Mobile Net, que se estudiarán posteriormente, se define una variante de ReLU como sigue (es la función ReLU6), y cuya representación se muestra en la figura 2-7(a). A partir de ella se define hard-swish o h-swish (Hs) representada en la figura 2-7(b), aunque a veces en esta función se utiliza:

\begin{equation}
\text{ReLU6}(x) = \min(\max(x,0),6)
\end{equation}

\begin{equation}
\text{HS}(x) = x \cdot \frac{\text{ReLU6}(x + 3)}{6}
\end{equation}

\begin{figure}[h]
    \centering
    \includegraphics[width=0.45\textwidth]{relu6.png}
    \includegraphics[width=0.45\textwidth]{hswish.png}
    \caption{Funciones: (a) ReLU6; (b) Hard-Swish}
\end{figure}

La función Paramétrica ReLU (PReLU, Parametric ReLU, He y col., 2015b) se define según:

\begin{equation}
f(x, \alpha) = \begin{cases} 
x & \text{si } x > 0 \\ 
\alpha x & \text{si } x \leq 0 
\end{cases}
\end{equation}

De forma que si el parámetro \(\alpha = 0\) la función es exactamente ReLU; si \(\alpha > 0\) se trata de la función LReLU, es cuando el parámetro \(\alpha\) se incluye como un parámetro a aprender durante el proceso de entrenamiento, cuando la función toma su verdadero significado, de ahí su nombre.

Por otra parte, existe la Unidad lineal exponencial (ELU, Exponential Linear Unit) definida en Clevert y col. (2016) como sigue con \(\alpha > 0\):

\begin{equation}
f(x, \alpha) = \begin{cases} 
x & \text{si } x > 0 \\ 
\alpha (e^x - 1) & \text{si } x \leq 0 
\end{cases}
\end{equation}

El parámetro \(\alpha\) controla el valor para el cual se produce la saturación para valores de \(x\) negativos. En la figura 2-8 se muestran sendas funciones ELU con valores de \(\alpha = 0.1\) en (a) y 1.0 en (b), respectivamente.

\begin{figure}[h]
    \centering
    \includegraphics[width=0.45\textwidth]{elu_0.1.png}
    \includegraphics[width=0.45\textwidth]{elu_1.0.png}
    \caption{Funciones ELU: (a) con \(\alpha= 0.1\); (b) con \(\alpha= 1.0\)}
\end{figure}

La función SELU hace referencia a Unidad lineal exponencial escalada (Scaled Exponential Linear Unit), siendo una versión ligeramente modificada de ELU por Klambauer y col. (2017) y definida como sigue:

\begin{equation}
f(x, \alpha) = \begin{cases} 
\lambda x & \text{si } x > 0 \\ 
\lambda \alpha (e^x - 1) & \text{si } x \leq 0 
\end{cases}
\end{equation}

En la figura 2-9 se muestra la representación gráfica de la función SELU con valores \(\lambda = 1.0\) y \(\alpha = 2.0\).

\begin{figure}[h]
    \centering
    \includegraphics[width=0.45\textwidth]{selu.png}
    \caption{Función SELU}
\end{figure}

Las funciones ReLU tienen la ventaja de acelerar el entrenamiento, ya que el cálculo del gradiente es simple (0 o 1 dependiendo del signo de la entrada), y no existe una constante en la parte positiva del dominio.























\section{Introducción}
En este capítulo se inicia la presentación, junto con las características más relevantes de las redes neuronales profundas, que, desde sus primeros desarrollos a mediados del siglo XX, han recibido diferentes nombres entre los que destacan: redes neuronales, computadores neuronales, sistemas distribuidos paralelos, modelos conexionistas, entre otros.

Se inicia el capítulo con los fundamentos de este tipo de redes, para abordar a continuación el modelo del perceptrón, como unidad básica, y posteriormente la red de retropropagación. Se finaliza con una pincelada de los modelos conocidos como redes de creencia o bayesianas. En todos los casos con la vista puesta en el concepto de profundidad dentro del aprendizaje profundo.

\section{Fundamentos Generales}
El interés por las redes neuronales data de los años 40, a partir del trabajo de McCulloch y Pitts (1943), donde se proponen modelos de neuronas en la forma de dispositivos binarios basados en un umbral y algoritmos estocásticos que implicaban cambios binarios 0-1 y 1-0 en los estados de las neuronas como la base para el desarrollo de sistemas neuronales. El trabajo posterior de Hebb (1949) estaba basado en modelos matemáticos que intentaban capturar el concepto de aprendizaje por refuerzo o asociación. Durante la primera mitad de los años 50 y principios de los 60, las denominadas máquinas de aprendizaje propuestas por Rosenblatt (1962) supusieron una revolución entre los investigadores en la teoría de reconocimiento de patrones. La razón del gran interés de dichas máquinas llamadas perceptrones fue el desarrollo de las correspondientes demostraciones matemáticas llegando a la conclusión de que los perceptrones, cuando son entrenados con conjuntos de entrenamientos linealmente separables, convergen a una solución en un número finito de iteraciones. La solución tomó la forma de coeficientes de hiperplanos capaces de separar correctamente las clases representadas por patrones del conjunto de entrenamiento.

Desafortunadamente, las expectativas que siguieron a este descubrimiento perdieron fuerza porque el modelo anterior era inapropiado para muchas tareas de reconocimiento de patrones. Intentos posteriores para extender la potencia del perceptrón considerando múltiples capas de perceptrones fracasaron también. Un estado del arte sobre las máquinas de aprendizaje a mitad de los años 60 fue recopilado por Nilsson (1965). Unos pocos años más tarde, Minsky y Papert (1969) presentaron un estudio desalentador sobre las limitaciones de las máquinas de aprendizaje basadas en el perceptrón. Esta idea negativa se mantuvo hasta mitad de la década de los 80, llegándose incluso a rechazar el uso del perceptrón en algunos trabajos como el presentado por Simon (1986).

Los nuevos algoritmos de aprendizaje para perceptrones multicapa presentados por Rumelhart et al. (1986), conocidos como regla delta generalizada para aprendizaje por retropropagación, modificaron el interés por las redes neuronales. En los años sucesivos, nuevos algoritmos fueron presentados por diferentes autores. Aunque no se ha probado la convergencia de dichos algoritmos hacia una solución óptima, han sido aplicados con éxito a muchos problemas de interés práctico, lo que les otorgó una cierta validez en los años 90. A continuación, estas redes cayeron en popularidad en favor de otras técnicas diferentes. Los motivos fueron múltiples. Por ejemplo, no había un lenguaje estándar de facto para modelar redes neuronales. Tampoco se observaban mejoras significativas cuando se añadían muchas capas a la red, resultando a veces contraproducente. Además, el entrenamiento era computacionalmente costoso con los ordenadores disponibles en la época de referencia. Y en muchas condiciones no había suficientes datos para evitar el sobreajuste del modelo.

Para entender correctamente los fundamentos de las redes neuronales profundas conviene empezar estudiando el modelo más básico. Este consiste en especificar un conjunto de funciones indicador \(A(x,w)\) que toman los valores \{0,1\}, considerando que son los parámetros que definen cada posible elemento que hay que clasificar. Además, se considera que se dispone de \(n\) muestras \((x_i,y_i)\) de entrenamiento (con \(i=1,...,n\)) de las que se conoce su clase \(y_i\). A continuación se minimiza el riesgo empírico de la función indicador sobre todo elemento del conjunto de entrenamiento. Para ello se suele minimizar el error de clasificación incorrecta \(R_o(w) = \sum_i [f(x_i,w) - y_i]^2\).

Para entender mejor el proceso, considérese primero el siguiente caso especial de funciones indicador, donde \(h()\) es una función que indica si el elemento pertenece a la clase (toma el valor 1) o no (toma el valor 0) en base a la combinación lineal, sopesada por los valores de los pesos del modelo \(w\), de los \(q\) parámetros de entrada en \(x\).
\[
f(x,w) = h \left( \sum_{j=1}^q w_j x_j \right)
\]
En este caso, si se supone que el conjunto de datos de entrenamiento es linealmente separable, existe un simple procedimiento de optimización para encontrar \(f(x,w^*)\), que es conocido como el algoritmo del perceptrón. Cuando los datos son no separables este algoritmo no proporciona una solución óptima, motivo por el que se han desarrollado otros procedimientos. Uno de los métodos considerados es el entrenamiento conocido como Widrow-Hoff o regla delta de mínimos cuadrados, que minimiza una función del tipo dado en la ecuación anterior. Una alternativa diferente consiste en utilizar un perceptrón multicapa (MLP, Multi-Layer Perceptron), que es capaz de manejar adecuadamente tanto las clases separables como las no separables y que utiliza el riesgo empírico funcional dado por:
\[
R_{emp}(w,v) = \sum_{i=1}^n \left( f(x_i, w, v) - y_i \right)^2
\]
que debe ser minimizado con respecto a los parámetros o pesos \(w\) y \(v=[v_1, v_2, ..., v_m]\). En el caso del perceptrón multicapa, la función \(f(x,w,v)\) se parametriza como:
\[
f(x,w,v) = h \left( g(x, w, v) \right)
\]
donde \(g(x,w,v) = \sum_{j=1}^m v_j h \left( \sum_{i=1}^q w_{ji} x_i \right)\). 

\section{El Perceptrón}
El modelo más básico que se puede utilizar en los modelos de redes neuronales recibe el nombre de perceptrón. Debido a su expresión matemática, introducida brevemente en la sección anterior, es capaz de aprender el modelo (obtener los pesos en \(w\)) que permite distinguir los elementos de dos clases linealmente separables. En esta sección se detallan los elementos y formas de representar este modelo, y se explican diferentes métodos de entrenamiento para ajustar los valores de sus parámetros.

\subsection{Arquitectura del Perceptrón}
En su forma más básica, el perceptrón aprende una función discriminante lineal \(f_d(x)\) que se corresponde con la función \(f(x, w^*)\) introducida en la sección anterior. Esta función establece una dicotomía entre dos conjuntos de entrenamiento linealmente separables. Dados dos conjuntos de puntos, se dice que son linealmente separables si existe un hiperplano en el espacio patrón que separa ambos conjuntos de datos. En el caso de un espacio de dos dimensiones, el hiperplano se puede representar gráficamente y a efectos pedagógicos como una recta, tal y como se muestra en la Figura 3-1.

\begin{figure}[h]
    \centering
    \includegraphics[width=0.5\textwidth]{figura_3_1.png}
    \caption{Hiperplano de separación para el caso bidimensional}
    \label{fig:figura_3_1}
\end{figure}

La Figura 3-2(a) muestra el esquema del modelo del perceptrón para dos clases. La respuesta de este dispositivo está basada en dos etapas. En la primera, se calcula la suma promediada de sus entradas mediante una función de decisión lineal con respecto a las componentes de los vectores patrón:
\[
f_d(x) = \sum_{i=1}^n w_i x_i + w_{n+1}
\]
Los coeficientes \(w_i\), \(i= 1, 2, ..., n, n+1\), llamados pesos, modifican las entradas antes de que sean sumadas y suministradas al elemento de umbral. Una de las entradas es el sesgo (bias) externo \(w_{n+1}\). En este sentido, los pesos son similares a las sinapsis en el sistema neuronal humano.

En la segunda, se transforma la salida de la primera etapa mediante una función de activación (también conocida como función de transferencia). En su forma más simple, la función de activación es una función escalón (Figura 3-2(b)). Sin embargo, en otras arquitecturas de red también puede ser una función sigmoide o sigmoide binaria.

\begin{figure}[h]
    \centering
    \begin{subfigure}[b]{0.45\textwidth}
        \includegraphics[width=\textwidth]{figura_3_2a.png}
        \caption{}
        \label{fig:figura_3_2a}
    \end{subfigure}
    \begin{subfigure}[b]{0.45\textwidth}
        \includegraphics[width=\textwidth]{figura_3_2b.png}
        \caption{}
        \label{fig:figura_3_2b}
    \end{subfigure}
    \caption{Esquema del modelo del perceptrón para dos clases}
    \label{fig:figura_3_2}
\end{figure}

A continuación, se presenta un conjunto de ejemplos de patrones de entrada con el objetivo de encontrar el hiperplano que permite separar dos clases de forma adecuada. El conjunto de patrones de entrada es denotado como \(x_1, x_2, ..., x_n\), y cada patrón de entrada tiene una dimensión de \(m\). La salida deseada es denotada como \(d_1, d_2, ..., d_n\), donde cada \(d_i\) pertenece a \{0,1\}.


\end{comment}



\begin{comment}

\textbf{MXNet} \citep{mxnet} es un marco de trabajo de aprendizaje profundo de código abierto fundado como una colaboración entre la Universidad Carnegie Mellon, la Universidad de Washington y Microsoft. Es una librería escalable que permite el entrenamiento de redes neuronales profundas utilizando diferentes lenguajes de programación, incluyendo C++, Python, MATLAB, JavaScript, R, Julia y Scala. MXNet incluye la interfaz de Gluon que permite a los desarrolladores con cualquier nivel de experiencia comenzar a usar el aprendizaje profundo en la nube, en dispositivos de borde \footnote{dispositivos más cercanos al usuario, como teléfonos móviles, sistemas ciberfísicos (CPS), dispositivos portátiles, IoT, \ldots.} y en aplicaciones para dispositivos móviles. MXNet admite además el paralelismo de datos en múltiples CPUs o GPUs, así como el paralelismo de modelos. Ofrece dos modos de entrenamiento diferentes: síncrono y asíncrono \footnote{síncrono: interactúan en el mismo momento en el que tiene lugar la comunicación; asíncrono: la interacción no es inmediata y puede tener lugar en momentos diferentes} y además proporciona operaciones primitivas de tolerancia a fallas a través de guardar y cargar: guardar almacena los parámetros del modelo en un archivo de punto de control y cargar restaura los parámetros del modelo desde un archivo de punto de control. MXNet admite tanto la programación declarativa como la programación imperativa.
\textbf{Theano} \citep{theano} es una biblioteca de Python de código abierto para cálculos a gran escala que ha sido desarrollado por investigadores y desarrolladores de la Universidad de Montreal. Es una biblioteca que facilita la construcción de modelos de aprendizaje profundo y se puede ejecutar en diferentes plataformas informáticas, incluyendo CPU y GPU. Los cálculos se expresan utilizando una sintaxis similar a la de Numpy y funciona creando una representación simbólica de las operaciones que se traducen a C++ y luego se compilan en moléculas Python. Theano admite tanto el paralelismo de datos como el paralelismo de modelos.
\textbf{Chainer} \citep{chainer} es un marco de aprendizaje profundo de código abierto implementado en Python. El desarrollo de Chainer está liderado por investigadores y desarrolladores de la Universidad de Tokio. Chainer proporciona APIs de diferenciación automática para construir y entrenar redes neuronales, con un enfoque de ``definir por ejecutar'', lo que permite construir el grafo computacional durante el entrenamiento y permite al usuario cambiarlo en cada iteración. Chainer es un marco flexible ya que proporciona una API imperativa en Python y NumPy. Se admiten tanto cálculos en CPU como en GPU.
\end{comment}











\chapter{Aplicación en la ciberseguridad} \label{Capitulo_3}

\section{Clasificación de Malware}

Hoy en día, uno de los principales retos que enfrenta el software anti-malware es la enorme cantidad de datos y archivos que se requieren evaluar en busca de posibles amenazas maliciosas. Una de las razones principales de este volumen tan elevado de archivos diferentes es que los creadores de malware introducen variaciones en los componentes maliciosos para evadir la detección. Esto implica que los archivos maliciosos pertenecientes a la misma ``familia'' de malware (con patrones de comportamiento similares), se modifican constantemente utilizando diversas tácticas, lo que hace que parezcan ser múltiples archivos distintos. 


Para poder analizar y clasificar eficazmente estas cantidades masivas de archivos, es necesario agruparlos e identificar sus respectivas familias. Además, estos criterios de agrupación pueden aplicarse a nuevos archivos encontrados en computadoras para detectarlos como maliciosos y asociarlos a una familia específica. 



\subsection{Microsoft Malware Classification Challenge}



El conjunto de datos utilizado en este estudio proviene del Microsoft Malware Classification Challenge (BIG 2015) \citep{kagglebig2015}, una competición dirigida a la comunidad científica con el objetivo de promover el desarrollo de técnicas efectivas para agrupar diferentes variantes de malware. Se decidió escoger este dataset porque el objetivo que tengo en este trabajo es el de aprender y desarrollar diferentes métodos de aprendizaje automático y este dataset nos permite utilizar tanto una \acrshort{cnn} como un Autoencoder según \citep{podder2021artificial}. 

Se puede descargar desde su página web \citep{kagglebig2015}. Tiene un tamaño de 0.5 TB sin comprimir. Para poder manipularla en mi ordenador, tuve que seguir los siguientes pasos. Primero, me descargué la carpeta comprimida (7z) con todo el dataset. Después, la subí al servidor Simba de la facultad de informática y finalmente, usando el comando \textit{7zz x file\_ name.7z}, la descomprimí. 


Este dataset contiene 5 archivos:
\begin{itemize}
\item dataSample.7z - Carpeta comprida(7z) con una muestra de los datos disponibles.
\item train.7z - Carpeta comprida(7z) con los datos para el conjunto de entrenamiento.
\item trainLabels.csv - Archivo csv con las etiquetas asociadas a cada archivo de train.
\item test.7z - Carpeta comprida 7z con los datos sin procesar para el conjunto de prueba.
\item sampleSubmission.csv - Archivo csv con el formato de envío válido de las soluciones.
\end{itemize}



Para nuestro estudio, nos enfocaremos exclusivamente en el conjunto de datos de entrenamiento, que consta de los archivos ``train.7z'' y ``trainLabels.csv''. Los archivos 'test.7z' y 'sampleSubmission.csv' están destinados específicamente para la competición. Nosotros no los utilizaremos debido a que son programas de malware sin etiquetar y para este problema de clasificación, es necesario conocerlas.  Además, la carpeta 'dataSample.7z' proporciona dos programas que se encuentran también en la carpeta train.7z, por lo que tampoco la utilizaremos. 


Cada programa malicioso tiene un identificador, un valor hash de 20 caracteres que identifica de forma única el archivo, y una etiqueta de clase, que es un número entero que representa una de las 9 familias de malware al que puede pertenecer. Por ejemplo, el programa \textit{0ACDbR5M3ZhBJajygTuf} tiene como etiqueta el valor 7. Esta información se puede consultar en el archivo ``trainLabels.csv''. Cada programa tiene dos archivos, uno asm\footnote{Es una secuencia de bytes que lo podemos visualizar como si fuera un texto plano} con el código extraído por la herramienta de desensamblado IDA y otro bytes\footnote{Realmente no es un archivo bytes, sino un archivo de texto con letras} con la representación hexadecimal del contenido binario del programa pero sin los encabezados ejecutables (para garantizar esterilidad). Para nuestro estudio vamos a utilizar únicamente este ultimo archivo. 

\begin{figure}[h]
    \begin{center}
    \includegraphics[width=0.7\textwidth]{img/previewMMC.png}
    \end{center}
    \caption{Explicación del contenido de ``0ACDbR5M3ZhBJajygTuf.bytes''.}
    \label{fig:previewMMC}
\end{figure} 


Como aparece en la figura \ref{fig:previewMMC}, los ocho primeros caracteres son direcciones de memoria, seguido de la representación hexadecimal del contenido binario del programa, que contiene 16 bytes (cada uno dos caracteres). A veces nos podemos encontrar con ``??'' en el lugar de un byte. Este símbolo se utiliza en estos archivos para representar que se desconoce su información porque su memoria no se puede leer \citep{cahyani2022influence}. 


\subsubsection{Distribución del dataset}

Hay un total de 21.741 programas de malware, pero nosotros tan solo usaremos los 10.868 pertenecientes al entrenamiento. Estos programas pertenecen a una de estas 9 familias de malware: Rammit, Lollipop, Kelihos\_ ver3, Vundo, Simda, Tracur, Kelihos\_ ver1, Obfuscator y Gatak. Según \cite{hu2019machine}, podemos deinirlas como: 

  

\begin{enumerate}

\item \textbf{Ramnit} es un malware tipo gusano que infecta archivos ejecutables de Windows, archivos de Microsoft Office y archivos HTML. Cuando los ha infectado, el ordenador pasa a formar parte de una red de bots, que son controladas por un nodo central de forma remota. Este malware puede robar información sensible y puede propagarse a través de conexiones de red y unidades extraíbles.

\end{enumerate}


\begin{wrapfigure}{l}{0.33\textwidth}
  \begin{center}
    \includegraphics[width=0.33\textwidth]{img/botnetStructure.png}
  \end{center}
    \caption{Estructura de un botnet. Imagen sacada de \citep{botnetstructure}.}
  \label{fig:botnetstructure}
\end{wrapfigure}
\vspace{-\baselineskip}

\textbf{2.} \textbf{Lollipop} es un tipo de programa adware que muestra anuncios no deseados en los navegadores web. También puede redirigir los resultados de búsqueda a recursos web ilegítimos, descargar aplicaciones maliciosas y robar la información del ordenador monitoreando sus actividades web. Este adware se puede descargar desde el sitio web del programa o empaquetarse con algunos programas de terceros.


\textbf{3.} \textbf{Simda} es un troyano backdoor que infecta ordenadores descargando y ejecutando archivos arbitrarios que pueden incluir malware adicional. Los ordenadores infectados pasar a ser parte de una botnet, lo que les permite cometer aciones criminales como robo de contraseñas, credenciales bancarias o descargar otros tipos de malware.

\textbf{4.} \textbf{Vundo} es otro troyano conocido por causar publicidad emergente para programas de antivirus falsos. A menudo se distribuye como un archivo DLL(Dynamic Link Library) \footnote{Una parte del programa que se ejecuta cuando una aplicación se lo pide. Se suele guardar en un directorio del sistema.} y se instala en el ordenador como un Objeto Auxiliar del Navegador (BHO) sin su consentimiento. Además, utiliza técnicas

\begin{enumerate}
\item[] avanzadas para evitar su detección y eliminación.
\item[5.] \textbf{Kelihos\_ ver3} es un troyano tipo backdoor \footnote{Un backdoor permite que una entidad no autorizada tome el control completo del sistema de una víctima sin su consentimiento.} que distribuye correos electrónicos que pueden contener enlaces falsos a instaladores de malware. Consta de tres tipos de bots \citep{kerkers2014characterisation}: controladores (operados por los dueños y donde se crean las instrucciones), enrutadores (redistribuyen las instrucciones a otros bots) y trabajadores (ejecutan las instrucciones).

\item[6.] \textbf{Tracur} es un descargador troyano que agrega el proceso 'explorer.exe' a la lista de excepciones del Firewall de Windows para disminuir deliberadamente la seguridad del sistema y permitir la comunicación no autorizada a través del firewall. Además, esta familia también te puede redirigir a enlaces maliciosos para descargar e instalar otros tipos de malware.

\item[7.] \textbf{Kelihos\_ ver1} es una versión más antigua del troyano Kelihos\_ ver3 pero tiene el mismo funcionamiento y las mismas partes.

\item[8.] \textbf{Obfuscator.ACY} es un tipo de malware sofisticado que oculta su propósito y podría sobrepasar las capas de seguridad del software.  Se puede propagar mediante archivos adjuntos de correo electrónico, anuncios web y descargas de archivos.

\item[9.] \textbf{Gatak} es un troyano que abre una puerta trasera en el ordenador. Se propaga a través de sitios web falsos que ofrecen claves de licencias de productos. Una vez infectado el sistema, Gatak recopila información del ordenador.
\end{enumerate} 



Como ya mencionamos antes, vamos a entrenar nuestras redes neuronales con 10.868 archivos bytes. De estos archivos, solo nos son válidos 10.860 porque en los 8 restantes (pertenecientes a la familia Ramnit), todo su contenido son ``??''. Con estos programas finales, vamos a ver sus datos en el gráfico \ref{img: circularMMC}. 
 
 \begin{figure}[h]
    \begin{center}
    \includegraphics[width=0.6\textwidth]{img/circularMMC.png}
    \end{center}
    \caption{Distribución del BIG 2015 training dataset.}
    \label{img: circularMMC}
\end{figure}  

Analizando \ref{img: circularMMC}, podemos observar como la distribución entre las clases no es uniforme. Mientras que de la clase Simbda hay 42 muestras, de la clase Kelihos\_ ver3 hay 2942, es decir, un 98 \% más. En \citep{kebede2017classification} deciden prescindir de esta clase, pero nosotros hemos decidido hacer el análisis con las 9 clases. A la hora de crear nuestros modelos, hemos dividido el conjunto de datos aleatoriamente en grupos del 75\%, 15\% y 10\% para entrenamiento, test y validación respectivamente. La imagen \ref{img: train_test_valMMC} muestra como quedarían distribuidas las clases en los diferentes grupos.



\begin{figure}[h]
    \begin{center}
    \includegraphics[width=0.6\textwidth]{img/barras4MMC.png}
    \end{center}
    \caption{Distribución de las clases en cada grupo.}
    \label{img: train_test_valMMC}
\end{figure}  



Como podemos observar, todos los programas proporcionados son malware, lo que nos indica que con este dataset no podemos crear un modelo que nos prediga si un programa es malware o no, sino que tan solo podemos hacer un modelo de clasificación. Para ello vamos a abordar este experimento de dos formas diferentes. Una de ellas es utilizando una CNN y la otra es usando un autoencoder. utilizar dos redes neuronales





\begin{comment}
\section{Resumen del Conjunto de Datos}

El conjunto de datos utilizado en este estudio proviene del Desafío de Clasificación de Malware de Microsoft (BIG 2015), accesible a través de Kaggle. Este conjunto de datos consta de programas de malware pertenecientes a 9 categorías diferentes. Se proporcionan conjuntos de datos de entrenamiento y prueba, pero solo se utilizan los datos de entrenamiento debido a la falta de etiquetas para el conjunto de prueba para su entrega. Además

El conjunto de entrenamiento contiene 10,868 muestras de malware, cada una con un identificador único, un valor hash de 20 caracteres y una etiqueta de clase que representa una de las 9 familias de malware. Cada archivo de malware tiene una representación hexadecimal de su contenido binario, junto con un archivo 'bytes' que contiene esta información. No se excluye ninguna categoría de malware en este estudio.

La distribución de los datos se realiza con un 72\% para entrenamiento, un 8\% para validación y un 20\% para pruebas. Se realizan varias divisiones y análisis de los conjuntos de datos para ajustar y evaluar diferentes modelos y enfoques de aprendizaje profundo para la clasificación de malware.

Este conjunto de datos, con su amplia gama de muestras etiquetadas y la diversidad de familias de malware representadas, proporciona una base sólida para la investigación y desarrollo de métodos efectivos de detección y clasificación de amenazas informáticas.

\end{comment}


















\begin{comment}

\item Worm (Gusano)- Los gusanos informáticos son un tipo de malware que puede propagarse a través de redes sin la intervención de un programa huésped o de un humano. Son caballos de Troya maliciosos que pueden replicarse y propagarse de una computadora a otra. Los gusanos infectan a sus anfitriones mediante el engaño y la astucia, pudiendo causar graves daños a las computadoras comprometidas al consumir ancho de banda, sobrecargar sistemas, eliminar o modificar archivos e instalar virus adicionales. Además, las vulnerabilidades en el software, los archivos adjuntos de correo electrónico y las conexiones de red pueden facilitar su propagación. 

\item Adware (programa publicitario) - El adware es una variedad de malware que muestra anuncios no deseados a los usuarios, típicamente como ventanas emergentes o banners. A menudo se distribuye como parte de descargas de software gratuitas, pero también se puede obtener a través de un ciberataque o una vulnerabilidad. El adware tiene como objetivo generar ingresos para sus desarrolladores mostrando anuncios a los consumidores. Sin embargo, puede causar un daño significativo a una computadora infectada al degradar su rendimiento, reducir su disponibilidad y violar la privacidad del usuario al recopilar información sobre sus actividades de navegación.

\item Backdoor (Puerta trasera) - Un backdoor permite que una entidad no autorizada tome el control completo del sistema de una víctima sin su consentimiento. Un troyano de puerta trasera siempre se presenta como una herramienta de software legítima esencialmente requerida por el usuario. Otras opciones pueden ser visitar un sitio web malicioso o hacer clic en un enlace no deseado. Al ejecutarse, se añade a sí mismo en la rutina de inicio del sistema y busca una conexión a Internet. Una vez que el sistema está en línea, se conecta con su autor, quien luego toma el control del sistema para realizar diferentes tareas, como descargar/cargar archivos, registrar pulsaciones de teclas, enviar correos electrónicos no deseados o robar contraseñas, entre otras cosas.

\item Trojan (Troyano)- Los troyanos son programas maliciosos que se disfrazan como programas o archivos legítimos y pueden tomar el control de una computadora para ejecutar operaciones maliciosas como robar datos, dañar el sistema o abrir puertas traseras para otros virus. El término "Troyano" proviene de la leyenda griega del caballo de Troya, una táctica engañosa utilizada para conquistar Troya. En el ámbito digital, los troyanos son parásitos digitales hostiles capaces de leer contraseñas, grabar pulsaciones de teclas y propagar otros virus. Se utilizan técnicas de ingeniería social, como correos electrónicos de phishing o descargas maliciosas, para propagarlos. A diferencia de los virus informáticos y los gusanos, los troyanos no pueden replicarse y deben ser instalados o ejecutados por los usuarios.

\item Trojan downloader (Descargador troyano) - Es un programa malicioso que se descarga e instala en una computadora infectada. Puede abrir conexiones de red ilícitas, mutarse a sí mismo, deshabilitar herramientas de seguridad y transferir información personal del usuario sin permiso. Además, su función principal es la de descargar e instalar otro malware dañino en el sistema infectado.

\item Obfuscated malaware (Malware obfuscado) - La obfuscación es una técnica utilizada para hacer que el código de un programa sea más difícil de entender o de analizar. Esto implica modificar el código del programa malicioso de manera que sea más complicado para los investigadores de seguridad o los programas antivirus detectarlo y comprender cómo funciona. Su objetivo principal es eludir la detección y análisis por parte de los programas antivirus y otros sistemas de seguridad.













\item Ramnit es un malware de tipo gusano detectado en el 2011 y que afecta sistemas operativos Windows. , pueden ser utilizadas para múltiples objetivos maliciosos. Ramnit es una avanzada herramienta para criminales con funcionalidades de rootkit, no detección por antivirus, inyección Web y uso de comunicaciones cifradas con el centro de Comando y Control. Ramnit es un malware que ha sido utilizado para realizar actividades criminales, entre las que se pueden destacar. Monitorización de la navegación web del sistema infectado y detectar la visita de sitios de banca online. Manipulación de webs de banca online con el objetivo de parecer legitimar. Robo de cookies de sesión de los navegadores web para poder suplantar a la víctima en sitios seguros. Escaneo los discos duros de la computadora y robo archivos en base a palabras clave (como contraseñas). Acceso de forma remota a los ordenadores afectados. Recopilación de las credenciales de acceso de clientes de FTP.
¿Cómo me infecta? Se cree que su propagación es través de enlaces de confianza enviados a través de correos electrónicos de phishing o mediante redes sociales, que están principalmente orientados principalmente al robo de dinero de cuentas bancarias de victimas con sistemas operativos Windows. También se ha detectado el uso servidores FTP públicos para la distribución del malware.


\item Lollipop (L) is an adware that produces profits for the developer by automatically showing advertisements on the user
computer.
\item 1 Swizzor Swizzor is a type of malware flies that under the radar to deliver unsolicited advertisements, modifying browser setting without
user permission.
2 Vundo Vundo is either a Trojan horse or a computer worm that is known to cause Pop-up advertising for rogue antispyware programs.
3 Spybot Spybot is a type of worm that usually arrives on a computer through peer-to-peer file sharing, specifically through the Kazaa file
sharing network. Its many variants sometimes have other ways of spreading.
4 Ransom Ransom is a type of malware that can be covertly installed on a computer without knowledge or intention of the user that
restricts access to the infected computer system in some way, and demands that the user pay a ransom to the malware
operators to remove the restriction.
5 Ramnit Ramnit is a type of virus that infects Windows executable files and HTML files. It can also give a malicious hacker access to
your computer. It spreads through infected removable drives, such as USB flash drives.
6 Lollipop Lollipop is a type of adware program that shows advertisements as you browse the web. It can also redirect your search engine
results, monitor what you do on your computer, download applications, and send information about your computer to a hacker.
7 Kelihos Kelihos is a type of trojan that can give a malicious hacker access and control of your computer. The family spreads by sending
spam emails that have links to other malware.
8 Delf Delf is a type of trojan that reports and intercepts Internet traffic and may also download unwanted applications onto your
computer.
9 Banker Banker is a type of data-stealing trojans that can capture your online banking details, such as your log 

\citep{tang2019dynamic}

\item Simda es un malware de tipo troyano que infecta ordenadores con sistema operativo Windows. Los ordenadores infectados con este malware pasan a ser parte de una botnet, con lo que pueden ser utilizados para cometer acciones criminales o maliciosas.

¿Qué hace?	
Esta familia de malware incluye funcionalidades de backdoor, robo de contraseñas, robo de credenciales bancarias y es capaz de descargar otro tipo de malware.

Sistemas afectados	
Principalmente sistemas Windows:

Windows XP
Windows Vista
Windows 7
Windows 8
¿Cómo me infecta?	
El contagio de esta amenaza es mediante la distribución de archivos previamente infectados o a través de redes compartidas.

\item Obfuscation malware, also known as obfuscation techniques or code obfuscation, is a strategy employed by cybercriminals to hide the true intent and functionality of their malicious code. Essentially, it’s a way of making malware more difficult to detect and analyze by security software and researchers.

Compression, encryption, and encoding are some of the most common obfuscation methods used by threat actors. Multiple methods are often used in tandem to evade a wider variety of cyber security tools at the initial point of intrusion.

\item Gatak is a backdoor trojan that first appeared in 2012. Another name for this threat is Stegoloader, and its main distinctive feature is its ability to communicate with its C&C servers via steganography.

El grupo detrás de Trojan Gatak (Trojan.Gatak) continúa siendo una amenaza para las compañías, especialmente para el sector salud, fuertemente impactado por los ataques. Gatak es conocido por infectar a sus víctimas por medio de sitios web que prometen claves de licencias de producto para software pirata. Si bien el grupo se enfocó inicialmente en blancos en los Estados Unidos, a lo largo de los últimos dos años se ha diversificado y los ataques se llevan a cabo ahora contra compañías en varios países.

La mayor parte de las infecciones de Gatak (62%) ocurre en computadoras corporativas. El análisis de recientes ataques corporativos indica que el sector de la salud es, sin lugar a dudas, el más impactado por Gatak. De las 20 principales compañías más afectadas (compañías con más computadoras infectadas), 40% eran en el sector de la salud. En el pasado, el sector de seguros también quedó fuertemente en la mira del grupo.

Sitio web de generación de claves para licencias usado para atraer a las víctimas inocentesLas víctimas de Gatak son infectadas usando sitios web que ofrecen generar claves para licencias de productos o "keygens" de software pirata. El malware es empaquetado con la clave del producto y, si la víctima es inducida a bajar y abrir uno de esos archivos, el malware se instala clandestinamente en su computadora.

Los responsables del ataque parecen enfocarse en ofrecer claves de producto para softwares que son más probables que se utilicen en entornos profesionales. Los sitios web usados ​​en los ataques son controlados por los grupos de ataque y no tienen relación con los desarrolladores del software. En ningún momento se comprometen las versiones legítimas del software. Entre las marcas de softwares usadas ​​como anzuelos se han identificado:
SketchList3D (software de diseño para carpintería)
Native Instruments Drumlab (software de ingeniería de audio)
BobCAD-CAM (software de manufactura/metalúrgica)
BarTender Enterprise Automation (software de creación de etiquetas y código de barras)
HDClone (utilitario de clonaje de disco rígido)
Siemans SIMATIC STEP 7 (software de automación industrial)
CadSoft Eagle Professional (software de diseño de placa de circuito impreso)
PremiumSoft Navicat Premium (software de administración de base de datos)
Originlab Originpro (software gráfico para análisis de datos)
Manctl Skanect (software de digitalización 3D)
Symantec System Recovery (software de backup y recuperación de datos; ahora parte de Veritas)
Las claves de producto bajadas a partir de estos sitios no funcionan y simplemente generan una secuencia pseudoaleatoria de caracteres. Eso significa que todo lo que la víctima recibe con la descarga es un archivo inútil y una posible infección de Gatak.
Herramientas de Malware
Trojan Gatak (también conocido por Stegoloader) ha sido utilizado en ataques por lo menos desde 2011. Existen dos componentes principales de malware. Un módulo de implementación leve (Trojan.Gatak.B) puede realizar una identificación detallada del sistema en las computadoras infectadas e instalar selectivamente cargas adicionales. El módulo principal (Trojan.Gatak) es un verdadero Troyano de backdoor, que mantiene una presencia persistente en una computadora infectada y roba su información.

Una característica notable de Gatak es el uso de esteganografía, una técnica que esconde datos dentro de archivos de imagen. Cuando Gatak se instala en una computadora, intenta bajar un archivo de imagen PNG de una serie de direcciones URL codificadas en el malware. La imagen se parece a una fotografía común, sin embargo, contiene un mensaje cifrado dentro de sus datos de pixel. Trojan Gatak es capaz de decodificar este mensaje, que contiene comandos y archivos para su ejecución.
En casi el 62% de los incidentes, el movimiento lateral en la red de la víctima se produce dentro de las dos horas tras la infección. En los casos restantes, el movimiento lateral se inició en algún momento después de dos horas. La variación indica que el movimiento lateral no es automático, sino que se ejecuta manualmente por los grupos de ataque. No se sabe con precisión si los grupos de ataque poseen los recursos para explorar todas las infecciones inmediatamente o si ellos priorizan algunas infecciones sobre otras.

Poco se sabe acerca de cómo los grupos de ataque se mueven en la red de una compañía. La explicación más probable es que ellos exploran contraseñas débiles y una seguridad pobre en archivos compartidos y en las unidades de red. No existe evidencia de ataques de día cero o que se empleen herramientas sofisticadas de hacking.

En algunos casos, los grupos de ataque infectaron computadoras con otros tipos de malware, incluso varias variantes de ransomware y el Troyano financiero Shylock (Trojan.Shylock). En el caso de Shylock, estas parecen ser versiones más antiguas de la amenaza y pueden incluso ser infecciones de "bandera falsa". Pueden utilizarse ​por el grupo cuando creen que su ataque fue descubierto, a fin de engañar a los investigadores.
¿Por qué el área de la salud?
Poco se sabe acerca del grupo responsable de Gatak, si bien la naturaleza corporativa de sus blancos, en conjunto con la ausencia de vulnerabilidades de día cero o módulos de malware avanzados, sugieren que pueden ser cibercriminales en su esencia, aunque también existen recursos dentro del malware para operaciones más tradicionales de espionaje.

No está claro como Gatak está obteniendo lucro con sus ataques. Una posibilidad es el robo de datos, con los grupos de ataque vendiendo información de identificación personal y otros datos robados en el cibermercado clandestino. Esto podría explicar el foco determinado de los grupos de ataque en el sector de la salud, pues los registros con información de salud normalmente se venden a mejores precios comparados con otra información personal.

Sin embargo, los medios de distribución de Gatak, a través de sitios de generación de claves de licencia, indican que los grupos de ataque pueden ser más oportunistas. Al utilizar un abordaje watering-hole, los grupos de ataque desempeñan un papel en gran parte pasivo, con relativamente poco control sobre quien es infectado. Si este es el caso, el sector de la salud puede simplemente ser el más susceptible a este tipo de ataque.

Las compañías de salud muchas veces pueden estar bajo presión, con pocos recursos, y muchas usan sistemas de software legados cuyas actualizaciones son muy caras. Por consiguiente, los profesionales podrían ser más propensos a tomar atajos e instalar software pirata. Si bien parece que las compañías de otros sectores se infectan con menos frecuencia, los grupos de ataque no parecen ignorar o remover esas infecciones cuando se producen.






















un conjunto de datos de malware sin precedentes y promoviendo el desarrollo de técnicas efectivas de código abierto para agrupar variantes de archivos de malware en sus respectivas familias.

Est Fue un challenge que promocionó Microsoft para que la comunidad científica propuesiera una solución para este problema.Es un dataset que fue creado para un Challenge proporcionado por Microsoft para 
Dataset Description
Warning: this dataset is almost half a terabyte uncompressed! We have compressed the data using 7zip to achieve the smallest file size possible. Note that the rules do not allow sharing of the data outside of Kaggle, including bit torrent (why not?).

You are provided with a set of known malware files representing a mix of 9 different families. Each malware file has an Id, a 20 character hash value uniquely identifying the file, and a Class, an integer representing one of 9 family names to which the malware may belong:

Ramnit
Lollipop
Kelihos_ver3
Vundo
Simda
Tracur
Kelihos_ver1
Obfuscator.ACY
Gatak
For each file, the raw data contains the hexadecimal representation of the file's binary content, without the PE header (to ensure sterility).  You are also provided a metadata manifest, which is a log containing various metadata information extracted from the binary, such as function calls, strings, etc. This was generated using the IDA disassembler tool. Your task is to develop the best mechanism for classifying files in the test set into their respective family affiliations.

The dataset contains the following files:

train.7z - the raw data for the training set (MD5 hash = 4fedb0899fc2210a6c843889a70952ed)
trainLabels.csv - the class labels associated with the training set
test.7z - the raw data for the test set (MD5 hash = 84b6fbfb9df3c461ed2cbbfa371ffb43)
sampleSubmission.csv - a file showing the valid submission format
dataSample.csv - a sample of the dataset to preview before downloading
\end{comment}





\subsection{Autoencoder}
\subsection{Red Neuronal Convolucional}
\subsection{Resultados}
El entorno de hardware en el que he realizado todos los experimentos es un sistema operativo Debian 12.2 con Linux version 6.1.0-17-amd64. La CPU utilizada es un Intel(R) Xeon(R) W-2235 CPU con 3.8 GHz ($3.80*10^9$Hz) con 6 núcleos. La memoria RAM disponible es de 128 GB. 


\section{Detección de intrusiones}
\subsection{KDD Cup 1999}
\subsection{Autoencoder}
Para la clasificacion binaria usar autoencoder con el entrenamiento de las imágenes (buenas o malas) y según el error que den, se clasifica.
Para la multiclasificación, tenemos dos opciones:
\begin{itemize}
\item Usamos autoencoder para comprimir la información de entrada y despues esa informacion la usamos para clasificarla usando una DNN \citep{lopes2022effective}
\item Usamos una cadena de autoencoders en el cual la salida de h es la entrada del autoencoder h+1. Utilizo el articulo \citep{farahnakian2018deep} donde se desarrolla todo el modelo y explicacion y ademas se hace referencia al artículo \citep{bengio2006greedy} porque se basa en él (lo de salida de h es la entrada de h+1). Ver tambien:
\begin{itemize}
\item Asymmetric Stacked Autoencoder
\item Constrained Nonlinear Control Allocation based on Deep Auto-Encoder Neural Networks.

\end{itemize} El algoritmo consiste en entrenar las capas por separado en la que el input del autoencoder es la salida del autoencoder anterior. Lo que de verdad nos interesa es la capa oculta, que tiene una representación comprimida de los datos de entrada y sus pesos. Estos pesos son con los que se inicializa el entrenamiento de la stacked autoencoder acabando en softmax. He usado el url para enterlo \url{https://amiralavi.com/tied-autoencoders/}. Además en \citep{bao2017deep} explica bastante bien la diferencia entre capa autoencoder y un autoencoder.
\end{itemize}
\subsection{Red Neuronal Convolucional}
Para clasificar los datos del dataset KDD 1999 usando las \gls{cnn} vamos a seguir los siguientes articulos \citep{kim2020cnn, yang2006anomaly, nguyen2018design, kim2018encoding}. Prácticamente todo el cuerpo del experimento se encuentra en el artículo \citep{kim2020cnn}, pero en el artículo \citep{kim2018encoding} aparece la parte de normalización de los datos y algunos hiperparametros de inicio.

\subsection{Red Neuronal Profunda}
Por otro lado, el método \gls{dnn} utiliza una arquitectura muy parecida a una CNN. Podemos ver todo el procesamiento de los datos y el modelo en el artículo \citep{maithem2021network}. Además, hay buena explicacion del experimento en \citep{vigneswaran2018evaluating}. Por último, en el articulo \citep{elmasry2019empirical} están los experimentos con DNN, RNN,  RBM que puedo tomar también como referencia porque está muy bien explicado las capas e hiperparametros que utiliza.

\subsection{Red Neuronal Recurrente}
En el articulo \citep{elmasry2019empirical} están los experimentos con DNN, RNN,  RBM que puedo tomar también como referencia porque está muy bien explicado las capas e hiperparametros que utiliza.

\subsection{Restricted Boltzmann Machine}
En el articulo \citep{elmasry2019empirical} están los experimentos con DNN, RNN,  RBM que puedo tomar también como referencia porque está muy bien explicado las capas e hiperparametros que utiliza.


\subsection{Resultados}
El entorno de hardware en el que he realizado todos los experimentos es un sistema operativo Debian 12.2 con Linux version 6.1.0-17-amd64. La CPU utilizada es un Intel(R) Xeon(R) W-2235 CPU con 3.8 GHz ($3.80*10^9$Hz) con 6 núcleos. La memoria RAM disponible es de 128 GB. 





\chapter{Detección de intrusiones} \label{Capitulo4}

En la era actual, Internet se ha convertido en una herramienta esencial en nuestra vida cotidiana, facilitando actividades en áreas como los negocios, el entretenimiento y la educación \citep{tsai2009intrusion}. Sin embargo, esta tendencia también ha acarreado un mayor riesgo de ataques en la red, lo que subraya la necesidad de sistemas de seguridad mas potentes. Para garantizar la seguridad de las redes, es esencial contar con sistemas de detección de intrusiones capaces de identificar estos ataques (IDS, por sus siglas en inglés) \citep{liu2019machine}.


La detección de anomalías se centra en identificar desviaciones de los patrones normales de uso, asumiendo que tales desviaciones pueden representar intentos de intrusión. Estos sistemas han evolucionado significativamente, incorporando técnicas avanzadas de aprendizaje automático para identificar diferencias entre datos anómalos y benignos con alta precisión. Además, estas técnicas han demostrado ser efectivas para manejar grandes volúmenes de datos y mejorar la capacidad de detección de nuevos tipos de ataques \citep{liu2019machine}.

En este capítulo vamos a enfrentar este problema, escogiendo una de las bases de datos disponibles en \citep{podder2021artificial} para poder identificar intrusiones. Como el objetivo principal de este trabajo es el estudio y puesta en práctica de diferentes algoritmos de aprendizaje automático, se ha decidido tomar como base de datos \textit{KDD CUP 1999} \citep{kddcup99}, uno de los conjuntos de datos más utilizados para evaluar sistemas de detección de intrusiones. Otro de los motivos de esta elección ha sido la gran variedad de métodos que están documentados en el review para poder resolverlo, en particular de cuatro formas diferentes. Usando una \acrfull{cnn}, una \acrfull{dnn}, una \acrfull{rnn} y un \textit{autoencoder}. 




\section{KDD Cup 1999 Data}

La base de datos\textit{ KDD CUP 1999} se desarrolló a partir de un conjunto de datos recopilado durante el Programa de Evaluación de Detección de Intrusiones DARPA de 1998, gestionado por los Laboratorios Lincoln del MIT (\textit{Massachusetts Institute of Technology}). Este programa tenía como objetivo evaluar métodos de detección de intrusiones mediante la simulación de ataques en una red de área local (LAN) que imitaba el entorno de un área de la Fuerza Aérea de EE.UU \citep{kddcup99}.

\newpage

Para crear el conjunto de datos, los Laboratorios Lincoln capturaron datos de TCP durante nueve semanas en esta red simulada. El conjunto de datos de entrenamiento incluye aproximadamente siete semanas de tráfico de red comprimido en cuatro \textit{Gbytes} de datos binarios, procesados en unos cinco millones de registros de conexión. Los datos de prueba abarcan dos semanas adicionales y contienen alrededor de dos millones de registros de conexión \citep{kddcup99}.

Una conexión en el conjunto de datos se define como una secuencia de paquetes TCP entre una dirección IP de origen y una dirección IP de destino, con un inicio y fin definidos. Cada conexión está etiquetada como normal o como un ataque, con uno de los varios tipos específicos de ataque. Este enfoque permite analizar el comportamiento de las conexiones en un entorno controlado y detectar patrones asociados con actividades maliciosas \citep{kddcup99}.

En \citep{kddcup99} nos encontramos con los siguientes archivos:

\begin{itemize}
    \item \textit{taskdescription}: Esta es la descripción original de la tarea proporcionada a los participantes de la competencia.
    \item \textit{kddcup.names}: Una lista de características.
    \item \textit{kddcup.data.gz}: El conjunto de datos completo (18M; 743M sin comprimir).
    \item \textit{kddcup.data\_10\_percent.gz}: Un subconjunto del 10\% (2.1M; 75M sin comprimir).
    \item \textit{kddcup.newtestdata\_10\_percent\_unlabeled.gz}: Un subconjunto del 10\% de datos de prueba sin etiquetas (1.4M; 45M sin comprimir).
    \item \textit{kddcup.testdata.unlabeled.gz}: Datos de prueba sin etiquetas (11.2M; 430M sin comprimir).
    \item \textit{kddcup.testdata.unlabeled\_10\_percent.gz}: Un subconjunto del 10\% de datos de prueba sin etiquetas (1.4M; 45M sin comprimir).
    \item \textit{training\_attack\_types}: Una lista de tipos de intrusión.
    \item \textit{typo-correction.txt}: Una breve nota sobre un error tipográfico en el conjunto de datos que ha sido corregido (26/6/07).
\end{itemize}

En este caso, utilizaremos únicamente el conjunto de datos \textit{kddcup.data.gz} para hacer nuestros experimentos ya que necesitamos saber si una instancia es buena o mala para entrenarla en nuestras redes neuronales. 

Cada registro del conjunto de datos contiene 41 características diferentes, las cuales se puede ver su descripción en la tabla \ref{tab: caracKDD}. Además, contiene una característica final con el nombre con el que se clasifica la instancia. Podemos visualizar un ejemplo de como nos aparecen las instancias en el archivo \textit{kddcup.data.gz} en la figura \ref{img: kddpreview}. Estas características se dividen en tres grupos principales: características de tráfico, características de conexión TCP y características de contenido.
\begin{figure}[h]
    \begin{center}
    \includegraphics[width=\textwidth]{img/kddpreview.png}
    \end{center}
    \caption{Visualización del archivo \textbf{kddcup.data.gz}.}
    \label{img: kddpreview}
\end{figure}

\newpage


\begin{table}[h!]
    \centering
    \begin{tabular}{|c|l|l|}
        \hline
        \textbf{No.} & \textbf{Descripción} & \textbf{Tipo} \\
        \hline
        \multicolumn{3}{|c|}{\textbf{Características de Conexiones TCP}} \\
        \hline
        1 & duración de la conexión (s) & int64 \\
        2 & tipo protocolo & str \\
        3 & tipo servicio red destino & str \\
        4 & estado de la conexión & str \\
        5 & bytes origen-destino & int64 \\
        6 & bytes destino-origen & int64 \\
        7 & 1 si puerto/host similar, 0 otro caso & int64 \\
        8 & fragmentos incorrectos & int64 \\
        9 & paquetes urgentes & int64 \\
        \hline
        \multicolumn{3}{|c|}{\textbf{Características de Contenido}} \\
        \hline
        10 & indicaciones calientes & int64 \\
        11 & intentos de inicio de sesión fallidos & int64 \\
        12 & 1 si inicio de sesión aceptado & int64 \\
        13 & operaciones comprometidas & int64 \\
        14 & 1 si acceso root en shell & int64 \\
        15 & invocación de su root & int64 \\
        16 & accesos concedidos como root & int64 \\
        17 & creaciones de archivos & int64 \\
        18 & sesiones de shell & int64 \\
        19 & operaciones de acceso a archivos & int64 \\
        20 & instrucciones salientes en ftp & int64 \\
        21 & 1 si inicio de sesión es de host & int64 \\
        22 & 1 si inicio de sesión como invitado & int64 \\
        \hline
        \multicolumn{3}{|c|}{\textbf{Características de Tráfico}} \\
        \hline
        23 & enlaces en 2s al mismo host & int64 \\
        24 & enlaces en 2s al mismo servicio & int64 \\
        25 & \% fallos SYN & float64 \\
        26 & \%  conexiones hacia mismo servicio con errores SYN & float64 \\
        27 & \% fallos REJ & float64 \\
        28 & \%  conexiones hacia el mismo servicio que tuvieron errores REJ & float64 \\
        29 & \% conexiones a un solo servicio & float64 \\
        30 & \% conexiones a varios servicios & float64 \\
        31 & \% conexiones a varios hosts & float64 \\
        32 & conexiones al mismo host de destino & int64 \\
        33 & conexiones al mismo servicio & int64 \\
        34 & \% conexiones al mismo host & float64 \\
        35 & \% conexiones varios servicios & float64 \\
        36 & \% conexiones mismo puerto de origen & float64 \\
        37 & \% conexiones varios hosts al mismo servicio & float64 \\
        38 & \% conexiones fallos S0 al host actual & float64 \\
        39 & \% conexiones fallos S0 al host y servicio & float64 \\
        40 & \% fallos RST al host actual & float64 \\
        41 & \% fallos RST al host y servicio & float64 \\
        42 & tipo de ataque & str \\
        \hline
    \end{tabular}
    \caption{Características del Conjunto de Datos KDD}
    \label{tab: caracKDD}
\end{table}

\begin{itemize}
    \item \textbf{Características de conexiones TCP}: Examina las características que describen los atributos básicos de cada conexión TCP.
    \item \textbf{Características basadas en el contenido}: Estas características examinan el contenido de los paquetes para identificar patrones sospechosos.
    \item \textbf{Características de tráfico}: Estas características analizan patrones de tráfico en la red para detectar anomalías.
\end{itemize}


El conjunto de datos de entrenamiento del \textit{KDD Cup 1999} comprenden diferentes formas de amenazas y datos normales. Cada uno de los datos anómalos pertenecen a una de las siguientes 22 formas diferentes de ciberataques (ver tabla \ref{tab: catKDD}). Estos ataques pueden clasificarse en 4 grupos, que según \citep{arief2022comparison}, podemos describirlos como:
\begin{itemize}
    \item \textbf{Ataque de Denegación de Servicio (DoS)}: ocurre cuando un intruso impide que los usuarios autorizados accedan a un sistema sobrecargando los recursos del sistema (computación o memoria), haciéndolo incapaz de ejecutar solicitudes válidas, como en el caso de una inundación SYN.
    \item \textbf{Ataque de Usuario a Root (U2R)}: ocurre cuando el intruso obtiene permisos locales para acceder al sistema como un usuario legítimo y luego intenta explotar debilidades para obtener acceso root al sistema, ganando así capacidades de supervisión.
    \item \textbf{Ataque de Remoto a Local (R2L)}: ocurre cuando el intruso transmite un paquete a través de la red desde una estación de trabajo remota sin proporcionar el permiso adecuado, como al intentar varias contraseñas.
    \item \textbf{Ataque de Sondeo}: ocurre cuando un intruso intenta obtener conocimiento sobre la red para descubrir fallos de seguridad. A través de esta operación, el intruso explora la arquitectura de la red e identifica las categorías de servicios disponibles en el sistema utilizando técnicas como un escaneo de puertos.
\end{itemize}

En la tabla \ref{tab: catKDD} podemos ver los tipos de ataque que incluye cada grupo y el número total de muestras que tiene.

\begin{table}[h!]
    \centering
    \begin{tabular}{|l|l|c|}
        \hline
        \textbf{Categoría} & \textbf{Ataque} & \textbf{Total}\\
        \hline
        DOS & pod, land, neptune, smurf, teardrop, back & 3.883.370\\
        R2L & warezclient, imap, phf, spy, ftpwrite, guesspswd, warezm, multihop & 1.126\\
        U2R & rootkit, loadmodule, perl, buffer\_ overflow & 52\\
        Probing & satan, nmap, portsweep, ipsweep & 41.102\\
        Normal & & 972.781 \\
        \hline
    \end{tabular}
    \caption{Categorías de Ataques}
    \label{tab: catKDD}
\end{table}


El propósito de la competición \textit{KDD Cup 1999}, en la cual se utilizó este conjunto de datos, era desarrollar un modelo predictivo capaz de distinguir entre conexiones normales y ataques \citep{kddcup99}. Este conjunto de datos se ha convertido en una referencia clave en estudios sobre la detección de intrusiones en redes debido a su riqueza y diversidad de ejemplos de ataques.



\section{Preparación datos} \label{sec:datoskdd}

De estas 41 características de tráfico, 38 son numéricas y 3 son simbólicas. Para unificar los formatos de datos, transformamos las características simbólicas en datos numéricos utilizando vectores \emph{one-hot}. Las 3 características simbólicas son: el tipo de protocolo en la capa TCP/IP, el tipo de servicio del sistema objetivo, y el tipo de bandera que indica el estado de la conexión de la sesión. Tenemos tres tipos de protocolos, ICMP, TCP, y UDP. Estos protocolos se transforman en un vector de tres dimensiones, resultando en las representaciones (1,0,0), (0,1,0) y (0,0,1) respectivamente. De manera similar, los 70 tipos de servicio, que incluyen HTTP y FTP, se transforman en vectores de 70 dimensiones, y las 11 características de tipo de bandera se convierten en vectores de 11 dimensiones. A través de estas transformaciones, generamos un vector de 84 dimensiones, que al combinar con las 38 características originales numéricas, obtenemos un vector final de 122 dimensiones.

Además, cada valor lo estandarizamos utilizando la siguiente fórmula:
\begin{equation}
    X_{inorm} = \frac{x_i-x_{imin}}{x_{imax}-x_{imin}}
\end{equation}

donde $X_{inorm} \in \mathbb{R}$ hace referencia al nuevo valor normalizado, $x_i \in \mathbb{R}$ al valor que se encuentra en la base de datos y $x_{imax}$, $x_{imin} \in \mathbb{R}$ al valor máximo y mínimo de esa característica respectivamente. De esta forma, todas las características se van a encontrar entre $[0,1]$ \citep{kim2018encoding}.

Ahora, al igual que en el problema de clasificación de malware del Capítulo \ref{Capitulo_3},  vamos a dividir el conjunto de datos aleatoriamente usando la función \lstinline|train_test_split()| en grupos del 75\%, 15\% y 10\% para entrenamiento, test y validación respectivamente.

Para abordar este problema de clasificación, vamos a realizar cuatro modelos de aprendizaje automático diferentes para luego comparar sus resultados. El primer método que vamos a utilizar es una \acrfull{dnn}, seguido de una \acrfull{rnn} y  un \textit{autoencoder}. Para finalizar, utilizaremos una \acrfull{cnn} para la detección de intrusiones. Todos los modelos que vamos a crear han sido sacados de algún artículo que aparece en el \textit{review} utilizado a lo largo de todo el trabajo \citep{podder2021artificial}







\section{Red Neuronal Profunda}

Vamos a empezar creando modelo de \acrfull{dnn} con una capa de entrada, dos capas ocultas y una capa de salida. En esta sección se describe en detalle la arquitectura del modelo, los hiperparámetros utilizados y el proceso de entrenamiento siguiendo principalmente los artículos \citep{maithem2021network, vigneswaran2018evaluating}. Aunque la gran mayoría del trabajo de esta sección ha sido replicar los experimentos de estos artículos, se ha añadido el \textit{callback }\lstinline|EarlyStopping| para prevenir el sobreajuste del modelo. 

\subsection{Arquitectura del Modelo}

La arquitectura del modelo DNN consta de varias capas densas interconectadas, comenzando con la capa de entrada que recibe los datos preprocesados. La capa de entrada está compuesta por 122 neuronas, cada una de las cuales corresponde a una característica del conjunto de datos tras el procesamiento visto anteriormente. 

A continuación, se encuentran dos capas ocultas. La primera capa oculta contiene 50 neuronas con la función de activación ReLU, seleccionada por su capacidad para manejar de manera eficiente las no linealidades y su facilidad de entrenamiento. La segunda capa oculta incluye 30 neuronas, también con la función de activación ReLU. Esta estructura permite al modelo aprender representaciones complejas de los datos a través de la jerarquía de capas. 

Por último, la capa de salida del modelo está diseñada para la tarea de clasificación, utilizando la función de activación \lstinline|softmax|. Esta capa convierte las salidas en probabilidades de pertenencia a cada una de las 5 clases. El número de neuronas en la capa de salida se corresponde entonces con 5, ya que con 5 neuronas podemos hacer un estudio multiclase y a su vez se puede modificar para obtener una clasificación binaria. La Figura~\ref{fig:kddDNN} ilustra la arquitectura general de la DNN propuesta.

\begin{figure}[h]
    \centering
    \includegraphics[width=0.4\textwidth]{img/kddDNN.png}
    \caption{Arquitectura de la Red Neuronal Profunda (DNN), donde el color verde hace referencia a que se clasifica como benigno y el color rojo como uno de los cuatro tipos de malware.}
    \label{fig:kddDNN}
\end{figure}

Una vez tenemos definida la arquitectura de nuestro modelo, es hora de elegir los hiperparámetros necesarios para entrenarlo. El optimizador elegido es Adam (\textit{Adaptive Moment Estimation}), configurado con una tasa de aprendizaje inicial de 0.001. Para la función de pérdida, se utilizó la entropía cruzada categórica (\lstinline|categorical_crossentropy|). Esta función es adecuada para problemas de clasificación multiclase, como nuestro problema con la base de datos \textit{KDD Cup 1999}. Aunque en el análisis que vamos a hacer posteriormente solo tenemos en cuenta si las instancias son malignas o benignas, se ha decidido hacer una clasificación multiclase ya que se obtiene mayor información y que además posteriormente se puede pasar a una clasificación binaria fácilmente. 

El modelo se entrenó durante 30 épocas, utilizando un tamaño de lote de 32 muestras por lote. Esta configuración permite un balance adecuado entre la velocidad de entrenamiento y la estabilidad de la convergencia. 

Además, he decidido añadir el \textit{callback} \lstinline|EarlyStopping| para detener el entrenamiento si la pérdida en el conjunto de validación no mejora después de 5 épocas, ayudando a prevenir el sobreajuste. Además, se emplea un \lstinline|CSVLogger| para registrar las métricas de entrenamiento y validación en cada época. La Figura~\ref{fig:kdd_dnn} presenta la evolución de la precisión y la pérdida a lo largo de las épocas de entrenamiento, destacando la mejora continua del modelo.

\begin{figure}[h]
    \centering
    \includegraphics[width=0.6\textwidth]{img/KDD_DNN.csv.png}
    \caption{Evolución de la Precisión y la Pérdida durante el Entrenamiento del Modelo DNN.}
    \label{fig:kdd_dnn}
\end{figure}

En la imagen \ref{fig:kdd_dnn} se aprecian fluctuaciones en la pérdida de entrenamiento y validación a lo largo de todo el entrenamiento. La pérdida desciende hasta la época 6 y luego asciende. En cuanto a la precisión, se observa que es muy alta a lo largo de todo el entrenamiento, alrededor del 0.99. Aunque las curvas muestran fluctuaciones, estas son en realidad pequeñas variaciones en las centésimas del porcentaje. 

Podemos apreciar que, a pesar de que el número de épocas iniciales era de 30, no aparecen resultados a partir de la 11. Esto es debido a que en la época 6 el modelo ha obtenido un \lstinline|val_loss = 0,001| y desde esta época a la 11 no ha obtenido un valor menor, luego se ha producido la parada temprana, restaurando los pesos de la época 6. El valor del \textit{training loss} en esta época es de 0,00231. Por otra parte, la precisión de clasificación multiclase en esta época es de 0,997 en el entrenamiento y 0,998 en la validación. 


\subsection{Evaluación del modelo}

Después de entrenar el modelo, se evalúa como una clasificación binaria, usando el conjunto de datos que habíamos reservado para esta parte. Para empezar vamos a ver la matriz de confusión \ref{fig: dnnKDDConfusion} con el número de instancias que están bien clasificadas y mal clasificadas.

\begin{figure}[H]
    \centering
    \includegraphics[width=0.4\textwidth]{img/confusionMatrixDNN_KDD.png}
    \caption{Matriz de confusión del modelo utilizando una DNN.}
    \label{fig: dnnKDDConfusion}
\end{figure}


En ella se puede observar como la mayoría de las predicciones están bien clasificadas ya que la mayoría de muestras se encuentran en la diagonal. Para poder ver con más detalle estos resultados, veamos ahora la tabla \ref{tab:kddDNN}, en la que podremos ver la precisión del modelo, su sensibilidad, la especificidad y el \textit{F1-Score}. Los resultados indican un desempeño sobresaliente en todas las métricas evaluadas, cada una con un valor de 0.999.

\begin{table}[H]
\centering
\begin{tabular}{|c|c|c|c|c|}
\hline
Modelo & Precisión & Sensibilidad & Especificidad & F1 Score \\ \hline
DNN & 0.999 & 0.999 & 0.999 & 0.999 \\ \hline
\end{tabular}
\caption{Métricas de rendimiento de la red neuronal profunda.}
\label{tab:kddDNN}
\end{table}

La precisión de 0.999 indica que el modelo clasifica correctamente prácticamente todos los registros de conexión. La especificidad de 0.999 muestra que el modelo detecta casi todos los registros anómalos, minimizando los falsos benignos. La sensibilidad de 0.999 revela que el modelo identifica correctamente casi todos los datos benignos. Finalmente, el \textit{F1-Score} obtiene un 0.999, lo que confirma gran capacidad del modelo para identificar correctamente tanto las instancias benignas como anómalas.










\section{Red Neuronal Recurrente}

Después de evaluar la red neuronal profunda, vamos a crear otro modelo predictivo para ver si podemos obtener unos resultados mejores. En este caso, vamos a enfocar el estudio en implementar una \acrfull{rnn} con \textit{Long Short-Term Memory} (RNN-LSTM) para este problema de detección de intrusiones. Se va a seguir principalmente las metodologías propuestas en \citep{yin2017deep,kim2016long}.

\subsection{Arquitectura del Modelo}

El modelo LSTM comienza con una capa de entrada que recibe los datos preprocesados anteriormente con una dimensión de 122 características. Estos datos son procesados por una capa LSTM que contiene 80 unidades ocultas. Este número de neuronas ha sido escogido después de obtener los mejores resultados comparado con otro números de neuronas ocultas en el artículo \citep{yin2017deep}. Esta capa LSTM es crucial ya que es capaz de capturar características a largo plazo en secuencias de datos, lo cual es esencial para nuestro estudio.

A continuación, la salida de la capa LSTM se pasa a una capa completamente conectada (capa Densa). Esta capa Densa tiene 5 neuronas, que corresponden a las 5 clases de salida. Se escoge 5 clases en vez de 2 porque creemos que este número de neuronas nos da más información y además nos da la posibilidad de hacer un análisis más extenso de sus resultados. La función de activación utilizada en esta capa es \textit{softmax}, la cual convierte los valores de salida en una distribución de probabilidad sobre las posibles clases, permitiendo así realizar la clasificación. Podemos ver un esquema de la arquitectura en la imagen \ref{fig:kddRNN}.

\begin{figure}[h]
    \centering
    \includegraphics[width=0.4\textwidth]{img/kddRNN.png}
    \caption{Arquitectura de la Red Neuronal Recurrente (RNN), donde el color verde hace referencia a que se clasifica como benigno y el color rojo como uno de los cuatro tipos de malware.}
    \label{fig:kddRNN}
\end{figure}

Una vez definido el modelo, se procede a su compilación utilizando el optimizador \textit{Stochastic Gradient Descent} (SGD) con una tasa de aprendizaje que se va ajustando a lo largo del entrenamiento. Este optimizador se elige por su simplicidad y eficacia en muchos problemas de aprendizaje automático. Para la función de pérdida, se utilizó la entropía cruzada categórica (\lstinline|categorical_crossentropy|) como en el modelo anterior ya que tenemos 5 neuronas en la capa de salida.

El modelo se entrena durante 50 épocas con un tamaño de lote de 50. Al igual que en el modelo de DNN, he decidido añadir los \textit{callbacks} \lstinline|CSVLogger| y \lstinline|EarlyStopping| para registrar el proceso de entrenamiento y para prevenir el sobreajuste respectivamente. La Figura~\ref{fig:kdd_rnn} presenta la evolución de la precisión y la pérdida a lo largo de las épocas de entrenamiento.

\begin{figure}[h]
    \centering
    \includegraphics[width=0.55\textwidth]{img/KDD_RNN.csv.png}
    \caption{Evolución de la Precisión y la Pérdida durante el Entrenamiento del Modelo RNN - LSTM.}
    \label{fig:kdd_rnn}
\end{figure}

En la imagen \ref{fig:kdd_rnn}, observamos la evolución de la pérdida de entrenamiento y validación (gráfica izquierda). La pérdida de entrenamiento comienza alta, con un valor de 0.21, y disminuye rápidamente en las primeras 10 épocas. La pérdida de validación sigue un patrón similar, disminuyendo rápidamente y convergiendo con la pérdida de entrenamiento alrededor de la época 10. Ambas curvas se estabilizan alrededor de la época 20 y permanecen bajas, llegando a un valor de 0.004 para ambos subconjuntos. 

En la gráfica de la derecha, se observa la precisión de entrenamiento y validación. La precisión crece rápidamente en las primeras 10 épocas, para estabilizarse a partir de esa ápoca. Los datos de entrenamiento comienzan con un valor de 0.843, y aumenta rápidamente, alcanzando un valor de 0.999. La precisión de validación sigue de cerca a la precisión de entrenamiento, también alcanzando valores muy altos, 0.999. 

En ambas gráficas se observa un comportamiento muy parecido de ambas curvas, además de converger a unos resultados excelentes rápidamente, lo cual es un buen indicio de que el modelo está aprendiendo eficientemente y generalizando bien. 


\subsection{Evaluación del modelo}

Después de entrenar el modelo, se evalúa como una clasificación binaria, usando el conjunto de prueba que habíamos reservado anteriormente. Para empezar, vamos a ver la matriz de confusión \ref{fig: rnnKDDConfusion} con el número de registros de la \textit{KDD CUP 1999} que están bien clasificados y mal clasificados.

\begin{figure}[H]
    \centering
    \includegraphics[width=0.4\textwidth]{img/confusionMatrixRNN_KDD.png}
    \caption{Matriz de confusión del modelo utilizando una RNN.}
    \label{fig: rnnKDDConfusion}
\end{figure}


En ella, al igual que en el modelo anterior (DNN), se puede observar como la mayoría de las predicciones se encuentran en la diagonal. Para poder ver con más detalle estos resultados, veamos ahora la tabla \ref{tab:kddRNN}, en la que podremos ver la precisión del modelo, su sensibilidad, la especificidad y el \textit{F1-Score}. Los resultados indican un desempeño sobresaliente en todas las métricas evaluadas, cada una con un valor de 0.999.

\begin{table}[H]
\centering
\begin{tabular}{|c|c|c|c|c|}
\hline
Modelo & Precisión & Sensibilidad & Especificidad & F1 Score \\ \hline
RNN & 0.998 & 0.999 & 0.998 & 0.998 \\ \hline
\end{tabular}
\caption{Métricas de rendimiento de la red neuronal recurrente.}
\label{tab:kddRNN}
\end{table}

En la tabla podemos volver a apreciar el gran rendimiento de este modelo con el conjunto de datos \textit{KDD CUP 1999}. Las métricas expuestas anteriormente nos muestran unos resultados excelentes, con valores prácticamente iguales a 1. Esto nos indica que la RNN - LSTM tiene una capacidad casi perfecta para distinguir entre registros anómalos y benignos. 












\section{Autoencoder}
A continuación, se plantea el problema de detección de intrusiones utilizando uno de los métodos más utilizados para resolver este problema, el \textit{autoencoder} \citep{podder2021artificial}. La estrategia consiste en entrenar el \textit{autoencoder} exclusivamente con un subconjunto de datos benignos. Posteriormente, se evalúa la capacidad del modelo para reconstruir los demás registros. En función del error de reconstrucción, los registros se clasificarán como normales o como malware, basándose en un umbral predefinido. Este enfoque se basa principalmente en el trabajo de \citep{xu2021improving}.


\subsection{Arquitectura del modelo}

El \textit{autoencoder} propuesto para esta sección contiene 5 capas densas en las que se codifica una representación de características de 122 dimensiones en un vector de 5 dimensiones en su capa latente. Antes de llegar a esta dimensión, primero se reduce de 122 a 32 para a continuación reducir aun más hasta llegar a 5. Después, este representación comprimida de los datos se decodifican de vuelta a un vector de 32 dimensiones y después a 122 como los datos de entrda. En la imagen \ref{fig:architectureAE} podemos ver de manera más gráfica su arquitectura.

\begin{figure}[H]
    \centering
    \includegraphics[width=0.4\textwidth]{img/kddAE.png} 
    \caption{Arquitectura del \textit{autoencoder} con 5 capas densas.}
    \label{fig:architectureAE}
\end{figure}

Las capas ocultas son capas densas, es decir, están completamente conectadas y utilizan como función de activación ReLU. El error de reconstrucción entre la entrada $x$ y la salida reconstruida $\hat{x}$ se calcula usando el error absoluto medio (MAE).

Una vez definido el modelo, se procede a compilarlo utilizando el optimizador Adam con un \textit{learning rate} inicial de 0,001. 

\begin{minipage}{0.6\textwidth}
    El modelo se entrena durante 30 épocas en mini lotes de 30 registros. Al igual que en los dos modelos anteriores, he decidido añadir los \textit{callbacks} \lstinline|CSVLogger| y \lstinline|EarlyStopping| para registrar el proceso de entrenamiento y para prevenir el sobreajuste respectivamente. La Figura~\ref{fig:trainingAE} presenta la evolución de la precisión y la pérdida a lo largo de las épocas de entrenamiento.

    \bigskip

    En la figura, se observa la evolución de la pérdida de entrenamiento y validación a lo largo de las 30 épocas. Al inicio del entrenamiento, se aprecia un descenso rápido en ambas curvas de pérdida. Esto indica que el modelo está aprendiendo eficazmente las características de los datos durante las primeras épocas, lo que es un buen signo de que la fase inicial del entrenamiento está ajustando bien los pesos.
\end{minipage}
\begin{minipage}{0.05\textwidth}
    \textbf{ }
\end{minipage}
\begin{minipage}{0.33\textwidth}
    
    \begin{figure}[H]
    \centering
    \includegraphics[width=\textwidth]{img/KDD_AE.csv.png}
    \captionof{figure}{Evolución de la pérdida durante el entrenamiento del \textit{autoencoder}}
    \label{fig:trainingAE}
    \end{figure}

\end{minipage}


A partir de la época 15, tanto la pérdida de entrenamiento como la de validación comienzan a estabilizarse y converger alrededor de un valor constante 0,0064. Este comportamiento sugiere que el modelo ha alcanzado un punto de equilibrio donde las actualizaciones de los pesos no producen mejoras significativas, indicando una buena adaptación del modelo a los datos. Es importante destacar que a pesar de la buena adaptación del modelo a los datos de entrenamiento, ambas curvas son muy similares a lo largo de todas las épocas, lo que indica que el modelo generaliza bien los datos de entrenamiento sin llegar al sobreajuste.


\subsection{Evaluación del modelo}


Después de entrenar el modelo con los datos normales de entrenamiento, se procede a la evaluación del modelo. Para ello, primero tenemos que fijar un umbral que nos diga si un registro es malware o normal dependiendo de su error de reconstrucción. Para ello, utilizando la función de \lstinline|sklearn.metrics|, \lstinline|precision_recall_curve|, se calcula la curva de precisión y recuperación (\textit{Precision-Recall}) en función de diferentes umbrales que se utilicen para clasificar los registros. A partir de esta curva, se selecciona un umbral alto que balancee la precisión y la recuperación para la detección de anomalías. Este proceso se ilustra en la siguiente gráfica:

\begin{figure}[H]
    \centering
    \includegraphics[width=0.4\textwidth]{img/umbral.png} 
    \caption{Gráfica de precisión y recuperación en función del umbral}
    \label{fig:precision_recall}
\end{figure}

Nuestro objetivo consiste en minimizar los registros anómalos que se clasifiquen como normales. Para ello, tenemos que encontrar un umbral que maximice el \textit{recall}. Un valor que maximiza tanto el recall como la precisión es 0.02, luego se decide utilizar este valor como umbral. 

Con este umbral, vamos a evaluar el modelo usando los datos de prueba. Para empezar, vamos a ver la matriz de confusión \ref{fig: aeKDDConfusion} con el número de registros de la \textit{KDD CUP 1999} que están bien clasificados y mal clasificados.

\begin{figure}[H]
    \centering
    \includegraphics[width=0.4\textwidth]{img/confusionMatrixAE_KDD.png}
    \caption{Matriz de confusión del modelo utilizando una \textit{autoencoder}.}
    \label{fig: aeKDDConfusion}
\end{figure}


En ella, al igual que en los otros dos modelos, se obtienen unos resultados excelentes. Veamos ahora la tabla \ref{tab:kddAE} con el resto de métricas.
\begin{table}[H]
\centering
\begin{tabular}{|c|c|c|c|c|}
\hline
Modelo & Precisión & Sensibilidad & Especificidad & F1 Score \\ \hline
Autoencoder & 0.999 & 0.93 & 0.999 & 0.991 \\ \hline
\end{tabular}
\caption{Métricas de rendimiento del \textit{autoencoder}.}
\label{tab:kddAE}
\end{table}

En la tabla podemos volver a apreciar el gran rendimiento de este modelo con el conjunto de datos \textit{KDD CUP 1999}. Las métricas expuestas anteriormente nos muestran unos resultados excelentes, con valores prácticamente iguales a 1. Estos nos indica que la arquitectura de \textit{autoencoder} utilizada tiene una capacidad casi perfecta para detectar tanto registros benignos como registros anómalos. 






\section{Red Neuronal Convolucional}

Por último, vamos a plantear el problema de detección de intrusiones utilizando una CNN. Este problema no estaba clasificado en el \textit{review} \citep{podder2021artificial}, pero con el conocimiento que ya poseíamos sobre este método unido a la búsqueda de un artículo que tratara este tema, hemos decidido optar por hacer un estudio con esta red neuronal. Para ello, se va a seguir la metodología que se usa en \citep{kim2020cnn, yao2021intrusion}.

\subsection{Visualización de los datos}

En la subsección \ref{sec:datoskdd} estudiamos como procesar los datos originales de la base de datos, a datos que pudieran ser utilizados por la redes neuronales. Pasamos de un vector de 41 características con diferentes tipos de datos, a un vector de 122 características donde todos los elementos eran números reales entre 0 y 1. Como bien hemos visto en el capítulo \ref{Capitulo_2}, las CNN son redes neuronales que están diseñadas para procesar datos almacenados en matrices. En este caso, al igual que en el modelo de CNN utilizado en la clasificación de malware, vamos a pasar del vector unidimensional a un vector bidimensional. 

En el artículo \citep{kim2020cnn}, los datos son más antiguos que los actuales y en vez de tener un vector de 122 dimensiones, ellos tienen un vector de 117 dimensiones que redimensionan a una matriz $13 \times 9$. En el artículo \citep{yao2021intrusion}, pasan del vector de 41 características a una matriz $7 \times 7$ y los valores restantes le asignan un 0. En nuestro caso, he decido hacer un estudio más exhaustivo de las 122 características para ver que opciones podía plantear para convertir el vector en matriz.

Primero se observó que $122 = 61 \times 2$. Creí que esta matriz no sería muy efectiva de primera mano, ya que había una gran diferencia entre el número de filas y el número de columnas. En este punto, decidí hacer un estudio de las 122 características. Como para estandarizar los datos tuve que calcular los máximos y mínimos de cada característica, decidí fijarme en ellos. Se me ocurrió ver si había alguna característica que siempre valiera lo mismo. Esto era tan fácil como comprobar si algún valor máximo era igual a su valor mínimo. Sea $D_i$ el conjunto de todas las características $i$-ésimas del \textit{dataset}, entonces, 
\begin{equation}
    \exists i \in [1,122] \hspace{2mm} _{\shortparallel} \hspace{2mm} \forall x \in D_i \hspace{2mm} _{\shortparallel} \hspace{2mm} \min(D_i) \leq x \leq \max(D_i) = \min(D_i)
\end{equation}
luego $\forall x \in D_i, x = \min(D_i)$. Comprobando esta propiedad, me dí cuenta que la característica 101 siempre valía 0, es decir la característica 20 de los datos originales (las instrucciones salientes en FTP, que se refiere al número de comandos enviados desde el servidor FTP hacia un cliente durante una sesión de transferencia de archivos). Entonces, he decidido prescindir de esta característica para pasar de un vector de 122 dimensiones a un vector de 121 dimensiones que puede redimensionarse a una matriz $11 \times 11$. 

\begin{figure}[H]
    \centering
    \includegraphics[width=0.2\textwidth]{img/descarga.png}
    \caption{Visualización de un registro benigno del \textit{KDD CUP 1999}.}
    \label{fig: kddimage}
\end{figure}


\subsection{Arquitectura del Modelo}

La arquitectura del modelo CNN se compone de varias capas convolucionales, de \textit{max-pooling} y densas, diseñadas para procesar las matrices \(11 \times 11\) generadas en la fase de preprocesamiento. La primera capa convolucional aplica 64 filtros convolucionales con un tamaño de kernel de \(3 \times 3\), utilizando la función de activación ReLU. Esta capa inicial se encarga de detectar las características locales más importantes de la imagen. 

A continuación, se utiliza una capa de Max-Pooling con un tamaño de ventana de \(2 \times 2\) para reducir la dimensionalidad de los mapas de características, conservando la información relevante y disminuyendo la carga computacional.

La segunda capa convolucional aplica 128 filtros convolucionales con un tamaño de kernel de \(3 \times 3\), también con función de activación ReLU. Esta capa permite capturar características más complejas a un nivel superior de abstracción, detectando patrones más detallados en los datos.

Después de la segunda capa convolucional, se emplea otra capa de \textit{max-pooling} con un tamaño de ventana de \(2 \times 2\), para continuar reduciendo la dimensionalidad de los mapas de características y conservar los patrones más importantes.

Los mapas de características obtenidos se aplanan y pasan a través de una capa densa de 128 neuronas con función de activación ReLU. Esta capa permite compactar la información obtenida por las capas convolucionales. Después, se pasa la información por una capa dens con 64 neuronas y función de activación ReLU refina aún más esta representación. 

Finalmente, se pasa por la capa de salida con la función de activación softmax, que es una capa densa con 5 neuronas como en la RNN y la DNN, ya que de esta forma se obtiene más información aunque luego se haga un análisis de clasificación binaria.

\begin{figure}[h]
    \centering
    \includegraphics[width=0.7\textwidth]{img/kddCNN.png}
    \caption{Arquitectura de la Red Neuronal Convolucional para la detección de intrusiones de la \textit{KDD CUP}.}
    \label{fig:kdd_cnn_architecture}
\end{figure}


Una vez creada la arquitectura de la CNN, se compila el modelo con el optimizador Adam con una tasa de aprendizaje inicial de 0,001. La función de pérdida seleccionada es \lstinline|categorical_crossentropy|. El modelo se entrena en lotes de tamaño 32 durante 30 época. 

Como en los modelos anteriores, se añaden los \textit{callbacks} \lstinline|EarlyStopping| y \lstinline|CSVLogger| para evitar el sobreajuste y registrar las métricas de entrenamiento respectivamente. La evolución del entrenamiento a lo largo de las 30 épocas se puede ver en la imagen \ref{fig:cnn_training_process}.

\begin{figure}[h]
    \centering
    \includegraphics[width=0.5\textwidth]{img/KDD_CNN.csv.png}
    \caption{Evolución de la precisión y la pérdida durante el entrenamiento de la CNN.}
    \label{fig:cnn_training_process}
\end{figure}


En la imagen, podemos observar la evolución de la pérdida de entrenamiento y validación (gráfica izquierda). La pérdida de entrenamiento comienza en un valor bajo, 0.006, y muestra fluctuaciones a lo largo de las épocas, sin una tendencia clara de descenso o ascenso. Es importante notar que estas fluctuaciones son de centésimas, lo que implica que son casi mínimas. La pérdida de validación muestra grandes picos y valles, especialmente al inicio, lo cual indica una inestabilidad en la capacidad del modelo para generalizar en diferentes épocas.

En la gráfica de la derecha, se observa la precisión de entrenamiento y validación. La precisión de entrenamiento comienza en un valor alto, 0.9972, y muestra fluctuaciones a lo largo de las épocas, similar a la pérdida. Nuevamente, estas fluctuaciones son de centésimas, indicando variaciones mínimas. La precisión de validación muestra un patrón muy variable, con grandes picos y valles, reflejando la inestabilidad observada en la pérdida.

Como se puede observar en la gráfica, a pesar de que el entrenamiento estaba previsto para 30 épocas, se ha detenido en la vuelta 19. Esto es porque en la época 9 se obtiene el menor \lstinline|val_loss = 0.0041|. Además, en esta época, la pérdida de entrenamiento es \lstinline|loss = 0.0056|, la precisión de validación es \lstinline|val_accuracy = 0.998| y la precisión de entrenamiento vale \lstinline|accuracy = 0.998|. Esto sugiere que, a pesar de las fluctuaciones observadas, el modelo alcanzó un rendimiento óptimo en esta época.


\subsection{Evaluación del modelo}


Después de entrenar el modelo, se evalúa como una clasificación binaria, usando el 15\% del conjunto de datos reservado para esta parte. Para empezar, vamos a ver la matriz de confusión \ref{fig: cnnKDDConfusion} con el número de registros de la \textit{KDD CUP 1999} que están bien clasificados y mal clasificados.

\begin{figure}[H]
    \centering
    \includegraphics[width=0.4\textwidth]{img/confusionMatrixCNN_KDD.png}
    \caption{Matriz de confusión del modelo utilizando una CNN.}
    \label{fig: cnnKDDConfusion}
\end{figure}


En ella, al igual que en los 3 modelos anteriores, se obtienen unos resultados excelentes. Veamos ahora la tabla \ref{tab:kddCNN} con el resto de métricas.
\begin{table}[H]
\centering
\begin{tabular}{|c|c|c|c|c|}
\hline
Modelo & Precisión & Sensibilidad & Especificidad & F1 Score \\ \hline
CNN & 0.999 & 0.999 & 0.998 & 0.998 \\ \hline
\end{tabular}
\caption{Métricas de rendimiento de la red neuronal convolucional.}
\label{tab:kddCNN}
\end{table}

En la tabla podemos volver a apreciar el gran rendimiento de este modelo con el conjunto de datos \textit{KDD CUP 1999}. Las métricas expuestas anteriormente nos muestran unos resultados excelentes, con valores prácticamente iguales a 1. Estos nos indica que la arquitectura de CNN utilizada tiene una capacidad casi perfecta para detectar tanto registros benignos como registros anómalos. 







\section{Resultados}


En este estudio, se han implementado y evaluado cuatro modelos de redes neuronales diferentes para la detección de intrusiones utilizando el conjunto de datos \textit{KDD Cup 1999}. Los modelos considerados fueron una \acrfull{dnn}, una \acrfull{rnn}, un \textit{autoencoder} y una \acrfull{cnn}. A continuación, se resumen los resultados obtenidos por cada uno de estos modelos.

El modelo DNN mostró un rendimiento sobresaliente con una precisión, sensibilidad, especificidad y \textit{F1-Score} de 0.999 en la clasificación binaria de registros benignos y anómalos. Aunque las fluctuaciones en las curvas de pérdida y precisión indicaron pequeñas variaciones, el modelo logró una excelente generalización. Los resultados detallados se presentan en la Tabla~\ref{tab:conclusionModels}.

El modelo RNN-LSTM también obtuvo un rendimiento notable, con métricas de rendimiento casi perfectas, similar al DNN. La precisión y la pérdida se estabilizaron rápidamente, alcanzando valores de 0.999 en las métricas evaluadas. Las curvas de precisión y pérdida mostraron la mejor estabilidad, inclinándose de manera constante sin fluctuaciones notables, lo que indica una excelente capacidad de aprendizaje y generalización de las características temporales de los datos. Los resultados se detallan en la Tabla~\ref{tab:conclusionModels}.

El \textit{autoencoder} mostró un rendimiento excelente en la detección de intrusiones, con una precisión de 0.999 y un \textit{F1-Score} de 0.991. Aunque su sensibilidad fue ligeramente inferior (0.93), su capacidad para reconstruir los registros benignos y detectar anomalías fue altamente efectiva. El umbral de 0.02 se utilizó para clasificar los registros, maximizando tanto la precisión como el \textit{recall}. Los resultados se presentan en la Tabla~\ref{tab:conclusionModels}.

Por último, la CNN también obtuvo resultados excelentes, con métricas de rendimiento cercanas a 1. Al transformar los vectores de características en matrices $11 \times 11$, el modelo pudo aprender eficazmente las características espaciales de los datos. Los resultados indicaron una gran capacidad para detectar tanto registros benignos como anómalos, con valores muy altos en todas las métricas evaluadas, a pesar de sufrir fluctuaciones a lo largo de todo el entrenamiento (tabla \ref{tab:conclusionModels}).

Aunque todos los modelos obtuvieron resultados excelentes en la detección de intrusiones, el modelo RNN-LSTM destacó por su capacidad de aprender y generalizar eficientemente las características temporales de los datos. Su rendimiento consistente a lo largo de las épocas, junto con sus resultados en las métricas de precisión, sensibilidad, especificidad y \textit{F1-Score}, lo convierten en el mejor modelo entre los evaluados. Por lo tanto, se concluye que la RNN-LSTM es la opción más efectiva para la detección de intrusiones en el conjunto de datos \textit{KDD Cup 1999}, proporcionando resultados sobresalientes y un rendimiento superior en comparación con los otros modelos.

\begin{table}[H]
\centering
\begin{tabular}{|l|c|c|c|c|}
\hline
\textbf{Modelo} & \textbf{Precisión} & \textbf{Sensibilidad} & \textbf{Especificidad} & \textbf{F1 Score} \\ \hline
DNN & 0.999 & 0.999 & 0.999 & 0.999 \\ \hline
RNN & 0.998 & 0.999 & 0.998 & 0.998 \\ \hline
Autoencoder & 0.999 & 0.93 & 0.999 & 0.991 \\ \hline
CNN & 0.999 & 0.998 & 0.999 & 0.999 \\ \hline
\end{tabular}
\caption{Métricas de rendimiento de los diferentes modelos.}
\label{tab:conclusionModels}
\end{table}






\begin{comment}
Improving Performance of Autoencoder-Based
Network Anomaly Detection on NSL-KDD Dataset

Malware traffic classification using convolutional neural network for representation learning

Deep Learning-based Feature Selection for Intrusion Detection
System in Transport Layer 1)


    Para implementar el proceso descrito, puedes usar un Autoencoder (AE), que es un tipo de red neuronal utilizada para aprender representaciones eficientes (codificaciones) de datos, típicamente para la reducción de dimensionalidad.

Explicación del Proceso:
Fase de Entrenamiento:

Extracción y Reducción de Características: Las características originales del tráfico de red se extraen y se reducen mediante la operación de codificación del Autoencoder. Esto significa que los datos se pasan a través de la parte de codificación del AE para obtener una representación en un espacio latente.
Reconstrucción del Tráfico: El espacio latente se usa para reconstruir la salida (decodificación).
Cálculo del Error de Reconstrucción: La diferencia entre la muestra de tráfico original y la reconstruida se calcula para obtener el error de reconstrucción.
Determinación del Umbral: Se calcula el valor máximo de todos los errores de reconstrucción como el umbral para identificar anomalías.
Fase de Prueba:

Cálculo del Error de Reconstrucción: Las muestras de tráfico de red se ingresan al modelo AE entrenado y se calcula el error de reconstrucción, ahora llamado "puntaje de anomalía".
Comparación con el Umbral: Si el puntaje de anomalía es mayor que el umbral determinado durante la fase de entrenamiento, la muestra de tráfico se considera anómala.
Código en Python
Aquí tienes un ejemplo de cómo implementar este proceso en Python utilizando Keras y TensorFlow:

python
Copiar código
import numpy as np
from tensorflow.keras.models import Model
from tensorflow.keras.layers import Input, Dense
from tensorflow.keras import backend as K

# Fase de Entrenamiento
def crear_autoencoder(input_dim, encoding_dim):
    # Definición del Autoencoder
    input_layer = Input(shape=(input_dim,))
    encoder = Dense(encoding_dim, activation="relu")(input_layer)
    decoder = Dense(input_dim, activation="sigmoid")(encoder)
    autoencoder = Model(inputs=input_layer, outputs=decoder)
    autoencoder.compile(optimizer="adam", loss="mean_squared_error")
    return autoencoder

def entrenar_autoencoder(autoencoder, datos):
    # Entrenar el Autoencoder
    autoencoder.fit(datos, datos, epochs=50, batch_size=256, shuffle=True, validation_split=0.2)
    return autoencoder

def calcular_umbral(autoencoder, datos):
    # Calcular errores de reconstrucción
    reconstrucciones = autoencoder.predict(datos)
    errores = np.mean(np.power(datos - reconstrucciones, 2), axis=1)
    umbral = np.max(errores)
    return umbral

# Fase de Prueba
def detectar_anomalias(autoencoder, datos, umbral):
    # Calcular errores de reconstrucción
    reconstrucciones = autoencoder.predict(datos)
    errores = np.mean(np.power(datos - reconstrucciones, 2), axis=1)
    anomalias = errores > umbral
    return anomalias, errores

# Ejemplo de uso
input_dim = 30  # Supongamos que tenemos 30 características de tráfico de red
encoding_dim = 14  # Dimensión del espacio latente
datos_entrenamiento = np.random.rand(10000, input_dim)  # Datos de entrenamiento (ejemplo)

# Crear y entrenar el Autoencoder
autoencoder = crear_autoencoder(input_dim, encoding_dim)
autoencoder = entrenar_autoencoder(autoencoder, datos_entrenamiento)

# Calcular el umbral de errores de reconstrucción
umbral = calcular_umbral(autoencoder, datos_entrenamiento)

# Datos de prueba
datos_prueba = np.random.rand(1000, input_dim)

# Detectar anomalías en los datos de prueba
anomalias, puntajes_anomalia = detectar_anomalias(autoencoder, datos_prueba, umbral)

print("Umbral:", umbral)
print("Anomalías detectadas:", np.sum(anomalias))
Explicación del Código:
Crear el Autoencoder:

Definimos la arquitectura del Autoencoder con una capa de entrada, una capa de codificación y una capa de decodificación.
Entrenar el Autoencoder:

Entrenamos el Autoencoder utilizando los datos de entrenamiento. Aquí usamos un conjunto de datos de ejemplo generado aleatoriamente.
Calcular el Umbral:

Después de entrenar el Autoencoder, calculamos los errores de reconstrucción en los datos de entrenamiento y tomamos el valor máximo de estos errores como el umbral para la detección de anomalías.
Detectar Anomalías:

En la fase de prueba, calculamos los errores de reconstrucción para las nuevas muestras de tráfico de red y los comparamos con el umbral. Si el error de reconstrucción de una muestra es mayor que el umbral, se considera anómala.
Este enfoque proporciona un marco básico para la detección de anomalías en el tráfico de red utilizando un Autoencoder. Puedes ajustar los hiperparámetros, como el tamaño del espacio latente y el número de épocas, según tus necesidades y la naturaleza de tus datos.

entrenamos las muestras benignas o malignas?
ChatGPT
En el contexto de la detección de anomalías utilizando un Autoencoder (AE), generalmente entrenas el modelo con muestras benignas (normales). La idea es que el Autoencoder aprende a reconstruir bien las muestras normales, pero tendrá un mayor error de reconstrucción para las muestras anómalas (malignas) que no ha visto durante el entrenamiento.

Razones para Entrenar con Muestras Benignas:
Generalización: El Autoencoder se entrena para capturar la estructura de las muestras benignas. Las anomalías, al no seguir esta estructura, resultarán en mayores errores de reconstrucción.
Detección Eficaz de Anomalías: Un mayor error de reconstrucción sugiere que la muestra es diferente de lo que el modelo ha aprendido como normal, lo cual es indicativo de una anomalía.
Escasez de Datos Anómalos: En muchos casos, las muestras anómalas son escasas y pueden no estar disponibles en cantidad suficiente para un entrenamiento robusto.
Modificación del Código para Asegurarse de Entrenar Solo con Muestras Benignas
Aquí se asume que datos_entrenamiento contiene solo muestras benignas.

python
Copiar código
import numpy as np
from tensorflow.keras.models import Model
from tensorflow.keras.layers import Input, Dense

# Funciones para crear y entrenar el Autoencoder
def crear_autoencoder(input_dim, encoding_dim):
    input_layer = Input(shape=(input_dim,))
    encoder = Dense(encoding_dim, activation="relu")(input_layer)
    decoder = Dense(input_dim, activation="sigmoid")(encoder)
    autoencoder = Model(inputs=input_layer, outputs=decoder)
    autoencoder.compile(optimizer="adam", loss="mean_squared_error")
    return autoencoder

def entrenar_autoencoder(autoencoder, datos_benignos):
    autoencoder.fit(datos_benignos, datos_benignos, epochs=50, batch_size=256, shuffle=True, validation_split=0.2)
    return autoencoder

def calcular_umbral(autoencoder, datos_benignos):
    reconstrucciones = autoencoder.predict(datos_benignos)
    errores = np.mean(np.power(datos_benignos - reconstrucciones, 2), axis=1)
    umbral = np.max(errores)
    return umbral

# Función para detectar anomalías
def detectar_anomalias(autoencoder, datos, umbral):
    reconstrucciones = autoencoder.predict(datos)
    errores = np.mean(np.power(datos - reconstrucciones, 2), axis=1)
    anomalias = errores > umbral
    return anomalias, errores

# Ejemplo de uso
input_dim = 30  # Número de características de tráfico de red
encoding_dim = 14  # Dimensión del espacio latente

# Datos de entrenamiento (muestras benignas)
datos_entrenamiento_benignos = np.random.rand(10000, input_dim)  # Reemplaza esto con tus datos reales

# Crear y entrenar el Autoencoder con muestras benignas
autoencoder = crear_autoencoder(input_dim, encoding_dim)
autoencoder = entrenar_autoencoder(autoencoder, datos_entrenamiento_benignos)

# Calcular el umbral de errores de reconstrucción
umbral = calcular_umbral(autoencoder, datos_entrenamiento_benignos)

# Datos de prueba (pueden contener tanto benignos como malignos)
datos_prueba = np.random.rand(1000, input_dim)  # Reemplaza esto con tus datos reales

# Detectar anomalías en los datos de prueba
anomalias, puntajes_anomalia = detectar_anomalias(autoencoder, datos_prueba, umbral)

print("Umbral:", umbral)
print("Anomalías detectadas:", np.sum(anomalias))
Pasos en el Código:
Entrenamiento del Autoencoder con Muestras Benignas:

Asegúrate de que datos_entrenamiento_benignos contenga solo muestras de tráfico de red benignas.
El Autoencoder se entrena para reconstruir estas muestras benignas.
Cálculo del Umbral:

Calcula los errores de reconstrucción en las muestras benignas y establece el umbral como el valor máximo de estos errores.
Detección de Anomalías:

En la fase de prueba, cualquier muestra cuyo error de reconstrucción sea mayor que el umbral se considera anómala.
De esta manera, entrenar solo con muestras benignas permite al Autoencoder detectar efectivamente las anomalías basándose en la desviación de las características normales aprendidas.

como calcular el umbral?
ChatGPT
Calcular el umbral en el contexto de un Autoencoder (AE) para la detección de anomalías implica determinar un valor de referencia que diferencie entre las muestras normales y anómalas. El proceso es el siguiente:

Entrenar el Autoencoder con Muestras Benignas:

Entrena el AE solo con datos benignos (normales).
Calcular los Errores de Reconstrucción en los Datos de Entrenamiento:

Después del entrenamiento, pasa las mismas muestras benignas a través del AE para obtener las reconstrucciones.
Calcula el error de reconstrucción para cada muestra. Un error de reconstrucción comúnmente usado es el error cuadrático medio (MSE).
Determinar el Umbral:

Utiliza las estadísticas de los errores de reconstrucción para determinar un umbral. Una estrategia común es utilizar el valor máximo del error de reconstrucción de las muestras benignas, pero otras estrategias incluyen usar un percentil alto (por ejemplo, 95%) para tener una tolerancia a ciertos errores benignos.
Aquí tienes un código que muestra este proceso:

python
Copiar código
import numpy as np
from tensorflow.keras.models import Model
from tensorflow.keras.layers import Input, Dense
from sklearn.model_selection import train_test_split

# Función para crear el Autoencoder
def crear_autoencoder(input_dim, encoding_dim):
    input_layer = Input(shape=(input_dim,))
    encoder = Dense(encoding_dim, activation="relu")(input_layer)
    decoder = Dense(input_dim, activation="sigmoid")(encoder)
    autoencoder = Model(inputs=input_layer, outputs=decoder)
    autoencoder.compile(optimizer="adam", loss="mean_squared_error")
    return autoencoder

# Función para entrenar el Autoencoder
def entrenar_autoencoder(autoencoder, datos_benignos):
    autoencoder.fit(datos_benignos, datos_benignos, epochs=50, batch_size=256, shuffle=True, validation_split=0.2)
    return autoencoder

# Función para calcular el umbral
def calcular_umbral(autoencoder, datos_benignos):
    # Obtener reconstrucciones del Autoencoder
    reconstrucciones = autoencoder.predict(datos_benignos)
    # Calcular errores de reconstrucción
    errores = np.mean(np.power(datos_benignos - reconstrucciones, 2), axis=1)
    # Calcular el umbral como el valor máximo del error de reconstrucción
    umbral = np.max(errores)
    return umbral

# Función para detectar anomalías
def detectar_anomalias(autoencoder, datos, umbral):
    reconstrucciones = autoencoder.predict(datos)
    errores = np.mean(np.power(datos - reconstrucciones, 2), axis=1)
    anomalias = errores > umbral
    return anomalias, errores

# Ejemplo de uso
input_dim = 30  # Número de características de tráfico de red
encoding_dim = 14  # Dimensión del espacio latente

# Datos de entrenamiento (muestras benignas)
datos_entrenamiento_benignos = np.random.rand(10000, input_dim)  # Reemplaza esto con tus datos reales

# Crear y entrenar el Autoencoder con muestras benignas
autoencoder = crear_autoencoder(input_dim, encoding_dim)
autoencoder = entrenar_autoencoder(autoencoder, datos_entrenamiento_benignos)

# Calcular el umbral de errores de reconstrucción
umbral = calcular_umbral(autoencoder, datos_entrenamiento_benignos)

# Datos de prueba (pueden contener tanto benignos como malignos)
datos_prueba = np.random.rand(1000, input_dim)  # Reemplaza esto con tus datos reales

# Detectar anomalías en los datos de prueba
anomalias, puntajes_anomalia = detectar_anomalias(autoencoder, datos_prueba, umbral)

print("Umbral:", umbral)
print("Anomalías detectadas:", np.sum(anomalias))
Explicación del Código:
Entrenamiento del Autoencoder con Muestras Benignas:

La función crear_autoencoder define la arquitectura del Autoencoder.
La función entrenar_autoencoder entrena el Autoencoder usando solo muestras benignas.
Cálculo del Umbral:

La función calcular_umbral pasa las muestras benignas a través del AE para obtener las reconstrucciones.
Calcula los errores de reconstrucción usando el error cuadrático medio (MSE) y determina el umbral como el valor máximo de estos errores.
Detección de Anomalías:

La función detectar_anomalias calcula los errores de reconstrucción para los datos de prueba y marca como anomalías aquellas muestras cuyo error supera el umbral.
Estrategias Alternativas para Determinar el Umbral:
Percentil: En lugar de usar el valor máximo, puedes usar un percentil alto (por ejemplo, 95%). Esto puede ser útil si hay algunas muestras benignas que son atípicas pero no anómalas.

python
Copiar código
def calcular_umbral(autoencoder, datos_benignos, percentil=95):
    reconstrucciones = autoencoder.predict(datos_benignos)
    errores = np.mean(np.power(datos_benignos - reconstrucciones, 2), axis=1)
    umbral = np.percentile(errores, percentil)
    return umbral
Media + Desviación Estándar: Usa la media de los errores de reconstrucción más una cierta cantidad de desviaciones estándar.

python
Copiar código
def calcular_umbral(autoencoder, datos_benignos, num_std_dev=3):
    reconstrucciones = autoencoder.predict(datos_benignos)
    errores = np.mean(np.power(datos_benignos - reconstrucciones, 2), axis=1)
    media = np.mean(errores)
    desviacion_estandar = np.std(errores)
    umbral = media + num_std_dev * desviacion_estandar
    return umbral
Seleccionar el método adecuado para calcular el umbral dependerá de tus datos específicos y de tus necesidades de detección de anomalías.
\end{comment}


\begin{comment}

Para clasificar los datos del dataset KDD 1999 usando las \gls{cnn} vamos a seguir los siguientes artículos \citep{kim2020cnn, yang2006anomaly, nguyen2018design, kim2018encoding}. Prácticamente todo el cuerpo del experimento se encuentra en el artículo \citep{kim2020cnn}, pero en el artículo \citep{kim2018encoding} aparece la parte de normalización de los datos y algunos hiperparametros de inicio.



\subsection{Procesamiento de Datos}

El conjunto de datos se somete a un preprocesamiento que incluye la estandarización y la codificación de etiquetas. Las características originales del conjunto de datos, que comprenden 41 dimensiones, se redimensionan a un vector de 122 dimensiones como vimos anteriormente. Además, se reduce su valor para encontrarse en valores dentro del rango $[0, 1]$. Este paso es crucial para asegurar que los datos sean adecuados para el entrenamiento del modelo LSTM, evitando la influencia desproporcionada de algunas características. 

Adicionalmente, las etiquetas de clasificación se convierten en categorías utilizando la función \lstinline|to_categorical| de Keras, transformando las etiquetas en una representación de una matriz de tamaño igual al número de clases (en este caso, 5 clases). Esta codificación es esencial para la aplicación de la función de activación \textit{softmax} en la capa de salida del modelo, que requiere que las etiquetas sean categóricas.

Para entrenar el modelo LSTM, los datos se estructuran en secuencias temporales. Se emplean ventanas deslizantes con un tamaño de 100 \textit{timesteps} para generar secuencias temporales de características, resultando en un tensor de entrada de forma \((\text{número de muestras}, 100, 41)\). Este formato es adecuado para el modelo LSTM, que está diseñado para procesar datos secuenciales.



\subsection{Arquitectura del Modelo}

La arquitectura del modelo LSTM consiste en una única capa LSTM seguida de una capa densa para la clasificación final. La capa LSTM contiene 80 unidades ocultas, seleccionadas en base a experimentación en el artículo \citep{yin2017deep}, para equilibrar la capacidad de modelar la complejidad temporal y la eficiencia computacional. Esta capa procesa secuencias de entrada de longitud 100 \textit{timesteps} y 122 características cada una, aprendiendo las dependencias temporales en los datos.

La capa LSTM se conecta a una capa densa con tantas neuronas como clases objetivo (5 en este caso) y utiliza la función de activación \textit{softmax}. Esta capa transforma las salidas de la LSTM en probabilidades de clasificación para cada clase, permitiendo al modelo realizar predicciones categóricas sobre el tipo de intrusión.


El modelo LSTM se optimiza utilizando el optimizador \textit{Stochastic Gradient Descent} (SGD), configurado con una tasa de aprendizaje de 0.01. Este optimizador se selecciona por su capacidad para proporcionar un control preciso sobre el proceso de ajuste de parámetros. La función de pérdida empleada es el error cuadrático medio (\textit{mean squared error}), que se utiliza aquí debido a la naturaleza continua de las salidas del modelo antes de la aplicación de \textit{softmax}.

El modelo se entrena con un tamaño de lote de 50, lo que permite una actualización eficiente de los gradientes mientras se maneja una cantidad razonable de datos en cada iteración. Se ejecutan 500 épocas de entrenamiento, suficientes para asegurar la convergencia del modelo sin llegar a sobreajustarse. 


La partición de los datos en conjuntos de entrenamiento y prueba se realiza de manera que el 75\% de los datos se utilizan para entrenamiento, el 15\% para prueba y el 10\% restante para validación, permitiendo una evaluación adecuada del modelo. Se implementa el \textit{EarlyStopping} para detener el entrenamiento si la precisión en el conjunto de validación no mejora después de un número específico de épocas, ayudando a prevenir el sobreajuste. Además, se emplea un \textit{CSVLogger} para registrar las métricas de entrenamiento y validación en cada época, facilitando un análisis detallado del rendimiento del modelo.

\subsection{Proceso de Entrenamiento}

Durante el entrenamiento, se monitorean tanto la precisión como la pérdida en el conjunto de validación, ajustando los parámetros del modelo para mejorar su rendimiento. Este proceso iterativo permite al modelo aprender las dependencias temporales en los datos de intrusión, mejorando su capacidad para clasificar correctamente nuevas instancias de intrusión en el conjunto de prueba.

\begin{figure}[h]
    \centering
    \includegraphics[width=0.8\textwidth]{img/KDD_RNN.csv.png}
    \caption{Evolución de la Precisión y la Pérdida durante el Entrenamiento del Modelo LSTM.}
    \label{fig:lstm_training_process}
\end{figure}

\subsection{Evaluación del Modelo}

El modelo LSTM se evalúa utilizando el conjunto de datos de prueba, logrando una alta precisión en la detección de diferentes tipos de intrusiones. La matriz de confusión, presentada en la Tabla~\ref{tab:lstm_confusion_matrix}, detalla el rendimiento del modelo en términos de instancias correctamente clasificadas y errores de clasificación en cada clase.

\begin{table}[h]
    \centering
    \begin{tabular}{|c|c|c|c|c|c|}
        \hline
        \textbf{Clase Real} & \textbf{DoS} & \textbf{Probe} & \textbf{R2L} & \textbf{U2R} & \textbf{Normal} \\
        \hline
        \textbf{DoS} & 1,165,359 & 1 & 0 & 0 & 13 \\
        \hline
        \textbf{Probe} & 5 & 12,293 & 1 & 0 & 100 \\
        \hline
        \textbf{R2L} & 1 & 0 & 269 & 0 & 79 \\
        \hline
        \textbf{U2R} & 0 & 0 & 0 & 0 & 9 \\
        \hline
        \textbf{Normal} & 42 & 7 & 50 & 0 & 291,391 \\
        \hline
    \end{tabular}
    \caption{Matriz de Confusión para la Clasificación Multiclase en el Conjunto de Prueba del Modelo LSTM.}
    \label{tab:lstm_confusion_matrix}
\end{table}

\subsection{Conclusiones}

La arquitectura RNN-LSTM propuesta ha demostrado ser eficaz en la detección de intrusiones en redes. La capacidad de las LSTM para capturar dependencias temporales complejas en los datos secuenciales permite una clasificación precisa y robusta de diferentes tipos de intrusiones. El enfoque basado en secuencias temporales mejora la detección de patrones de intrusión recurrentes y la capacidad del modelo para generalizar a nuevas situaciones.







En el articulo \citep{elmasry2019empirical} están los experimentos con DNN, RNN,  RBM que puedo tomar también como referencia porque está muy bien explicado las capas e hiperparámetros que utiliza.



Para evaluar el rendimiento del autoencoder convolucional en el problema de Microsoft Malware Clasification, se ha generado una matriz de confusión con los datos de prueba. 


\begin{figure}[H]
    \centering
    \includegraphics[width=0.6\textwidth]{img/confusionMatrixCNN_KDD.png}
    \caption{Matriz de confusión del modelo utilizando una CNN.}
    \label{fig: cnnKDDConfusion}
\end{figure}


En ella se puede observar como la mayoría de las predicciones están bien clasificadas al igual que para la \acrshort{cnn}, pero la clase con etiqueta 5 obtiene un 33\% de éxito también.
 
Además en la tabla \ref{tab:metrics}, se puede observar una alta precisión y sensibilidad en la mayoría de las clases, con F1-scores que oscilan entre 0.89 y 0.98 para las clases 1, 2, 3, 4, 6, 7, 8 y 9, lo que indica un gran equilibrio entre la precisión y la sensibilidad. La clase 3 sobresale con un F1-score de 0.98, reflejando una excelente precisión (0.97) y sensibilidad (0.99). Sin embargo, la clase 5 presenta un rendimiento significativamente inferior, con una precisión de 0.49, una sensibilidad de 0.33 y un F1-score de 0.39, señalando que el modelo tiene dificultades en identificar correctamente los ejemplos de esta clase. 


\begin{table}[H]
\centering
\begin{tabular}{|c|c|c|c|c|}
\hline
Clase & Precisión & TPR (Sensibilidad) & FPR & F1 Score \\ \hline
1 & 0.94 & 0.97 & 0.06 & 0.96 \\ \hline
\end{tabular}
\caption{Métricas de rendimiento de la CNN.}
\label{tab:kddCNN}
\end{table}

Este problema, al igual que en el caso de la \acrshort{cnn}, se podría resolver usando pesos de clase. 



\end{comment}

\chapter{Conclusiones y Trabajo Futuro} \label{Capítulo 5}



\section{Conclusiones} \label{Subsec: 5_1}



\section{Trabajo futuro} \label{Subsec: 5_2}


\begin{comment}

\section{Cita bibliografias}

%%Comentar estas notas para que no salgan en la memoria
{\Large\textbf{Nota: el título extendido (si procede), el resumen y el abstract deben estar en una misma página y su extensión no debe superar una página. Tamaño mínimo 11pto.}}
\vspace{1cm}

{\Large\textbf{Extensión máxima 20 páginas sin contar portada ni resumen (sí se incluye índice, introducción, conclusiones y bibliografía}}
\newpage

%%Inicio:
\citep{olabe1998redes} : ESPAÑOL : Muy completo, me gusta
\citep{parisi2019hands} : INGLES información práctica sobre cómo utilizar sistemas de inteligencia artificial para prevenir ataques cibernéticos, detectar amenazas y anomalías en las redes, y fortalecer la seguridad en el ámbito de la ciberseguridad

\subsection{Microsoft}

10800 archivos de prueba de tipo .bytes (otros 10800 de tipo .asm) pero solo usaremos los .bytes. Convertimos cada archivo en una imagen. Primero, cada código hexadecimal lo convertimos a numeros decimales y estos los pasamos a un array de numpy. Hacemos reshape (lado,lado2) de forma que obtengamos la mayor dimension posible que sea casi cuadrado consiguiendo perder la menor información posible. Este reshape lo pasamos a np.uint8 y finalmente lo interpolamos usando bilinal, cubic, bicubic, nearest y observamos cual es la mejor de todas {zhao2023new} (hacer una grafica o algo para comprobarlo). Además, para caragr los datos he usado multiprocessing con los diferentes datos de tiempo. En el caso de algunas imagenes, los archivos .bytes no contienen ningun tipo de informacion(?? ?? ?? ..) todos los byte son ??. Estos ficheros, no estan incluidos en las imagenes ni de entrenamiento ni validación porque no contienen ningun tpo de información. 

Mi modelo es una CNN secuencial. Tengo que generalizarlo por si quiero insertar la layer de 1000 o las de dropout y con que valor(param de entrada). AL igual que el tipo de intepolación. Tener en cuenta que he visto overfitting porque loss y val-loss tienen mucha diferencia por eso aplico dropout.



archivos con todo '??':
da3XhOZzQEbKVtLgMYWv.bytes
a9oIzfw03ED4lTBCt52Y.bytes
fRLS3aKkijp4GH0Ds6Pv.bytes
cf4nzsoCmudt1kwleOTI.bytes
58kxhXouHzFd4g3rmInB.bytes
6tfw0xSL2FNHOCJBdlaA.bytes
IidxQvXrlBkWPZAfcqKT.bytes
d0iHC6ANYGon7myPFzBe.bytes

\citep{narayanan2016performance} : habla sobre los caracteres '??'



\subsection{Páginas web}


\citep{kaggle_cnn_tutorial} : Un tutorial sobre una CNN en python

\citep{conceptos_RN_DesdeCero} : Primera red neuronal que solo hace forwardpropagation


\subsection{Redes neuronales}
\citep{matich2001redes} ESPAÑOL 

\citep{stamp2022artificial} : INLES habla de redes neuronales y ciberseguridad

\citep{tang2007neural} : INGLES redes neuronales



\citep{mirjalili2020python} : Tiene una lectura muy facil y conceptos bastante bien expresados y explicados. Es un libro muy práctico con bastantes ejemplos en python usando Scikit-learn y TensorFlow.



cuando usamos reshape, dimensiones tienen que coincidir. Sale en https://numpy.org/doc/stable/reference/generated/numpy.reshape.html
\citep{ni2018malware} : De donde sale que usamos interpolacion para cargar las imagenes

\citep{zhao2023new} : De donde sale que usamos interpolacion para cargar las imagenes

\citep{hinton2012improving} : porque dropout in full connected layers

\citep{zhao2023new} : diferentes tipos de interpolacion

\citep{he2019malware}: interpolation bilineal

\citep{gibert2019using} : resultados de microsoft big data

\citep{ni2018malware} : resultados Microsoft data

\citep{sharma2024migan} : Microsoft malware

\citep{kornish2018malware} : crear las imagenes de microsoft

\citep{luo2017binary} : conversion malware to image 

\citep{simonyan2014very} : random cropped for large datset


\citep{habibi2023performance} : dicen que adam, adammax y tal son mejores optimizadores

\subsection{Ciberseguridad}

MCFT-CNN: Malware classification with fine-tune convolution neural
networks using traditional and transfer learning in Internet of Things IMPORTANTE EN LA PAGINA 5 HAY IMAGEN DONDE PONE RESIZE AND RESHAPE

\cite{yagual2022revision} : ESPAÑOL, Analiza cuatro técnicas populares de redes neuronales y de aprendizaje profundo, en el contexto de la ciberseguridad : perceptrón multicapa (MLP), las redes neuronales convolucionales (CNN), las redes neuronales recurrentes (RNN) y el aprendizaje profundo por transferencia (DTL). Estas técnicas y sus modelos de seguridad híbridos pueden utilizarse para abordar de forma inteligente diferentes problemas de ciberdelitos, como la detección de intrusiones, el análisis de malware, el análisis de amenazas de seguridad, la predicción de ciberataques o anomalías, etc. 
\begin{enumerate}
\item Algoritmos NN
\item Que algoritmo de cada NN se aplica a cada problema de ciberseguridad. Nada explicativo.
\end{enumerate}

\citep{omar2022machine} : INGLES pero es libro que tiene muy buena pinta 50 pags

\citep{sarker2020cybersecurity} INGLES

\citep{xin2018machine} INGLES : La introducción destaca la creciente integración de Internet en la vida social y cómo esto está cambiando la forma en que las personas aprenden y trabajan, pero también expone a amenazas de seguridad cada vez más graves. El artículo se centra en la importancia de identificar diversos ataques informáticos, en particular, los no vistos previamente. Se menciona que la ciberseguridad es un conjunto de tecnologías y procesos diseñados para proteger computadoras, redes, programas y datos de ataques y accesos no autorizados.

Se explica que los sistemas de seguridad de red constan de un sistema de seguridad de red y un sistema de seguridad informática, y que cada uno de estos sistemas incluye cortafuegos, software antivirus y sistemas de detección de intrusiones. Se detalla que los sistemas de detección de intrusiones (IDS) ayudan a identificar comportamientos no autorizados en el sistema.

Se mencionan tres tipos principales de análisis de redes para los IDS: basado en el uso (conocido como basado en firmas), basado en anomalías y híbrido. Se destaca que las técnicas basadas en anomalías son atractivas porque pueden detectar ataques no conocidos y se adaptan a actividades normales personalizadas para cada sistema, lo que dificulta a los atacantes saber qué actividades pueden realizar sin ser detectados.

El artículo describe que las técnicas híbridas combinan la detección de uso y la detección de anomalías y se utilizan para aumentar la detección de intrusiones conocidas y reducir las tasas de falsos positivos de ataques desconocidos. Se menciona que la mayoría de los métodos de aprendizaje automático (ML) y aprendizaje profundo (DL) son híbridos.

Luego, se presenta una revisión de la literatura sobre métodos de ML y DL para aplicaciones de ciberseguridad, con un enfoque en tecnologías de seguridad de red y sus descripciones. El artículo se centra en proporcionar información detallada sobre los métodos de ML y DL, así como referencias a trabajos fundamentales en cada método.

Se señala que el artículo no describe todas las técnicas de detección de anomalías de red, sino que se concentra en los métodos de ML y DL. También se mencionan otros tipos de métodos de detección, como los basados en firmas y los híbridos.

Finalmente, se explica que el artículo está organizado en secciones que abordan similitudes y diferencias entre ML y DL, conjuntos de datos de ciberseguridad utilizados en ML y DL, descripciones de métodos y trabajos relacionados en ciberseguridad, estado actual de la investigación y direcciones futuras, y conclusiones.

En resumen, el artículo se centra en la importancia de la ciberseguridad en un mundo cada vez más conectado, describe diferentes enfoques de detección de intrusiones y proporciona una revisión de la literatura sobre métodos de ML y DL en ciberseguridad.


El texto discute diversas aproximaciones y métodos para la detección de intrusiones en redes utilizando diferentes técnicas de aprendizaje automático. Se centra en el uso de conjuntos de datos como los Conjuntos de Datos de Detección de Intrusiones DARPA, el Conjunto de Datos KDD Cup 99, el Conjunto de Datos NSL-KDD y el Conjunto de Datos ADFA para entrenar y probar modelos de detección de intrusiones. Se exploran los siguientes algoritmos de aprendizaje automático:

Máquina de Vectores de Soporte (SVM): Se utiliza SVM para clasificar diferentes tipos de ataques de red, y su rendimiento se evalúa en términos de precisión y tasa de falsas alarmas. Se han realizado varios estudios para optimizar SVM en la detección de intrusiones, con resultados que muestran tasas de detección prometedoras pero variaciones en las falsas alarmas.

K-Vecinos Más Cercanos (kNN): Se emplea kNN para clasificar el tráfico de red en función de su similitud con patrones de ataque conocidos. Los estudios han demostrado que kNN puede ser eficaz en la detección de intrusiones, especialmente cuando se combina con otras técnicas. Ayuda a reducir las falsas alarmas y mejorar las tasas de detección.

Árbol de Decisión: Se utilizan árboles de decisión para la detección de intrusiones, y se exploran diferentes enfoques para mejorar su rendimiento. Se aplican técnicas de selección de características, poda de árboles y optimización para mejorar la precisión y reducir las falsas alarmas. Los árboles de decisión pueden proporcionar alta precisión y son particularmente efectivos para detectar ataques conocidos.

En resumen, el texto destaca la importancia de elegir el algoritmo de aprendizaje automático adecuado y el conjunto de datos para la detección de intrusiones. Diferentes algoritmos tienen sus fortalezas y debilidades, y su rendimiento puede variar según los tipos específicos de ataques y conjuntos de datos utilizados. Los investigadores trabajan continuamente para mejorar la precisión y reducir las falsas alarmas en los sistemas de detección de intrusiones en redes.


 El texto proporcionado es un fragmento de un artículo de investigación que discute la aplicación de métodos de Aprendizaje Automático (ML) y Aprendizaje Profundo (DL) en el campo de la seguridad de redes, centrándose específicamente en la detección de intrusiones. Ofrece una visión general de diferentes técnicas de ML y DL, conjuntos de datos y tendencias en la investigación de detección de intrusiones.

El texto menciona varios métodos de ML y DL, como las Redes de Creencias Profundas (DBN), las Redes Neuronales Recurrentes (RNN) y las Redes Neuronales Convolucionales (CNN). Estos métodos se aplican para detectar y clasificar intrusiones y anomalías en redes. El fragmento también aborda el uso de conjuntos de datos de referencia y los desafíos relacionados con la calidad de los datos, los datos desequilibrados y la necesidad de conjuntos de datos más extensos y equilibrados.

El artículo destaca la tendencia de combinar diferentes algoritmos en modelos híbridos, lo que puede mejorar el rendimiento de la detección. También enfatiza la importancia de mejorar la velocidad de detección de intrusiones y el potencial del aprendizaje en línea para adaptarse a amenazas en constante evolución.

En resumen, este texto ofrece una visión general del estado de la investigación en la detección de intrusiones utilizando métodos de ML y DL, así como algunos de los desafíos y direcciones futuras en este campo. Si tienes alguna pregunta específica o necesitas más información sobre alguna parte de este texto, no dudes en preguntar.


\citep{martinez2019machine} Ingles


\citep{sarker2022machine} Ingles


\citep{stamp2022artificial} : INLES Redes neuronales y ciberseguridad

\citep{haber2017data} : INGLES muchos articulos, filtrar interesnates ciberseguridad

\citep{kim2018network} : INGLES Cubre temas relacionados con la detección de intrusiones y la aplicación de modelos de aprendizaje profundo en este campo, lo que es altamente relevante para la ciberseguridad.

La serie tiene como objetivo desarrollar y difundir una comprensión de innovaciones, paradigmas, técnicas y tecnologías en el contexto de la investigación y estudios relacionados con sistemas y redes de ciberseguridad. Publica revisiones exhaustivas y cohesivas de temas de vanguardia en ciberseguridad, así como técnicas sofisticadas, presentaciones de investigaciones originales y estudios de casos en profundidad en sistemas y redes cibernéticas. La serie también proporciona un único punto de cobertura de temas avanzados y emergentes oportunos, así como un foro para conceptos fundamentales que pueden no haber alcanzado un nivel de madurez para justificar un libro de texto integral. Se abordan problemas de seguridad, privacidad, disponibilidad y confiabilidad para sistemas y redes cibernéticas, y se dan la bienvenida a tecnologías emergentes, como inteligencia artificial, computación en la nube, sistemas ciberfísicos y análisis de grandes datos relacionados con la investigación en ciberseguridad. Se centra principalmente en los siguientes temas de investigación:
Fundamentos y Teorías
• Criptografía para la ciberseguridad
• Teorías de la ciberseguridad
• Seguridad demostrable

Sistemas y Redes Cibernéticas
• Seguridad de sistemas cibernéticos
• Seguridad de redes
• Servicios de seguridad
• Seguridad y privacidad de redes sociales
• Ataques y defensa cibernéticos
• Ciberseguridad basada en datos
• Computación y sistemas de confianza

Aplicaciones y Otros
• Seguridad de hardware y dispositivos
• Seguridad de aplicaciones cibernéticas
• Aspectos humanos y sociales de la ciberseguridad





Este monográfico presenta avances recientes en el Sistema de Detección de Intrusiones (IDS) utilizando modelos de aprendizaje profundo, que han logrado un gran éxito recientemente, especialmente en el campo de la visión por computadora, el procesamiento del lenguaje natural y el procesamiento de imágenes. El monográfico ofrece una visión sistemática y metódica de los últimos desarrollos en aprendizaje profundo y realiza una comparación entre IDS basados en aprendizaje profundo. También ofrece una descripción general completa de las aplicaciones del aprendizaje profundo en IDS, seguida de métodos de aprendizaje de características profundas que incluyen una novedosa extracción y selección de características profundas, y el aprendizaje profundo para el agrupamiento.
Está destinado a estudiantes, investigadores y profesionales interesados en el aprendizaje profundo y la detección de intrusiones, y puede utilizarse como libro de referencia. La comparación exhaustiva de varias aplicaciones de aprendizaje profundo ayuda a los lectores con una comprensión básica del aprendizaje automático e inspira aplicaciones en IDS y otras áreas de ciberseguridad.
El monográfico consta de varios capítulos, donde el Capítulo 1 destaca la importancia de los IDS en las redes de computadoras en la actualidad, proporcionando una encuesta de violaciones de seguridad en redes de computadoras. Se resalta que los modelos de aprendizaje profundo pueden mejorar el rendimiento de los IDS y se explica la motivación para examinar los IDS basados en aprendizaje profundo. El Capítulo 2 define IDS y explica los diferentes tipos de IDS actuales, así como las métricas de rendimiento comunes y conjuntos de datos de referencia públicos disponibles.
El Capítulo 3 realiza un estudio preliminar breve sobre el aprendizaje automático clásico, que abarca enfoques supervisados, no supervisados, semi-supervisados, débilmente supervisados, de refuerzo y adversariales. También se revisan brevemente 22 artículos que utilizan técnicas de aprendizaje automático en sus IDS. El Capítulo 4 se centra en diversos modelos de aprendizaje profundo que incluyen enfoques generativos, discriminatorios y híbridos.
El Capítulo 5 realiza un análisis de varios IDS que aprovechan modelos de aprendizaje profundo, divididos en cuatro clases: generativos, discriminativos, híbridos y aprendizaje profundo por refuerzo.
El Capítulo 6 aborda la importancia de los modelos de aprendizaje profundo como enfoque de aprendizaje de características (FL) en investigaciones de IDS. También se explican dos modelos adicionales: la extracción y selección de características profundas y el aprendizaje profundo para el agrupamiento.
El Capítulo 7 concluye este monográfico proporcionando una visión general de los desafíos y las futuras direcciones de investigación en las aplicaciones de aprendizaje profundo para IDS.
En el Apéndice, se discuten varios artículos sobre la detección de malware en una red utilizando modelos de aprendizaje profundo. La detección de malware también es un aspecto importante.


\citep{xu2023machine} : INGLES Mucho texto 2 conferencia de no se que

\citep{alfaries2019advances} : INLES Ciberseguridad pero mucha basura. Filtrar.
\begin{enumerate}
\item Comparison of Supervised and Unsupervised Fraud Detection 
\item Cybersecurity: Design and Implementation of an Intrusion Detection
and Prevention System.
\end{enumerate}


\citep{kalash2018malware} : Articulo en ingles del review. Es el caso de deteccionde malware usando CNN con el dataset de Microsoft. Te visualiza y explica como pasar de los datos a imagenes. Además te dice como sería el modelo de CNN con las capas y dimensiones. Por ultimo te dice la informacion necesaria como el indice de aprendizaje y las epochs necesarias para obtener el 98,7 \% de acurrancy

\end{comment}













  
\newpage

\medskip
\nocite{*}
\bibliographystyle{plain}
\bibliography{biblio}
\clearpage

\newpage

\printglossary

\newpage

\appendix

\chapter{Anexo A} \label{anexoa}

Para determinar estos pesos, deben calcularse usando la siguiente fórmula:

\[
W_j = \sqrt{\frac{s}{c \cdot S_j}}
\]

Donde $W_j$ representa el peso de la clase $j$, $s$ es el número total de muestras, $c$ es el número total de clases y $S_j$ es el número de muestras de la clase $j$. Estos pesos se emplean durante el cálculo de la función de pérdida. Una vez que se ha calculado la pérdida para cada clase, esta se multiplica por el peso correspondiente de la clase. Este procedimiento permite ``balancear'' la importancia relativa de cada clase durante el entrenamiento de la red.

\chapter{Anexo B} \label{anexob}

\begin{table}[H]
\centering
\begin{tabular}{|c|c|c|c|c|}
\hline
Clase & Precisión & TPR (Sensibilidad) & FPR & F1 Score \\ \hline
1 & 0.94 & 0.97 & 0.06 & 0.96 \\ \hline
2 & 0.92 & 0.97 & 0.07 & 0.94 \\ \hline
3 & 0.97 & 0.99 & 0.03 & 0.98 \\ \hline
4 & 0.91 & 1.00 & 0.09 & 0.95 \\ \hline
5 & 0.49 & 0.33 & 0.06 & 0.39 \\ \hline
6 & 0.94 & 0.90 & 0.03 & 0.92 \\ \hline
7 & 0.95 & 0.95 & 0.05 & 0.95 \\ \hline
8 & 0.88 & 0.90 & 0.12 & 0.89 \\ \hline
9 & 0.94 & 0.93 & 0.06 & 0.94 \\ \hline
\end{tabular}
\caption{Métricas de rendimiento por clase}
\label{tab:metrics}
\end{table}

\end{document}


