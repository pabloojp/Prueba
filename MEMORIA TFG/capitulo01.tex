\chapter{Introducción} \label{Capitulo 1}
      



\section{Motivación y objetivos del trabajo} \label{Sec:1_1}

    
\section{Contexto y antecedentes del trabajo} \label{Sec:1_2}

\subsection{Redes neuronales} \label{Subsubsec: 1_2_1}
  
\subsection{Importancia de la detección y prevención de ataques} \label{Subsec: 1_2_2}

Destaca la importancia crítica de la detección y prevención de ataques cibernéticos en entornos empresariales y gubernamentales, así como en la protección de datos sensibles y la infraestructura crítica.

\subsection{Evolución de las amenazas cibernéticas} \label{Subsec: 1_2_3}

Describe brevemente cómo han evolucionado las amenazas en el ámbito de la ciberseguridad a lo largo del tiempo, desde virus simples hasta ataques sofisticados como el ransomware y el phishing.

\subsection{Avances en el aprendizaje automático para ciberseguridad} \label{Subsubsec: 1_2_4}

Proporciona una visión general de cómo los algoritmos de aprendizaje automático han revolucionado el campo de la ciberseguridad, permitiendo la detección temprana de amenazas, el análisis de comportamiento anómalo y la automatización de respuestas.

\section{Metodología} \label{Subsubsec: 1_3}
\section{Estructura de la memoria} \label{Subsubsec: 1_4}
  
El entorno de hardware en el que he realizado todos los experimentos es un servidor proporcionado por la facultad de informática de la Universidad Complutense de Madrid llamado Simba. Tiene un sistema operativo Debian 12.2 con Linux version 6.1.0-17-amd64 con memoria RAM disponible de 128 GB. La CPU utilizada es un Intel(R) Xeon(R) W-2235 CPU con 3.8 GHz con 6 núcleos. 


