\chapter{Introducción} \label{Capitulo 1}
      
En la última década, la inteligencia artificial (IA) se ha convertido en un tema popular tanto dentro como fuera de la comunidad científica. Una abundancia de artículos en revistas tecnológicas y no tecnológicas han cubierto los temas de aprendizaje automático (o \textit{machine learning}), aprendizaje profundo (o \textit{deep learning}) e inteligencia artificial (\textit{IA}). Sin embargo, todavía existe confusión en torno a estos tres términos. Aunque están relacionados, no son intercambiables.

En 1956, un grupo de informáticos propuso que las computadoras podrían ser programadas para pensar y razonar. Esta visión fue expresada en la conferencia de 1956 \citep{moor2006dartmouth} como:``que cada aspecto del aprendizaje o cualquier otra característica de la inteligencia podría, en principio, ser descrito tan precisamente que una máquina podría simularlo'' . Describieron este principio como ``Inteligencia Artificial''. En pocas palabras, la IA es un campo enfocado en automatizar tareas intelectuales que normalmente realizan los humanos. Principalmente lo que diferencia la Inteligencia Artificial de otros programas informáticos es que no hace falta modificarla para cada escenario particular. Se puede entrenar a las máquinas para que ellas solas 


\section{Motivación y objetivos del trabajo} \label{Sec:1_1}

    
\section{Contexto y antecedentes del trabajo} \label{Sec:1_2}

\subsection{Redes neuronales} \label{Subsubsec: 1_2_1}
  
\subsection{Importancia de la detección y prevención de ataques} \label{Subsec: 1_2_2}

Destaca la importancia crítica de la detección y prevención de ataques cibernéticos en entornos empresariales y gubernamentales, así como en la protección de datos sensibles y la infraestructura crítica.

\subsection{Evolución de las amenazas cibernéticas} \label{Subsec: 1_2_3}

Describe brevemente cómo han evolucionado las amenazas en el ámbito de la ciberseguridad a lo largo del tiempo, desde virus simples hasta ataques sofisticados como el ransomware y el phishing.

\subsection{Avances en el aprendizaje automático para ciberseguridad} \label{Subsubsec: 1_2_4}

Proporciona una visión general de cómo los algoritmos de aprendizaje automático han revolucionado el campo de la ciberseguridad, permitiendo la detección temprana de amenazas, el análisis de comportamiento anómalo y la automatización de respuestas.

\section{Metodología} \label{Subsubsec: 1_3}
\section{Estructura de la memoria} \label{Subsubsec: 1_4}
  
El entorno de hardware en el que he realizado todos los experimentos es un servidor proporcionado por la facultad de informática de la Universidad Complutense de Madrid llamado Simba. Tiene un sistema operativo Debian 12.2 con Linux version 6.1.0-17-amd64 con memoria RAM disponible de 128 GB. La CPU utilizada es un Intel(R) Xeon(R) W-2235 CPU con 3.8 GHz con 6 núcleos. 


