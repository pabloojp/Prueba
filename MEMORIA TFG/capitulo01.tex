\chapter{Introducción} \label{Capitulo 1}
      



\section{Motivación y objetivos del trabajo} \label{Sec:1_1}

    
\section{Contexto y antecedentes del trabajo} \label{Sec:1_2}

\subsection{Redes neuronales} \label{Subsubsec: 1_2_1}
  
\subsection{Importancia de la detección y prevención de ataques} \label{Subsec: 1_2_2}

Destaca la importancia crítica de la detección y prevención de ataques cibernéticos en entornos empresariales y gubernamentales, así como en la protección de datos sensibles y la infraestructura crítica.

\subsection{Evolución de las amenazas cibernéticas} \label{Subsec: 1_2_3}

Describe brevemente cómo han evolucionado las amenazas en el ámbito de la ciberseguridad a lo largo del tiempo, desde virus simples hasta ataques sofisticados como el ransomware y el phishing.

\subsection{Avances en el aprendizaje automático para ciberseguridad} \label{Subsubsec: 1_2_4}

Proporciona una visión general de cómo los algoritmos de aprendizaje automático han revolucionado el campo de la ciberseguridad, permitiendo la detección temprana de amenazas, el análisis de comportamiento anómalo y la automatización de respuestas.


\section{Estructura de la memoria} \label{Subsubsec: 1_3}
  


\section{Contribuciones} \label{Subsec: 1_4}
